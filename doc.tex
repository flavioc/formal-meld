\documentclass[9pt]{article}
% \documentclass[11pt,twoside]{article}

\usepackage{graphicx}
\usepackage{textcomp}
\usepackage{comment}
\usepackage{proof-dashed}
\usepackage{url}
\usepackage{amsmath}
\usepackage{turnstile}
\usepackage{amsthm}
\usepackage[in]{fullpage}

\usepackage{latexsym}
\usepackage{amssymb}            % for \multimap (-o)
\usepackage{stmaryrd}           % for \binampersand (&), \bindnasrepma (\paar)

\newcommand{\m}[1]{\mathsf{#1}}
\newcommand{\f}[1]{\framebox{#1}}

\newcommand{\eph}{\mathit{eph}}
\newcommand{\pers}{\mathit{pers}}
\newcommand{\um}[1]{\underline{\m{#1}}}

\newcommand{\seq}{\vdash}
\newcommand{\semi}{\mathrel{;}}
\newcommand{\lequiv}{\mathrel{\dashv\vdash}}

% symbols of linear logic
\newcommand{\lolli}{\multimap}
\newcommand{\tensor}{\otimes}
\newcommand{\with}{\mathbin{\binampersand}}
\newcommand{\paar}{\mathbin{\bindnasrepma}}
\newcommand{\one}{\mathbf{1}}
\newcommand{\zero}{\mathbf{0}}
\newcommand{\bang}{{!}}
\newcommand{\whynot}{{?}}
\newcommand{\bilolli}{\mathrel{\raisebox{1pt}{\ensuremath{\scriptstyle\circ}}{\lolli}}}
% \oplus, \top, \bot



\title{Meld 2.0 Semantics}
\author{Flavio Cruz}

\begin{document}

\newcommand{\defeq}{\buildrel\triangle\over =}
\newcommand{\trnstile}{\sststile{}{}}
\newcommand{\typ}[1]{\m{#1} \; \m{typ}}
\newcommand{\typelist}[1]{\m{#1} \; \m{type \; list}}
\newcommand{\eexpr}[2]{#1:\m{#2}}
\newcommand{\aexp}[4]{#1;#2 \sststile{}{} \eexpr{#3}{#4}}
\newcommand{\expr}[3]{\aexp{\Psi}{#1}{#2}{#3}}
\newcommand{\tab}[0]{\;\;\;\;}
\newcommand{\elet}[3]{\m{let} \; #1 \; = \; #2 \; \m{in} \; #3 \; \m{end}}
\newcommand{\const}[2]{\m{const}(\mathit{#1}, #2)}
\newcommand{\getconst}[1]{\m{getconst}(\mathit{#1})}
\newcommand{\external}[2]{\m{external}(\mathit{#1}, #2)}
\newcommand{\callexternal}[2]{\m{callexternal}(\mathit{#1}, #2)}
\newcommand{\fun}[3]{\m{fun}(\mathit{#1}, #2, #3)}
\newcommand{\callfun}[2]{\m{callfun}(\mathit{#1}, #2)}
\newcommand{\decl}[2]{\m{decl} \; #1 \; [#2]}
\newcommand{\val}[2]{\m{val} \; #1 : \m{#2}}
\newcommand{\declconst}[3]{\const{#1}{#2} \; \m{of} \; #3}
\newcommand{\declfun}[4]{\fun{#1}{#2}{#3} \; \m{of} \; #4}
\newcommand{\eval}[2]{\Psi \; ; \; #1 \rightarrow #2}
\newcommand{\constraint}[1]{\m{constraint} \; #1}
\newcommand{\fact}[3]{#1[@#2](#3)}
\newcommand{\mif}[3]{\m{if} \; #1 \; \m{then} \; #2 \; \m{else} \; #3 \; \m{end}}
\newcommand{\mrule}[4]{\Psi ; #1 ; #2 \vdash_{#4} #3 \; \m{rule}}
\newcommand{\mrulebody}[4]{\Psi ; #1 ; #2 ; #3 \vdash #4 \; \m{body}}
\newcommand{\mrulehead}[4]{\Psi ; #1 ; #2 \vdash_{#4} #3 \; \m{head}}
\newcommand{\mrulestart}[1]{\m{rule} \; \Psi \; / \; #1}
\newcommand{\comp}[0]{\m{comp} \;}
\newcommand{\aggregate}[4]{[\m{#1} ; #2 ; #3 \Rightarrow #4]}
\newcommand{\aggregatetype}[3]{[\m{#1}] \; / \; #2 \rightsquigarrow #3}
\newcommand{\aggregatestart}[2]{[\m{#1}] \hookrightarrow #2}
\newcommand{\aggregateop}[4]{[\m{#1}] \; #2 / #3 \Rightarrow #4}
\newcommand{\changes}[7]{#1 ; #2 ; #3 ; #4 \Rightarrow #5 ; #6 ; #7}
\newcommand{\changesb}[7]{#1 ; #2 ; #3 ; #4 \Rightarrow #5 ; #6 ; #7}
\newcommand{\apply}[6]{\m{apply} \; #1 ; #2 ; #3 \rightarrow #4 ; #5 ; #6}
\newcommand{\applyb}[6]{\m{apply} \; #1 ; #2 ; #3 \rightarrow #4 ; #5 ; #6}
\newcommand{\derive}[8]{\m{derive} \; \Psi ; #1 ; #2 ; #3 ; #4 ; #5 \rightarrow #6 ; #7 ; #8}
\newcommand{\deriveb}[9]{\m{derive} \; #1 ; #2 ; #3 ; #4 ; #5 ; #6 \rightarrow #7 ; #8 ; #9}
\newcommand{\match}[5]{\m{match} \; #1 ; #2 ; #3 \rightarrow #4 ; #5}
\newcommand{\equal}[2]{#1 = #2}
\newcommand{\at}[2]{#1 \; @ \; #2}
\newcommand{\compr}[1]{\m{def} \; #1}
\newcommand{\comprehension}[1]{\comp #1}
\newcommand{\comprrec}[1]{\m{comp2} \; #1}

\maketitle

\section{Static Semantics}

\subsection{Types}

\[
\infer[\m{addr}]
{\typ{addr}}{}
\tab
\infer[\m{int}]
{\typ{int}}
{}
\tab
\infer[\m{float}]
{\typ{float}}
{}
\tab
\infer[\m{bool}]
{\typ{bool}}{}
\tab
\infer[\m{string}]
{\typ{string}}{}
\]

\[
\infer[\m{list}]
{\typ{list \; \tau}}{\typ{\tau}}
\tab
\infer[\m{struct}]
{\typ{struct \; T}}{\typelist{T}}
\]

\[
\infer[\m{type \; list \; end}]
{\typelist{\cdot}}
\tab
\infer[\m{type \; list}]
{\typelist{\tau ; T}}{\typelist{T}}
\]

\subsection{Expressions}

\[
\infer[\m{addr \; literal}]
{\expr{\Gamma}{\m{addr}(N)}{\m{addr}}}
{}
\]

\[
\infer[\m{int \; literal}]
{\expr{\Gamma}{\m{int}(N)}{int}}{\m{N \; is \; a \; literal}}
\tab
\infer[\m{float \; literal}]
{\expr{\Gamma}{\m{float}(F)}{float}}{\m{F \; is \; a \; float \; literal}}
\]

\[
\infer[\m{string \; literal}]
{\expr{\Gamma}{\m{string}(S)}{string}}{\m{S \; is \; a \; string \; literal}}
\tab
\infer[\m{var}]
{\expr{\Gamma, \eexpr{X}{\tau}}{X}{\tau}}{}
\]

\[
\infer[\m{nil}]
{\expr{\Gamma}{[]}{\m{list \; \tau}}}{\m{\tau}}
\tab
\infer[\m{cons}]
{\expr{\Gamma}{[e_1 \; | \; e_2]}{\m{list \; \tau}}}
{\expr{\Gamma}{e_1}{\m{\tau}} &
   \expr{\Gamma}{e_2}{\m{list \; \tau}}}
\]

\[
\infer[\m{make \; struct}]
{\expr{\Gamma}{\m{struct} \; e_1; ...; e_n}{\m{struct \; \tau_1; ...; \tau_n}}}
{\expr{\Gamma}{e_1}{\tau_1} & ... & \expr{\Gamma}{e_n}{\tau_n}}
\tab
\infer[\m{get \; struct}]
{\expr{\Gamma}{\#i(e)}{\tau_i}}
{\expr{\Gamma}{e}{\m{struct \; \tau_1; ...; \tau_n}}}
\]

\[
\infer[\m{math \; int}]
{\expr{\Gamma}{e_1 \; \m{op} \; e_2}{\m{int}}}
{\expr{\Gamma}{e_1}{\m{int}} & \expr{\Gamma}{e_2}{\m{int}}}
\tab
\infer[\m{math \; float}]
{\expr{\Gamma}{e_1 \; \m{op} \; e_2}{\m{float}}}
{\expr{\Gamma}{e_1}{\m{float}} & \expr{\Gamma}{e_2}{\m{float}}}
\]

\[
\infer[\m{math \; cast1}]
{\expr{\Gamma}{e_1 \; \m{op} \; e_2}{\m{float}}}
{\expr{\Gamma}{e_1}{\m{int}} & \expr{\Gamma}{e_2}{\m{float}}}
\tab
\infer[\m{math \; cast2}]
{\expr{\Gamma}{e_1 \; \m{op} \; e_2}{\m{float}}}
{\expr{\Gamma}{e_1}{\m{float}} & \expr{\Gamma}{e_2}{\m{int}}}
\]

\[
\infer[\m{if}]
{\expr{\Gamma}{\mif{c}{e_1}{e_2}}{\tau}}
{\expr{\Gamma}{c}{\m{bool}} &
   \expr{\Gamma}{e_1}{\m{\tau}} &
      \expr{\Gamma}{e_2}{\m{\tau}}}
\]

\[
\infer[\m{cmp} \; \m{int}]
{\expr{\Gamma}{e_1 \; \m{cmp} \; e_2}{\m{bool}}}
{\expr{\Gamma}{e_1}{\m{int}} &
   \expr{\Gamma}{e_2}{\m{int}}}
\tab
\infer[\m{cmp} \; \m{float}]
{\expr{\Gamma}{e_1 \; \m{cmp} \; e_2}{\m{bool}}}
{\expr{\Gamma}{e_1}{\m{float}} &
   \expr{\Gamma}{e_2}{\m{float}}}
\]

\[
\infer[\m{cmp} \; \m{bool}]
{\expr{\Gamma}{e_1 \; \m{cmp} \; e_2}{\m{bool}}}
{\expr{\Gamma}{e_1}{\m{bool}} &
   \expr{\Gamma}{e_2}{\m{bool}}}
\tab
\infer[\m{cmp} \; \m{string}]
{\expr{\Gamma}{e_1 \; \m{cmp} \; e_2}{\m{bool}}}
{\expr{\Gamma}{e_1}{\m{string}} &
   \expr{\Gamma}{e_2}{\m{string}}}
\]

\[
\infer[\m{cmp} \; \m{addr}]
{\expr{\Gamma}{e_1 \; \m{cmp} \; e_2}{\m{bool}}}
{\expr{\Gamma}{e_1}{\m{addr}} &
   \expr{\Gamma}{e_2}{\m{addr}}}
\]

\[
\infer[\m{or}]
{\expr{\Gamma}{e_1 \; \m{or} \; e_2}{\m{bool}}}
{\expr{\Gamma}{e_1}{\m{bool}} & \expr{\Gamma}{e_2}{\m{bool}}}
\]

\[
\infer[\m{let}]
{\expr{\Gamma}{\elet{X}{e_1}{e_2}}{\tau}}
{\expr{\Gamma}{e_1}{\tau_1} &
   \expr{\Gamma, \eexpr{X}{\tau_1}}{e_2}{\tau}}
\]

\[
\infer[\m{const}]
{\aexp{\Psi, \eexpr{\const{name}{v}}{\tau}}{\Gamma}{\getconst{name}}{\tau}}
{}
\]

\[
\infer[\m{external}]
{\expr{\Gamma}{\callexternal{name}{[e_1, ..., e_n]}}{\tau}}
{\expr{\Gamma}{e_1}{\tau_1} & ... & \expr{\Gamma}{e_n}{\tau_n} &
   \eexpr{\external{name}{[arg_1, ..., arg_n]}}{(\tau_1, ..., \tau_n)\overrightarrow{\tau}} \in \Psi
}
\]

\[
\infer[\m{fun}]
{\expr{\Gamma}{\callfun{name}{[e_1, ..., e_n]}}{\tau}}
{\expr{\Gamma}{e_1}{\tau_1} & ... & \expr{\Gamma}{e_n}{\tau_n} &
   \eexpr{\fun{name}{[arg_1, ..., arg_n]}{e}}{(\tau_1, ..., \tau_n)\overrightarrow{\tau}} \in \Psi
}
\]

\[
\infer[\m{world}]
{\expr{\Gamma}{\m{world}}{\m{int}}}
{}
\tab
\infer[\m{arg}]
{\expr{\Gamma}{\m{arg}(N)}{\m{string}}}
{}
\]

\subsection{Declarations}

\[
\infer[\m{decl}]
{\decl{name}{\m{addr}, \tau_1, ..., \tau_n}}
{\typ{addr} & \typ{\tau_1} & ... & \typ{\tau_n}}
\tab
\infer[\m{\bang decl}]
{\bang\decl{name}{\m{addr}, \tau_1, ..., \tau_n}}
{\typ{addr} & \typ{\tau_1} & ... & \typ{\tau_n}}
\]

\[
\infer[\m{const}]
{\declconst{name}{v}{\tau}}
{\expr{\Gamma}{e}{\tau} & \eval{e}{v} & \val{v}{\tau}}
\]

\[
\infer[\m{fun}]
{\declfun{name}{arg_1 : \tau_1, ..., arg_n : \tau_n}{e}{\tau}}
{\expr{arg_1 : \tau_1, ..., arg_n : \tau_n}{e}{\tau}}
\]

\subsection{Rules}

\[
\infer[\m{rule \; start}]
{\mrulestart{\forall H : \m{addr}. A}}
{\mrule{\cdot}{H}{A}{1}}
\]

\[
\infer[\m{rule \; add \; var}]
{\mrule{\Gamma}{H}{\forall X : \tau. A}{N}}
{\mrule{\Gamma, X : \tau}{H}{A}{N}}
\]

\[
\infer[\m{rule \; body \; head}]
{\mrule{\Gamma}{H}{A \lolli B}{N}}
{\mrulebody{\Gamma, H : \m{addr}}{H}{\Gamma}{A} & \mrulehead{\Gamma, H:\m{addr}}{H}{B}{N}}
\]

\[
\infer[\m{rule \; body \; tensor}]
{\mrulebody{\Gamma}{H}{\Gamma', \Gamma''}{A \otimes B}}
{\mrulebody{\Gamma}{H}{\Gamma''}{B} &
   \mrulebody{\Gamma}{H}{\Gamma'}{A} & }
\]

\[
\infer[\m{rule \; body \; 1}]
{\mrulebody{\Gamma}{H}{\cdot}{1}}
{}
\]

\[
\infer[\m{rule \; body \; exists}]
{\mrulebody{\Gamma}{H}{\Gamma'}{\exists X : \tau. A}}
{\mrulebody{\Gamma, X : \tau}{H}{\Gamma', X : \tau}{A}}
\]

\[
\infer[\m{rule \; body \; fact}]
{\mrulebody{\Gamma, H_{fact} : \m{addr}}{H}{X_1 : \tau_1, ..., X_n : \tau_n}{\fact{name}{H_{fact}}{X_1, ..., X_n}}}
{\decl{name}{\m{addr}, \tau_1, ..., \tau_n} \in \Psi}
\]


\[
\infer[\m{rule \; body \; \bang fact}]
{\mrulebody{\Gamma, H_{fact} : \m{addr}}{H}{X_1 : \tau_1, ..., X_n : \tau_n}{\bang\fact{name}{H_{fact}}{X_1, ..., X_n}}}
{\bang\decl{name}{\m{addr}, \tau_1, ..., \tau_n} \in \Psi}
\]

\[
\infer[\m{rule}]
{\mrulebody{\Gamma}{H}{\cdot}{\bang (\constraint{e})}}
{\expr{\Gamma}{e}{\m{bool}}}
\]

\[
\infer[\m{rule \; head \; tensor}]
{\mrulehead{\Gamma}{H}{A \otimes B}{N}}
{\mrulehead{\Gamma}{H}{A}{N} & \mrulehead{\Gamma}{H}{B}{N}}
\]

\[
\infer[\m{rule \; head \; 1}]
{\mrulehead{\Gamma}{H}{\m{1}}{N}}
{}
\]

\[
\infer[\m{rule \; head \; fact}]
{\mrulehead{\Gamma}{H}{\fact{name}{e}{e_1, ..., e_n}}{N}}
{\expr{\Gamma}{e}{\m{addr}} & \expr{\Gamma}{e_1}{\tau_1} & ... & \expr{\Gamma}{e_n}{\tau_n} &
   \decl{name}{\m{addr}, \tau_1, ..., \tau_n} \in \Psi}
\]

\[
\infer[\m{rule \; head \; \bang fact}]
{\mrulehead{\Gamma}{H}{\bang\fact{name}{e}{e_1, ..., e_n}}{N}}
{\expr{\Gamma}{e}{\m{addr}} & \expr{\Gamma}{e_1}{\tau_1} & ... & \expr{\Gamma}{e_n}{\tau_n} &
   \bang\decl{name}{\m{addr}, \tau_1, ..., \tau_n} \in \Psi}
\]

\[
\infer[\m{rule \; head \; exists}]
{\mrulehead{\Gamma}{H}{\exists X : \m{addr}. A}{N}}
{\mrulehead{\Gamma, X : \m{addr}}{H}{A}{N}}
\]

\[
\infer[\m{rule \; head \; comprehension}]
{\mrulehead{\Gamma}{H}{\comprehension{A}}{1}}
{\mrule{\Gamma}{H}{A}{2}}
\]

\[
\infer[\m{rule \; head \; aggregate}]
{\mrulehead{\Gamma}{H}{\aggregate{Op}{X}{A}{B}}{1}}
{\aggregatetype{Op}{\tau_1}{\tau_2} & \mrule{\Gamma, X : \tau_1}{H}{A \lolli 1}{2} & \mrulehead{\Gamma, X : \tau_2}{H}{B}{2}}
\]

\subsection{Aggregate Types}

\[
\infer[\m{agg \; count}]
{\aggregatetype{count}{\m{int}}{\m{int}}}
{}
\tab
\infer[\m{agg \; collect \; int}]
{\aggregatetype{collect \; int}{\m{int}}{\m{list \; int}}}
{}
\]

\[
\infer[\m{agg \; int \; sum}]
{\aggregatetype{sum \; int}{\m{int}}{\m{int}}}
{}
\]

\[
\infer[\m{agg \; int \; max}]
{\aggregatetype{max \; int}{\m{int}}{\m{int}}}
{}
\tab
\infer[\m{agg \; int \; min}]
{\aggregatetype{min \; int}{\m{int}}{\m{int}}}
{}
\]

\section{Dynamic Semantics}


\subsection{Expression Values}

\[
\infer[\m{int}]
{\val{\m{int}(N)}{int}}
{}
\tab
\infer[\m{bool}]
{\val{\m{bool}(B)}{bool}}
{}
\tab
\infer[\m{float}]
{\val{\m{float}(F)}{float}}
{}
\]

\[
\infer[\m{string}]
{\val{\m{string}(S)}{string}}
{}
\tab
\infer[\m{addr}]
{\val{\m{addr}(A)}{addr}}
{}
\]

\[
\infer[\m{nil}]
{\val{[]}{list \; \tau}}
{}
\tab
\infer[\m{cons}]
{\val{x :: ls}{list \; \tau}}
{\val{x}{\tau} & \val{ls}{list \; \tau}}
\]

\[
\infer[\m{struct}]
{\val{:(v_1, ..., v_n)}{\m{struct \; \tau_1; ...; \tau_n}}}
{\val{v_1}{\tau_1} & ... & \val{v_n}{\tau_n}}
\]

\subsection{Expression Evaluation}

\[
\infer[\m{int}]
{\eval{\m{int}(N)}{\m{int}(N)}}
{}
\tab
\infer[\m{float}]
{\eval{\m{float}(F)}{\m{float}(F)}}
{}
\tab
\infer[\m{addr}]
{\eval{\m{addr}(A)}{\m{addr}(A)}}
{}
\]

\[
\infer[\m{bool}]
{\eval{\m{bool}(B)}{\m{bool}(B)}}
{}
\]

\[
\infer[\m{string}]
{\eval{\m{string}(S)}{\m{string}(S)}}
{}
\tab
\infer[\m{list\; nil}]
{\eval{[]}{[]}}
{}
\tab
\infer[\m{list}]
{\eval{L}{L}}
{}
\]

\[
\infer[\m{cons}]
{\eval{[e_1 | e_2]}{v_1 :: v_2}}
{\eval{e_1}{v_1} & \eval{e_2}{v_2}}
\]

\[
\infer[\m{make \; struct}]
{\eval{\m{struct \; e_1; ...; e_n}}{:(v_1; ...; v_n)}}
{\eval{e_1}{v_1} & ... & \eval{e_n}{v_n}}
\]

\[
\infer[\m{get \; struct}]
{\eval{\m{\#i(e)}}{v_i}}
{\eval{e}{:(v_1; ...; v_i ; ... ; v_n)}}
\]

\[
\infer[\m{math \; int}]
{\eval{e_1 \; \m{op} \; e_2}{\m{int}(V)}}
{\expr{\cdot}{e_1}{\m{int}} & \expr{\cdot}{e_2}{\m{int}} & \eval{e_1}{\m{int}(A)} & \eval{e_2}{\m{int}(B)} &
   V = A \; \m{op} \; B}
\]

\[
\infer[\m{math \; float}]
{\eval{e_1 \; \m{op} \; e_2}{\m{float}(V)}}
{\expr{\cdot}{e_1}{\m{float}} & \expr{\cdot}{e_2}{\m{float}} &
   \eval{e_1}{\m{float}(A)} & \eval{e_2}{\m{float}(B)} &
   V = A \; \m{op} \; B}
\]

\[
\infer[\m{math \; cast1}]
{\eval{e_1 \; \m{op} \; e_2}{\m{float}(V)}}
{\expr{\cdot}{e_1}{\m{int}} & \expr{\cdot}{e_2}{\m{float}} &
   \eval{e_1}{\m{int}(A)} & \eval{e_2}{\m{float}(B)} &
   V = A \; \m{op} \; B}
\]

\[
\infer[\m{math \; cast2}]
{\eval{e_1 \; \m{op} \; e_2}{\m{float}(V)}}
{\expr{\cdot}{e_1}{\m{float}} & \expr{\cdot}{e_2}{\m{int}} &
   \eval{e_1}{\m{float}(A)} & \eval{e_2}{\m{int}(B)} &
   V = A \; \m{op} \; B}
\]

\[
\infer[\m{if \; true}]
{\eval{\mif{c}{e_1}{e_2}}{v_1}}
{\eval{c}{\m{bool}(true)} &
   \eval{e_1}{v_1}}
\tab
\infer[\m{if \; false}]
{\eval{\mif{c}{e_1}{e_2}}{v_2}}
{\eval{c}{\m{bool}(false)} &
   \eval{e_2}{v_2}}
\]

\[
\infer[\m{cmp \; int}]
{\eval{e_1 \; \m{cmp} \; e_2}{\m{bool}(V)}}
{\eval{e_1}{\m{int}(A)} &
   \eval{e_2}{\m{int}(B)} &
   V = A \; \m{cmp} \; B
}
\]

\[
\infer[\m{cmp \; float}]
{\eval{e_1 \; \m{cmp} \; e_2}{\m{bool}(V)}}
{\eval{e_1}{\m{float}(A)} &
   \eval{e_2}{\m{float}(B)} &
   V = A \; \m{cmp} \; B
}
\]

\[
\infer[\m{cmp \; bool}]
{\eval{e_1 \; \m{cmp} \; e_2}{\m{bool}(V)}}
{\eval{e_1}{\m{bool}(A)} &
   \eval{e_2}{\m{bool}(B)} &
   V = A \; \m{cmp} \; B
}
\]

\[
\infer[\m{cmp \; string}]
{\eval{e_1 \; \m{cmp} \; e_2}{\m{bool}(V)}}
{\eval{e_1}{\m{string}(A)} &
   \eval{e_2}{\m{string}(B)} &
   V = A \; \m{cmp} \; B
}
\]

\[
\infer[\m{cmp \; addr}]
{\eval{e_1 \; \m{cmp} \; e_2}{\m{bool}(V)}}
{\eval{e_1}{\m{addr}(A)} &
   \eval{e_2}{\m{addr}(B)} &
   V = A \; \m{cmp} \; B
}
\]

\[
\infer[\m{or}]
{\eval{e_1 \; \m{or} \; e_2}{\m{bool}(V)}}
{\eval{e_1}{\m{bool}(A)} &
   \eval{e_2}{\m{bool}(B)} &
   V = A \; \m{or} \; B
}
\]

\[
\infer[\m{let}]
{\eval{\elet{X}{e_1}{e_2}}{v}}
{\eval{e_1}{v_1} &
   \eval{[v_1/x]e_2}{v}}
\]

\[
\infer[\m{const}]
{\eval{\getconst{name}}{v}}
{\const{name}{v} : \tau \in \Psi}
\]

\[
\infer[\m{external}]
{\eval{\callexternal{name}{[e_1, ..., e_n]}}{v}}
{\begin{array}{ccc}
   \eval{e_1}{v_1} & ... & \eval{e_n}{v_n} \\
   \multicolumn{3}{c}{\eexpr{\external{name}{[arg_1, ..., arg_n]}}{(\tau_1, ..., \tau_n)\overrightarrow{\tau}} \in \Psi} \\
   \multicolumn{3}{c}{v = \callexternal{name}{[v_1, ..., v_n]}} \\
 \end{array}
}
\]

\[
\infer[\m{fun}]
{\eval{\callfun{name}{[e_1, ..., e_n]}}{v}}
{\begin{array}{ccc}
   \eval{e_1}{v_1} & ... & \eval{e_n}{v_n} \\
   \multicolumn{3}{c}{\eexpr{\fun{name}{[arg_1, ..., arg_n]}{e}}{(\tau_1, ..., \tau_n)\overrightarrow{\tau}} \in \Psi} \\
   \multicolumn{3}{c}{\eval{[v_1/\m{arg}_1]...[v_n/\m{arg}_n]e}{v}} \\
   \multicolumn{3}{c}{v = \callfun{name}{v_1, ..., v_n}} \\
 \end{array}
}
\]

\[
\infer[\m{world}]
{\eval{\m{world}}{\m{int}(N)}}
{}
\tab
\infer[\m{arg}]
{\eval{\m{arg}(N)}{\m{string}(S)}}
{}
\]

\subsection{Aggregates}

\[
\infer[\m{init \; count}]
{\aggregatestart{count}{\m{int}(0)}}
{}
\]

\[
\infer[\m{init \; collect \; int}]
{\aggregatestart{collect \; int}{[]}}
{}
\]

\[
\infer[\m{init \; sum}]
{\aggregatestart{sum}{\m{int}(0)}}
{}
\tab
\infer[\m{init \; max}]
{\aggregatestart{max}{\m{int}(-\infty)}}
{}
\tab
\infer[\m{init \; min}]
{\aggregatestart{min}{\m{int}(+\infty)}}
{}
\]

\[
\infer[\m{op \; sum}]
{\aggregateop{sum}{A}{B}{A + B}}
{}
\]

\[
\infer[\m{op \; count}]
{\aggregateop{count}{A}{B}{A + 1}}
{}
\]

\[
\infer[\m{op \; collect \; int}]
{\aggregateop{collect \; int}{A}{B}{[B | A]}}
{}
\]

\[
\infer[\m{op \; min}]
{\aggregateop{min}{A}{B}{\mif{A \leq B}{A}{B}}}
{}
\]

\[
\infer[\m{op \; max}]
{\aggregateop{max}{A}{B}{\mif{A \leq B}{B}{A}}}
{}
\]



\subsection{Global Semantics}


Meaning of variables:

\begin{description}
\item[$\Psi$]: Program state: constants, functions, external functions and declarations.
\item[$\Theta$]: Rules with priority.
\item[$\Phi$]: Rules without priority.
\item[$\Gamma$]: Persistent fact context.
\item[$\Delta$]: Linear fact context.
\end{description}

\[
\infer[\m{rule \; app}]
{\changes{\Psi}{\Theta, \Phi, R}{\Gamma}{\Delta}{\Gamma'}{\Delta'}{\Xi}}
{\apply{\Psi}{\Gamma}{\Delta, [R]}{\Gamma'}{\Delta'}{\Xi}}
\]

\[
\infer[\m{\lolli L}]
{\apply{\Psi}{\Gamma}{\Delta_1, \Delta_2, [A \lolli B]}{\Gamma'}{\Delta'}{\Delta_1, \Xi}}
{\match{\Psi}{\Gamma}{\Delta_1}{[A]}{\Delta_1} &
   \derive{\Gamma}{\Delta_2}{[B]}{\cdot}{\cdot}{\Gamma'}{\Delta'}{\Xi}
}
\]

\[
\infer[\m{\forall L}]
{\apply{\Psi}{\Gamma}{\Delta, [\forall X : \tau. A]}{\Gamma'}{\Delta'}{\Xi}}
{\val{M}{\tau} & \apply{\Psi}{\Gamma}{\Delta, [A\{M/X\}]}{\Gamma'}{\Delta'}{\Xi}}
\]

\subsubsection{Match}

\[
\infer[\m{match \; exists}]
{\match{\Psi}{\Gamma}{\Delta}{[\exists X. A]}{\Xi}}
{\match{\Psi}{\Gamma}{\Delta}{[[M/X]A]}{\Xi}}
\]

\[
\infer[\m{match \; one}]
{\match{\Psi}{\Gamma}{\cdot}{[1]}{\cdot}}
{}
\]

\[
\infer[\m{match \; split}]
{\match{\Psi}{\Gamma}{\Delta, \Delta'}{[A \otimes B]}{\Xi_1, \Xi_2}}
{\match{\Psi}{\Gamma}{\Delta}{[A]}{\Xi_1} &
   \match{\Psi}{\Gamma}{\Delta'}{[B]}{\Xi_2}
}
\]

\[
\infer[\m{match \; end \; constraint}]
{\match{\Psi}{\Gamma}{\cdot}{[\bang\constraint{e}]}{\cdot}}
{\eval{e}{\m{bool}(\m{true})}}
\]

\[
\infer[\m{match \; end \; linear}]
{\match{\Psi}{\Gamma}{\fact{name}{v_1}{v_2, ..., v_n}}{[\fact{name}{v_1}{v_2, ..., v_n}]}{\fact{name}{v_1}{v_2, ..., v_n}}}
{\equal{v_1}{v_1'} & ... & \equal{v_n}{v_n'}}
\]
%% do I need a val here?

\[
\infer[\m{match \; end \; persistent}]
{\match{\Psi}{\Gamma, \bang\fact{name}{v_1}{v_2, ..., v_n}}{\cdot}{[\bang\fact{name}{v_1'}{v_2', ..., v_n'}]}{\cdot}}
{\equal{v_1}{v_1'} & ... & \equal{v_n}{v_n'}}
\]

\[
\infer[\m{equal \; int}]
{\equal{\m{int}(N)}{\m{int}(N)}}
{}
\tab
\infer[\m{equal \; float}]
{\equal{\m{float}(F)}{\m{float}(F)}}
{}
\tab
\infer[\m{equal \; addr}]
{\equal{\m{addr}(A)}{\m{addr}(A)}}
{}
\]

\[
\infer[\m{equal \; string}]
{\equal{\m{string}(S)}{\m{string}(S)}}
{}
\tab
\infer[\m{equal \; bool}]
{\equal{\m{bool}(B)}{\m{bool}(B)}}
{}
\]

\[
\infer[\m{equal \; nil}]
{\equal{[]}{[]}}
{}
\tab
\infer[\m{equal \; cons}]
{\equal{x :: ls}{x' :: ls'}}
{\equal{x}{x'} & \equal{ls}{ls'}}
\]

\subsection{Derive}

\[
\infer[\m{derive \; blur}]
{\derive{\Gamma}{\Delta}{[A]}{\Delta_1}{\Gamma_1}{\Gamma'}{\Delta'}{\Xi}}
{\derive{\Gamma}{\Delta}{A}{\Delta_1}{\Gamma_1}{\Gamma'}{\Delta'}{\Xi}}
\]

\[
\infer[\m{derive \; \otimes}]
{\derive{\Gamma}{\Delta}{A \otimes B, \Omega}{\Delta_1}{\Gamma_1}{\Gamma'}{\Delta'}{\Xi}}
{\derive{\Gamma}{\Delta}{A, B, \Omega}{\Delta_1}{\Gamma_1}{\Gamma'}{\Delta'}{\Xi}}
\]

\[
\infer[\m{derive \; exists}]
{\derive{\Gamma}{\Delta}{\exists X : \m{addr}. A, \Omega}{\Delta_1}{\Gamma_1}{\Gamma'}{\Delta'}{\Xi}}
{\derive{\Gamma}{\Delta}{[x/X]A, \Omega}{\Delta_1}{\Gamma_1}{\Gamma'}{\Delta'}{\Xi} &
   x = \m{new} \; \m{addr}(A)}
\]

\[
\infer[\m{derive \; fact}]
{\derive{\Gamma}{\Delta}{\fact{name}{e_1}{e_2, ..., e_n}, \Omega}{\Delta_1}{\Gamma_1}{\Gamma'}{\Delta'}{\Xi}}
{\begin{array}{ccc}
   \eval{e_1}{v_1} & ... & \eval{e_n}{v_n} \\
   \multicolumn{3}{c}{\derive{\Gamma}{\Delta}{\Omega}{\Delta_1, \fact{name}{v_1}{v_2, ..., v_n}}{\Gamma_1}{\Gamma'}{\Delta'}{\Xi}} \\
   \end{array}
}
\]

\[
\infer[\m{derive \; \bang fact}]
{\derive{\Gamma}{\Delta}{\bang\fact{name}{e_1}{e_2, ..., e_n}, \Omega}{\Delta_1}{\Gamma_1}{\Gamma'}{\Delta'}{\Xi}}
{\begin{array}{ccc}
   \eval{e_1}{v_1} & ... & \eval{e_n}{v_n} \\
   \multicolumn{3}{c}{\derive{\Gamma}{\Delta}{\Omega}{\Delta_1}{\Gamma_1, \bang\fact{name}{v_1}{v_2, ..., v_n}}{\Gamma'}{\Delta'}{\Xi}} \\
   \end{array}
}
\]

\[
\infer[\m{derive \; 1}]
{\derive{\Gamma}{\Delta}{1, \Omega}{\Delta_1}{\Gamma_1}{\Gamma'}{\Delta'}{\Xi}}
{\derive{\Gamma}{\Delta}{\Omega}{\Delta_1}{\Gamma_1}{\Gamma'}{\Delta'}{\Xi}}
\]

\[
\infer[\m{derive \; comprehension}]
{\derive{\Gamma}{\Delta}{\comprehension{A}, \Omega}{\Delta_1}{\Gamma_1}{\Gamma'}{\Delta'}{\Xi}}
{\derive{\Gamma}{\Delta}{1 \with (A \otimes \comprehension{A}), \Omega}{\Delta_1}{\Gamma_1}{\Gamma'}{\Delta'}{\Xi}}
\]


\[
\infer[\m{derive \; \with \; left}]
{\derive{\Gamma}{\Delta}{A \with B, \Omega}{\Delta_1}{\Gamma_1}{\Gamma'}{\Delta'}{\Xi}}
{\derive{\Gamma}{\Delta}{A, \Omega}{\Delta_1}{\Gamma_1}{\Gamma'}{\Delta'}{\Xi}}
\]

\[
\infer[\m{derive \; \with \; right}]
{\derive{\Gamma}{\Delta}{A \with B, \Omega}{\Delta_1}{\Gamma_1}{\Gamma'}{\Delta'}{\Xi}}
{\derive{\Gamma}{\Delta}{B, \Omega}{\Delta_1}{\Gamma_1}{\Gamma'}{\Delta'}{\Xi}}
\]

\newcommand{\aggdef}[4]{\m{agg} \; \m{#1} \; #2 \; #3 \; #4}

\[
\infer[\m{derive \; aggregate}]
{\derive{\Gamma}{\Delta}{\aggregate{Op}{X}{A}{B}, \Omega}{\Delta_1}{\Gamma_1}{\Gamma'}{\Delta'}{\Xi}}
{\aggregatestart{Op}{V} & \derive{\Gamma}{\Delta}{\aggdef{Op}{V}{(x. A(x))}{(y. B(y))}, \Omega}{\Delta_1}{\Gamma_1}{\Gamma'}{\Delta'}{\Xi}}
\]

\[
\infer[\m{derive \; aggregate \; unfold}]
{\derive{\Gamma}{\Delta}{\aggdef{Op}{V'}{(x. A(x))}{(y. B(y))}, \Omega}{\Delta_1}{\Gamma_1}{\Gamma'}{\Delta'}{\Xi}}
{\begin{split}\derive{\Gamma}{\Delta}{B(V) \with (\forall X'. A(X') \lolli \aggdef{Op}{E}{(x. A(x))}{(y. B(y))}), \Omega}{\Delta_1}{\Gamma_1}{\Gamma'}{\Delta'}{\Xi} & \\ \aggregateop{Op}{V'}{X'}{E}& \end{split}
}
\]

\[
\infer[\m{derive \; forall}]
{\derive{\Gamma}{\Delta}{\forall X : \tau. A, \Omega}{\Delta_1}{\Gamma_1}{\Gamma'}{\Delta'}{\Xi}}
{\derive{\Gamma}{\Delta}{[M/X]A, \Omega}{\Delta_1}{\Gamma_1}{\Gamma'}{\Delta'}{\Xi} & \val{M}{\tau}}
\]

\[
\infer[\m{derive \; lolli}]
{\derive{\Gamma}{\Delta}{A \lolli B, \Omega}{\Delta_1}{\Gamma_1}{\Gamma'}{\Delta'}{\Xi, \Xi'}}
{\match{\Psi}{\Gamma}{\Delta}{A}{\Xi'} &
   \derive{\Gamma}{\Delta - \Xi'}{B, \Omega}{\Delta_1}{\Gamma_1}{\Gamma'}{\Delta'}{\Xi}
}
\]

\[
\infer[\m{derive \; end}]
{\derive{\Gamma}{\Delta}{\cdot}{\Delta_1}{\Gamma_1}{\Gamma_1}{\Delta_1}{\cdot}}
{}
\]

\subsection{Local Semantics}



\[
\infer[\m{rule \; app}]
{\at{\changes{\Psi}{\Theta, R}{\Gamma}{\Delta}{\Gamma, \Gamma'}{\Delta', N}}{\pi}}
{\at{\apply{\Psi}{\Gamma}{\Delta, [R]}{\Gamma'}{\Delta'}{\Xi}}{\pi} & N = \Delta - \Xi}
\]

\[
\infer[\m{\lolli L}]
{\at{\apply{\Psi}{\Gamma}{\Delta_1, \Delta_2, [A \lolli B]}{\Gamma'}{\Delta'}{\Xi}}{\pi}}
{\at{\match{\Psi}{\Gamma}{\Delta_1}{[A]}{\Xi}}{\pi} &
   \derive{\Gamma}{\Delta_2}{[B]}{\cdot}{\cdot}{\Gamma'}{\Delta'}{\Xi}}
\]

\[
\infer[\m{\forall L}]
{\at{\apply{\Psi}{\Gamma}{\Delta, [\forall X : \tau. A]}{\Gamma'}{\Delta'}{\Xi}}{\pi}}
{\val{M}{\tau} & \at{\apply{\Psi}{\Gamma}{\Delta, [A\{M/X\}]}{\Gamma'}{\Delta'}{\Xi}}{\pi}}
\]

\subsubsection{Match}

\[
\infer[\m{match \; exists}]
{\at{\match{\Psi}{\Gamma}{\Delta}{[\exists X. A]}{\Xi}}{\pi}}
{\at{\match{\Psi}{\Gamma}{\Delta}{[[M/X]A]}{\Xi}}{\pi}}
\]

\[
\infer[\m{match \; one}]
{\at{\match{\Psi}{\Gamma}{\cdot}{[1]}{\cdot}}{\pi}}
{}
\]

\[
\infer[\m{match \; split}]
{\at{\match{\Psi}{\Gamma}{\Delta, \Delta'}{[A \otimes B]}{\Xi_1, \Xi_2}}{\pi}}
{\at{\match{\Psi}{\Gamma}{\Delta}{[A]}{\Xi_1}}{\pi} &
   \at{\match{\Psi}{\Gamma}{\Delta'}{[B]}{\Xi_2}}{\pi}}
\]

\[
\infer[\m{match \; end \; constraint}]
{\at{\match{\Psi}{\Gamma}{\cdot}{[\bang\constraint{e}]}{\cdot}}{\pi}}
{\eval{e}{\m{bool}(\m{true})}}
\]

\[
\infer[\m{match \; end \; linear}]
{\at{\match{\Psi}{\Gamma}{\fact{name}{v_1}{v_2, ..., v_n}}{[\fact{name}{v_1}{v_2, ..., v_n}]}{\fact{name}{v_1}{v_2, ..., v_n}}}{\pi}}
{\equal{v_1}{v_1'} & ... & \equal{v_n}{v_n'} & v_1 = \m{addr}(\pi)}
\]
%% do I need a val here?

\[
\infer[\m{match \; end \; persistent}]
{\at{\match{\Psi}{\Gamma, \bang\fact{name}{v_1}{v_2, ..., v_n}}{\cdot}{[\bang\fact{name}{v_1'}{v_2', ..., v_n'}]}{\cdot}}{\pi}}
{\equal{v_1}{v_1'} & ... & \equal{v_n}{v_n'} & v_1 = \m{addr}(\pi)}
\]

\begin{comment}
\section{Comprehensions}

Comprehensions have the following syntax:

\begin{verbatim}
body -o {Var-List | CompBody | CompHead}.
\end{verbatim}

We can distinguish between two types of comprehensions:

\subsection{Persistent Only Comprehensions}

These comprehensions only use persistent facts in the body. The head may have linear facts.
Since we only use persistent facts, we are unable to check if we are done with the comprehension just by being unable to do further matchings.
Thus, the only way to check for a stop condition is to verify repeated variables in \texttt{CompBody}.

\begin{verbatim}
body(A) -o {X1, X2, X3 | !a(A, X1), !b(A, X2), !c(A, X3) | CompHead}

// is transformed into
body(A) -o do_comp(A, CommVar1, ..., CommVarN, []).

do_comp(A, CommVar1, ..., CommVarN, L),
!a(A, X1),
!b(A, X2),
!c(A, X3),
(X1, X2, X3) not in L
   -o do_comp(A, CommVar1, ..., CommVarN, [(X1, X2, X3) | L]),
      CompHead.
      
do_comp(A, CommVar1, ..., CommVarN, L) -o 1.
\end{verbatim}

This suffers from a few flaws though. In one hand, we may have several \texttt{!a(A, X1)}, where \texttt{X1} has the same value. With this scheme,
only one \texttt{CompHead} will be derived. On another hand, if \texttt{CompHead} also derives anything that is used in the body of the comprehension, the comprehension may never terminate, so we must constraint \texttt{CompHead} to not include predicates used in the body.

\subsection{Comprehensions with linear facts}

When the comprehension body also contains linear facts we may use another strategy, where we consume all the linear facts to derive all the possible comprehension heads.

\begin{verbatim}
body(A) -o {X1, X2, X3 | !a(A, X1), !b(A, X2), c(A, X3) | CompHead}.

// is transformed into
body(A) -o do_comp(A, CommVar1, ..., CommVarN).

do_comp(A, CommVar1, ..., CommVarN),
!a(A, X1),
!b(A, X2),
c(A, X3)
   -o do_comp(A, CommVar1, ..., CommVarN),
      CompHead.
      
do_comp(A, CommVar1, ..., CommVarN) -o 1.
\end{verbatim}

Of course, we can also use the other approach.

As I said before, problems will arise if \texttt{CompHead} uses predicates from \texttt{CompBody}, because the comprehension may not terminate.
\end{comment}

\subsection{Extending Linear Logic with Comprehensions}

\[
\m{comp} \; A \; B \defeq 1 \; \& \; ((\forall X. A \lolli B) \otimes \m{comp} \; A \; B)
\]

\[
\m{agg} \; V \; A \; C \defeq C \; \& \; (\forall X. A \lolli \m{agg} \; (X + V) \; A \; C)
\]

An example from Meld:

\begin{verbatim}
a(H) -o [sum => S | B | !edge(H, B), !weight(H, B, S) | total(H, S)].

a(H) -o agg1(H, 0).

agg1(H, V) := total(H, V) &
             (forall B, S. !edge(H, B), !weight(H, B, S) -o agg1(H, S + V)).
\end{verbatim}


These would be the left and right rules for definitions:

\[
\infer[\m{def} \; L]
{\Delta, \compr{A'} \trnstile C}
{
   \Delta, B\theta \trnstile C & A \defeq B & A' \doteq A\theta
}
\]

\[
\infer[\m{def} \; R]
{\Delta \trnstile \compr{A'}}
{\Delta \trnstile B \theta & A \defeq B & A' \doteq A\theta}
\]

Identity expansion:

\[
\infer[\m{def} \; R]
{\compr{A'} \trnstile \compr{A'}}
{
   \infer[\m{def} \; L]
   {
      \compr{A'} \trnstile B\theta
   }
   {
      \infer[\m{id}]
      {B \theta \trnstile B \theta}
      {
      }
      & A \defeq B & A' \doteq A\theta
   }
   & A \defeq B & A' \doteq A \theta
}
\]

Cut reduction:

\[
\infer[\m{cut}]
{\Delta \trnstile C}
{
   \infer[\m{def} \; R]
   {
      \Delta \trnstile \compr{A'}
   }
   {
      \Delta \trnstile B\theta & A \defeq B & A' \doteq A'\theta
   }
   &
   \infer[\m{def} \; L]
   {
      \Delta, \compr{A'} \trnstile C
   }
   {
      \Delta, B\theta \trnstile C & A \defeq B & A'\doteq A\theta
   }
}
\]

Reduces to:

\[
\infer[\m{cut}]
{\Delta \trnstile C}
{\Delta, B\theta \trnstile C
   &
   \Delta \trnstile B\theta
}
\]

\begin{comment}
\section{Aggregates}

Aggregates have the following syntax form:

\begin{verbatim}
body(A) -o [op => F | Var-List | CompBody(A) | CompHead(A, F)}.
\end{verbatim}

Like comprehensions, we may distinguish between two types of aggregates.

\subsection{Persistent Only Aggregates}

For these types of aggregates we only use persistent facts in the body.
The transformation verifies that the variable combination has not been tried before and then applies the operator function.

\begin{verbatim}
body(A) -o [sum => W | B | !edge(A, B, W) | total(A, W)].

// is transformed into

body(A) -o do_aggregate(A, CommVar1, ..., CommVarN, 0, []).

do_aggregate(A, CommVar1, ..., CommVarN, Sum, L),
!edge(A, B, W),
(B, W) not in L
   -o do_aggregate(A, CommVar1, ..., CommVarN, Sum + W, [(B, W) | L]).
   
do_aggregate(A, CommVar1, ..., CommVarN, Sum, L) -o total(A, Sum).
\end{verbatim}

\subsection{Aggregates with linear facts}

In this case, the aggregate uses linear facts. We don't need to restrict the predicates used in the body/head since there's only a body.

\begin{verbatim}
body(A) -o [sum => W | B | !edge(A, B), weight(A, B, W) | total(A, W)].

// is transformed into

body(A) -o do_aggregate(A, CommVar1, ..., CommVarN, 0).

do_aggregate(A, CommVar1, ..., CommVarN, Sum),
!edge(A, B),
weight(A, B, W)
   -o do_aggregate(A, CommVar1, ..., CommVarN, Sum + W).
   
do_aggregate(A, CommVar1, ..., CommVarN, Sum) -o total(A, Sum).
\end{verbatim}
\end{comment}

\section{Linear Logic}


\newcommand{\sequent}[3]{#1 ; #2 \vdash #3}
\newcommand{\seqnocut}[3]{#1 ; #2 \Rightarrow #3}

\[
\infer[\one R]
{\Psi ; \sequent{\Gamma}{\cdot}{\one}}
{}
\tab
\infer[\one L]
{\Psi ; \sequent{\Gamma}{\Delta, \one}{C}}
{\Psi ; \sequent{\Gamma}{\Delta}{C}}
\]

\[
\infer[\with R]
{\Psi ; \sequent{\Gamma}{\Delta}{A \with B}}
{\Psi ; \sequent{\Gamma}{\Delta}{A} & \sequent{\Gamma}{\Delta}{B}}
\tab
\infer[\with L_1]
{\Psi ; \sequent{\Gamma}{\Delta, A \with B}{C}}
{\Psi ; \sequent{\Gamma}{\Delta, A}{C}}
\tab
\infer[\with L_2]
{\Psi ; \sequent{\Gamma}{\Delta, B \with B}{C}}
{\Psi ; \sequent{\Gamma}{\Delta, B}{C}}
\]

\[
\infer[\otimes R]
{\Psi ; \sequent{\Gamma}{\Delta, \Delta'}{A \otimes B}}
{\Psi ; \sequent{\Gamma}{\Delta}{A} & \sequent{\Gamma}{\Delta}{B}}
\tab
\infer[\otimes L]
{\Psi ; \sequent{\Gamma}{\Delta, A \otimes B}{C}}
{\Psi ; \sequent{\Gamma}{\Delta, A, B}{C}}
\]

\[
\infer[\lolli R]
{\Psi ; \sequent{\Gamma}{\Delta}{A \lolli B}}
{\Psi ; \sequent{\Gamma}{\Delta, A}{B}}
\tab
\infer[\lolli L]
{\Psi ; \sequent{\Gamma}{\Delta, \Delta', A \lolli B}{C}}
{\Psi ; \sequent{\Gamma}{\Delta}{A} &
   \Psi ; \sequent{\Gamma}{\Delta', B}{C}}
\]

\[
\infer[\forall R]
{\Psi ; \sequent{\Gamma}{\Delta}{\forall n:\tau. A}}
{\Psi, m:\tau ; \sequent{\Gamma}{\Delta}{A\{m/n\}}}
\tab
\infer[\forall L]
{\Psi ; \sequent{\Gamma}{\Delta, \forall n:\tau. A}{C}}
{\Psi \vdash M : \tau & \Psi ; \sequent{\Gamma}{\Delta, A\{M/n\}}{C}}
\]

\[
\infer[\exists R]
{\Psi \; \sequent{\Gamma}{\Delta}{\exists n: \tau. A}}
{\Psi \vdash M : \tau &
   \Psi \; \sequent{\Gamma}{\Delta}{A \{M/n\}}}
\tab
\infer[\exists L]
{\Psi ; \sequent{\Gamma}{\Delta, \exists n:\tau. A}{C}}
{\Psi, m:\tau ; \sequent{\Gamma}{\Delta, A\{m/n\}}{C}}
\]

\[
\infer[\bang R]
{\Psi ; \sequent{\Gamma}{\cdot}{\bang A}}
{\Psi ; \sequent{\Gamma}{\cdot}{A}}
\tab
\infer[\bang L]
{\Psi ; \sequent{\Gamma}{\Delta, \bang A}{C}}
{\Psi ; \sequent{\Gamma, A}{\Delta}{C}}
\tab
\infer[\m{copy}]
{\Psi ; \sequent{\Gamma, A}{\Delta}{C}}
{\Psi ; \sequent{\Gamma, A}{\Delta, A}{C}}
\]

\[
\infer[\m{def} \; R]
{\Psi ; \sequent{\Gamma}{\Delta}{\compr{A'}}}
{\Psi ; \sequent{\Gamma}{\Delta}{B\theta} &
 A \defeq B & A' \doteq A\theta}
\tab
\infer[\m{def} \; L]
{\Psi ; \sequent{\Gamma}{\Delta, \compr{A'}}{C}}
{
   \Psi ; \sequent{\Gamma}{\Delta, B\theta}{C} & A \defeq B & A' \doteq A\theta
}
\]

\[
\infer[\m{cut}]
{\Psi ; \sequent{\Gamma}{\Delta, \Delta'}{C}}
{\Psi ; \sequent{\Gamma}{\Delta}{A} &
   \Psi ; \sequent{\Gamma}{\Delta', A}{C}}
\tab
\infer[\m{cut}\bang]
{\Psi ; \sequent{\Gamma}{\Delta}{C}}
{\Psi ; \sequent{\Gamma}{\cdot}{A} &
   \Psi ; \sequent{\Gamma, A}{\Delta}{C}}
\]

\subsection{Cut Free System}

\[
\infer[\one R]
{\Psi ; \seqnocut{\Gamma}{\cdot}{\one}}
{}
\tab
\infer[\one L]
{\Psi ; \seqnocut{\Gamma}{\Delta, \one}{C}}
{\Psi ; \seqnocut{\Gamma}{\Delta}{C}}
\]

\[
\infer[\with R]
{\Psi ; \seqnocut{\Gamma}{\Delta}{A \with B}}
{\Psi ; \seqnocut{\Gamma}{\Delta}{A} & \seqnocut{\Gamma}{\Delta}{B}}
\tab
\infer[\with L_1]
{\Psi ; \seqnocut{\Gamma}{\Delta, A \with B}{C}}
{\Psi ; \seqnocut{\Gamma}{\Delta, A}{C}}
\tab
\infer[\with L_2]
{\Psi ; \seqnocut{\Gamma}{\Delta, B \with B}{C}}
{\Psi ; \seqnocut{\Gamma}{\Delta, B}{C}}
\]

\[
\infer[\otimes R]
{\Psi ; \seqnocut{\Gamma}{\Delta, \Delta'}{A \otimes B}}
{\Psi ; \seqnocut{\Gamma}{\Delta}{A} & \seqnocut{\Gamma}{\Delta}{B}}
\tab
\infer[\otimes L]
{\Psi ;\seqnocut{\Gamma}{\Delta, A \otimes B}{C}}
{\Psi ; \seqnocut{\Gamma}{\Delta, A, B}{C}}
\]

\[
\infer[\lolli R]
{\Psi ; \seqnocut{\Gamma}{\Delta}{A \lolli B}}
{\Psi ; \seqnocut{\Gamma}{\Delta, A}{B}}
\tab
\infer[\lolli L]
{\seqnocut{\Gamma}{\Delta, \Delta', A \lolli B}{C}}
{\Psi ; \seqnocut{\Gamma}{\Delta}{A} &
   \Psi ; \seqnocut{\Gamma}{\Delta', B}{C}}
\]

\[
\infer[\forall R]
{\Psi ; \seqnocut{\Gamma}{\Delta}{\forall n:\tau. A}}
{\Psi, m:\tau ; \seqnocut{\Gamma}{\Delta}{A\{m/n\}}}
\tab
\infer[\forall L]
{\Psi ; \seqnocut{\Gamma}{\Delta, \forall n:\tau. A}{C}}
{\Psi \vdash M : \tau & \Psi ; \seqnocut{\Gamma}{\Delta, A\{M/n\}}{C}}
\]

\[
\infer[\exists R]
{\Psi ; \seqnocut{\Gamma}{\Delta}{\exists n: \tau. A}}
{\Psi \vdash M : \tau &
   \Psi ; \seqnocut{\Gamma}{\Delta}{A \{M/n\}}}
\tab
\infer[\exists L]
{\Psi ; \seqnocut{\Gamma}{\Delta, \exists n:\tau. A}{C}}
{\Psi, m:\tau ; \seqnocut{\Gamma}{\Delta, A\{m/n\}}{C}}
\]

\[
\infer[\bang R]
{\Psi ; \seqnocut{\Gamma}{\cdot}{\bang A}}
{\Psi ; \seqnocut{\Gamma}{\cdot}{A}}
\tab
\infer[\bang L]
{\Psi ; \seqnocut{\Gamma}{\Delta, \bang A}{C}}
{\Psi ; \seqnocut{\Gamma, A}{\Delta}{C}}
\tab
\infer[\m{copy}]
{\Psi ; \seqnocut{\Gamma, A}{\Delta}{C}}
{\Psi ; \seqnocut{\Gamma, A}{\Delta, A}{C}}
\]

\[
\infer[\m{def} \; R]
{\Psi ; \seqnocut{\Gamma}{\Delta}{\compr{A'}}}
{\Psi ; \seqnocut{\Gamma}{\Delta}{B\theta} &
 A \defeq B & A' \doteq A\theta}
\tab
\infer[\m{def} \; L]
{\Psi ; \seqnocut{\Gamma}{\Delta, \compr{A'}}{C}}
{
   \Psi ; \seqnocut{\Gamma}{\Delta, B\theta}{C} & A \defeq B & A' \doteq A\theta
}
\]

\subsection{Cut Elimination Theorem}

If $\seqnocut{\Gamma}{\Delta}{A}$ and $\seqnocut{\Gamma}{\Delta', A}{C}$ then $\seqnocut{\Gamma}{\Delta, \Delta'}{C}$


\begin{comment}
\section{Optimization Ideas}

\subsection{Data vs Control}

\begin{itemize}
   \item Discover facts that work like node fields and actually never go away (only their arguments change).
   \item Discover facts that drive the computation (usually are consumed and then go away).
\end{itemize}

\subsection{JIT Compilation}

Compile most used rules to assembly.

\subsection{Improve indexing}

Do just in time indexing by gathering statistics about indexing.

\subsection{Improve rule engine}

Indexing of the current set of facts needs to be vastly improved.

\subsection{Find consuming chain of linear facts}

Sometimes a linear fact $a$ derives a $b$ that derives a $c$, etc. Once $a$ is derived we know that a set of rules will be run in sequence. We need to prove that this will happen no matter what.

\subsection{Improved fact loading}

Allow compilation of facts to a separate file. Also, load facts faster.

\end{comment}

\section{Low Level Dynamic Semantics}



Low level dynamic semantics handle:

\begin{itemize}
\item Rule priorities.
\item No guessing of values for variables.
\item Maximality for definitions.
\end{itemize}

For the low level semantics, we consider that $\Theta$ (rules with priority)
is an ordered context of rules.

\newcommand{\applyl}[6]{\m{apply1} \; #1 ; #2 ; #3 \rightarrow #4 ; #5 ; #6}

\[
\infer[\m{rule \; app \; priority}]
{\Psi ; R, \Theta; \Phi; \Gamma; \Delta \hookrightarrow \Gamma'; \Delta' ; \Xi'}
{\applyl{\Psi}{\Gamma}{\Delta ; [R] ; (\m{rule}; \Theta; \Phi ; \Delta)}{\Gamma'}{\Delta'}{\Xi'}}
\]

\[
\infer[\m{rule \; app \; no \; priority}]
{\Psi ; \cdot; R, \Phi ; \Gamma ; \Delta \hookrightarrow \Gamma'; \Delta' ; \Xi'}
{\applyl{\Psi}{\Gamma}{\Delta ; [R] ; (\m{rule} ; \Theta; \Phi ; \Delta)}{\Gamma'}{\Delta'}{\Xi'}}
\]

Note that in the following rule, we do not guess the terms for the variables. Instead, we will try to match the variables against the available facts.

\[
\infer[\m{\forall L}]
{\applyl{\Psi}{\Gamma}{\Delta ; [\forall X : \tau. A] ; C}{\Gamma'}{\Delta'}{\Xi}}
{\applyl{\Psi}{\Gamma}{\Delta ; [A\{\m{var}(X)/X\}]; C}{\Gamma'}{\Delta'}{\Xi}}
\]

Once we get to the implication, we pick both body and head with a rule continuations. The continuation context will have the different facts that may be used to apply the rule. The body $A$ is an ordered context.

\[
\infer[\m{\lolli L}]
{\applyl{\Psi}{\Gamma}{\Delta ; [A \lolli B] ; C}{\Gamma'}{\Delta'}{\Xi'}}
{\m{matchbody} \; \Psi;\Gamma; \Delta ; \cdot ; A ; \cdot ; B ; (\cdot, C) \rightarrow \Gamma' ; \Delta' ; \Xi'}
\]

\subsection{Match Body}

This judgment goes through the ordered body context and matches the facts gainst the linear or persistent context. Constraints are put into the context at the right. Once we match everything correctly, we go through the constraint context (note: not an ordered context) and validate each constraint. If a constraint fails or a match fails, we pick the next continuation (body + facts).

For $\exists$, we do the same thing as we did above for $\forall$.

\[
\infer[\m{matchbody \; exists}]
{\m{matchbody} \; \Psi ; \Gamma ; \Delta ; \Xi ; \exists X. A', A ; B ; H ; C \rightarrow \Gamma' ; \Delta' ; \Xi'}
{\m{matchbody} \; \Psi ; \Gamma ; \Delta ; \Xi ; [\m{var}(X)/X]A', [\m{var}(X)/X]A ; [\m{var}(X)/X]B; [\m{var}(X)/X]H ; C \rightarrow \Gamma' ; \Delta' ; \Xi'}
\]


If we get $1$, we just skip it.

\[
\infer[\m{matchbody \; one}]
{\m{matchbody} \; \Psi;\Gamma;\Delta ; \Xi ; 1, A ; B ; H ; C \rightarrow \Gamma' ; \Delta' ; \Xi'}
{\m{matchbody} \; \Psi;\Gamma;\Delta ; \Xi ; A ; B ; H ; C \rightarrow \Gamma' ; \Delta' ; \Xi'}
\]

For $\otimes$ we simply deconstruct the connective and keep both elements in the ordered context.

\[
\infer[\m{matchbody \; split}]
{\m{matchbody} \; \Psi;\Gamma;\Delta; \Xi ; A_1 \otimes A_2, A ; B ; H ; C \rightarrow \Gamma' ; \Delta' ; \Xi'}
{\m{matchbody} \; \Psi;\Gamma;\Delta; \Xi ; A_1, A_2, A ; B ; H ; C \rightarrow \Gamma' ; \Delta' ; \Xi'}
\]

This is the constraint case. We simply move the constraint to the constraint context.

\[
\infer[\m{matchbody \; constraint}]
{\m{matchbody} \; \Psi;\Gamma;\Delta; \Xi ; \bang\constraint{e}, A; B ; H ; C \rightarrow \Gamma' ; \Delta' ; \Xi'}
{\m{matchbody} \; \Psi ; \Gamma; \Delta; \Xi ; A ; \bang\constraint{e}, B ; H ; C \rightarrow \Gamma' ; \Delta' ; \Xi'}
\]

Finally, the linear case! Here we have two cases, either we have facts in the linear context of this type or we don't.

\[
\infer[\m{matchbody \; linear}]
{\m{matchbody} \; \Psi;\Gamma;\Delta, \Delta_f; \Xi ; \fact{name}{e_1}{e_2, ..., e_n}, A; B; H; (C_1, C_2) \rightarrow \Gamma' ; \Delta'; \Xi'}
{\begin{array}{c}
   \Delta_f = \m{list} \; [ \fact{name}{v_1}{v_2, ..., v_n} | Xs] \\
    NC = (\m{body}; \fact{name}{e_1}{e_2, ..., e_n} ; Xs ; \Delta ; \Xi ; A ; B ; H; C_1) \\
    \Delta_1 = \Delta, \Delta_f - \{\fact{name}{v_1}{v_2, ..., v_n}\} \\
   \m{matchfact} \; \Psi;\Gamma; \Delta_1; \fact{name}{v_1}{v_2, ..., v_n}, \Xi ; [v_1, ..., v_n] ; [e_1, ..., e_n]; A ; B ; H; (NC, C_2) \rightarrow \Gamma' ; \Delta'; \Xi' \\
 \end{array}
}
\]

\[
\infer[\m{matchbody \; linear \; empty}]
{\m{matchbody} \; \Psi;\Gamma;\Delta,\Delta_f; \Xi ; \fact{name}{e_1}{e_2, ..., e_n}, A ; B; H; C \rightarrow \Gamma';\Delta';\Xi'}
{\begin{array}{c}
   \Delta_f = [] \\
   \m{cont} \; \Psi; \Gamma ; C \rightarrow \Gamma';\Delta';\Xi'
  \end{array}
}
\]

Persistent facts are very similar, except we don't mess with the linear context.

\[
\infer[\m{matchbody \; persistent}]
{\m{matchbody} \; \Psi;\Gamma, \Gamma_f;\Delta; \Xi ; \bang\fact{name}{e_1}{e_2, ..., e_n}, A; B; H; (C_1, C_2) \rightarrow \Gamma' ; \Delta'; \Xi'}
{\begin{array}{c}
   \Gamma_f = \m{list} \; [\bang\fact{name}{v_1}{v_2, ..., v_n} | Xs] \\
    NC = (\m{body}; \bang\fact{name}{e_1}{e_2, ..., e_n} ; Xs ; \Delta ; \Xi ; A ; B ; H; C_1) \\
   \m{matchfact} \; \Psi;\Gamma, \Gamma_f; \Delta; \Xi ; [v_1, ..., v_n] ; [e_1, ..., e_n]; A ; B ; H; (NC, C_2) \rightarrow \Gamma' ; \Delta'; \Xi' \\
 \end{array}
}
\]

\[
\infer[\m{matchbody \; persistent \; empty}]
{\m{matchbody} \; \Psi;\Gamma, \Gamma_f;\Delta; \Xi ; \bang\fact{name}{e_1}{e_2, ..., e_n}, A ; B; H; C \rightarrow \Gamma';\Delta';\Xi'}
{\begin{array}{c}
   \Gamma_f = [] \\
   \m{cont} \; \Psi; \Gamma ; C \rightarrow \Gamma';\Delta';\Xi'
  \end{array}
}
\]

Now we get to the case where we have no more facts to process. We use $\m{matchconstr}$ to match the required constraints. Note that all constraints will be instantiated at this point, so they can be evaluated.

\[
\infer[\m{matchbody \; end}]
{\m{matchbody} \; \Psi; \Gamma ; \Delta; \Xi; \cdot ; B ; H; C \rightarrow \Gamma'; \Delta'; \Xi'}
{\m{matchconstr} \; \Psi; \Gamma; \Delta; \Xi; B; H; C \rightarrow \Gamma'; \Delta'; \Xi'}
\]

\subsubsection{Match Facts}

The following judgments match the fact from the context with the fact from the rule.

\[
\infer[\m{matchfact \; var}]
{\m{matchfact} \; \Psi;\Gamma;\Delta;\Xi; [v | v_s] ; [\m{var}(X) \| e_s]; A; B; H; C \rightarrow \Gamma'; \Delta'; \Xi'}
{\m{matchfact} \; \Psi;\Gamma;\Delta;\Xi; v_s; [v/\m{var}(X)]e_s; [v/\m{var}(X)]A; [v/\m{var}(X)]B; [v/\m{var}(X)]H; C \rightarrow \Gamma'; \Delta'; \Xi'}
\]

\[
\infer[\m{matchfact \; equal}]
{\m{matchfact} \; \Psi;\Gamma;\Delta;\Xi; [v_1 | v_s] ; [v_2 | e_s]; A; B; H; C \rightarrow \Gamma'; \Delta'; \Xi'}
{\m{matchfact} \; \Psi;\Gamma;\Delta;\Xi; v_s; e_s; A; B; H; C \rightarrow \Gamma'; \Delta'; \Xi' & \equal{v_1}{v_2}}
\]

If they are not equal, we fail and grab the next continuation:

\[
\infer[\m{matchfact \; not \; equal}]
{\m{matchfact} \; \Psi;\Gamma;\Delta;\Xi; [v_1 | v_s] ; [v_2 | e_s]; A; B; H; C \rightarrow \Gamma'; \Delta'; \Xi'}
{\m{cont} \; \Psi ; \Gamma ; C \rightarrow \Gamma'; \Delta'; \Xi' & v_1 \neq v_2}
\]

\[
\infer[\m{matchfact \; done}]
{\m{matchfact} \; \Psi;\Gamma; \Delta; \Xi ; [] ; []; A ; B ; H; C \rightarrow \Gamma' ; \Delta'; \Xi'}
{\m{matchbody} \; \Psi;\Gamma;\Delta;\Xi; A; B; H; C \rightarrow \Gamma'; \Delta'; \Xi'}
\]

\subsubsection{Match Constraints}

If a constraint succeeds, we keep going on.

\[
\infer[\m{matchconstr \; true}]
{\m{matchconstr} \; \Psi;\Gamma;\Delta;\Xi; \bang\constraint{e}, B ; H; C \rightarrow \Gamma'; \Delta'; \Xi'}
{\eval{e}{\m{bool}(\m{true})} & \m{matchconstr} \; \Psi;\Gamma;\Delta;\Xi; B; H; C \rightarrow \Gamma'; \Delta'; \Xi'}
\]

If not, we get a continuation to try another fact.

\[
\infer[\m{matchconstr \; false}]
{\m{matchconstr} \; \Psi;\Gamma;\Delta;\Xi; \bang\constraint{e}, B ; H; C \rightarrow \Gamma'; \Delta'; \Xi'}
{\eval{e}{\m{bool}(\m{false})} & \m{cont} \; \Psi;\Gamma; C \rightarrow \Gamma'; \Delta'; \Xi'}
\]

Once all constraints are validated, we have succeeded in matching the body rule, so we can start deriving new facts.
Note that we get rid of all continuations.

\[
\infer[\m{matchconstr \; end}]
{\m{matchconstr} \; \Psi;\Gamma;\Delta;\Xi; \cdot ; H ; (C_1, (\m{rule}; ...)) \rightarrow \Gamma';\Delta';\Xi'}
{\m{derive1} \; \Psi;\Gamma;\Delta;\Xi; \cdot; \cdot; H ; \cdot \rightarrow \Gamma';\Delta';\Xi'}
\]

The derive continuation is kept however. This way we can return back to the original derivation.

\[
\infer[\m{matchconstr \; end}]
{\m{matchconstr} \; \Psi;\Gamma;\Delta;\Xi; \cdot ; H ; (C_1, (\m{derive}; \Delta''; \Xi''; \Gamma_1 ; \Delta_1 ; K ; \Omega)) \rightarrow \Gamma';\Delta';\Xi'}
{\m{derive1} \; \Psi;\Gamma;\Delta;\Xi; \cdot; \cdot; H ; (\m{derive}; \Delta'' ; \Xi''; \Gamma_1; \Delta_1 ; K; \Omega) \rightarrow \Gamma';\Delta';\Xi'}
\]

\subsubsection{Continuation}

If we have no more fact continuations, we need to get the rule continuation to try another rule.

\[
\infer[\m{cont \; rule}]
{\m{cont} \; \Psi ; \Gamma ; (\cdot , (\m{rule} ; \Theta ; \Phi ; \Delta)) \rightarrow \Gamma'; \Delta'; \Xi'}
{\Psi ; \Theta; \Phi; \Gamma ; \Delta \hookrightarrow \Gamma'; \Delta' ; \Xi'}
\]

... If there is a derive continuation, it means that an aggregate or continuation has failed.

\[
\infer[\m{cont \; comp}]
{\m{cont} \; \Psi ; \Gamma ; (\cdot , (\m{derive}; \Delta ; \Xi; \Gamma_1; \Delta_1; \comprehension{A}; \Omega)) \rightarrow \Gamma'; \Delta'; \Xi'}
{\m{derive1} \; \Psi ; \Gamma ; \Delta ; \Xi ; \Gamma_1 ; \Delta_1 ; \Omega ; \cdot \rightarrow \Gamma' ; \Delta' ; \Xi'}
\]

\[
\infer[\m{cont \; aggregate}]
{\m{cont} \; \Psi ; \Gamma ; (\cdot , (\m{derive}; \Delta ; \Xi; \Gamma_1; \Delta_1; \aggdef{Op}{V}{(x. A(x))}{(y. B(y))}; \Omega)) \rightarrow \Gamma'; \Delta'; \Xi'}
{\m{derive1} \; \Psi ; \Gamma ; \Delta ; \Xi ; \Gamma_1 ; \Delta_1 ; B(V), \Omega; \cdot \rightarrow \Gamma' ; \Delta' ; \Xi'}
\]

If we have a fact continuation but no more facts for that continuation, we fail and continue:

\[
\infer[\m{cont \; body \; fail}]
{\m{cont} \; \Psi ; \Gamma ; ((\m{body} ; \fact{name}{e_1}{e_2, ..., e_n} ; []; \Delta ; \Xi ; A ; B; H; C), C_2) \rightarrow \Gamma'; \Delta'; \Xi'}
{
   \m{cont} \; \Psi ; \Gamma ; (C, C_2) \rightarrow \Gamma'; \Delta'; \Xi'
}
\]

If we have a fact continuation and also more facts, restore the continuation and continue:

\[
\infer[\m{cont \; body \; ok}]
{\m{cont} \; \Psi ; \Gamma ; ((\m{body} ; \fact{name}{e_1}{e_2, ..., e_n} ; [\fact{name}{v_1}{v_2, ..., v_n} | Xs]; \Delta ; \Xi ; A ; B ; H; C_1), C_2) \rightarrow \Gamma'; \Delta'; \Xi'}
{
   \begin{array}{c}
   \Delta_1 = \Delta - {\fact{name}{v_1}{v_2, ..., v_n}}\\
   NC = (\m{body}; \fact{name}{e_1}{e_2, ..., e_n}; \Delta; A; B; H; C_1) \\
   \m{matchfact} \; \Psi ; \Gamma; \Delta_1 ; \fact{name}{v_1}{v_2, ..., v_n}, \Xi ; [v_1, ..., v_n]; [e_1, ..., e_n]; A; B; H; (NC, C_2) \rightarrow \Gamma' ; \Delta'; \Xi'\\
   \end{array}
}
\]

\subsubsection{Derive}

\[
\infer[\m{derive \; \otimes}]
{\m{derive1} \; \Psi; \Gamma ; \Delta ; \Xi ; \Gamma_1; \Delta_1 ; A \otimes B, \Omega ; C \rightarrow \Gamma'; \Delta'; \Xi'}
{\m{derive1} \; \Psi; \Gamma ; \Delta ; \Xi ; \Gamma_1; \Delta_1 ; A, B, \Omega; C \rightarrow \Gamma'; \Delta'; \Xi'}
\]

\[
\infer[\m{derive \; exists}]
{\m{derive1} \; \Psi ; \Gamma ; \Delta ; \Xi; \Gamma_1 ; \Delta_1 ; \exists X : \m{addr}. A, \Omega ; C \rightarrow \Gamma'; \Delta' ; \Xi'}
{\m{derive1} \; \Psi ; \Gamma ; \Delta ; \Xi; \Gamma_1 ; \Delta_1 ; [x/X]A, \Omega ; C \rightarrow \Gamma'; \Delta'; \Xi'
   & x = \m{new} \; \m{addr}(A)}
\]

\[
\infer[\m{derive \; 1}]
{\m{derive1} \; \Psi ; \Gamma ; \Delta ; \Xi; \Gamma_1 ; \Delta_1 ; 1, \Omega ; C \rightarrow \Gamma'; \Delta' ; \Xi'}
{\m{derive1} \; \Psi ; \Gamma ; \Delta ; \Xi; \Gamma_1 ; \Delta_1 ; \Omega ; C \rightarrow \Gamma'; \Delta' ; \Xi'}
\]


\[
\infer[\m{derive \; fact}]
{\m{derive1} \; \Psi ; \Gamma ; \Delta ; \Xi ; \Gamma_1; \Delta_1 ; \fact{name}{e_1}{e_2, ..., e_n}, \Omega ; C \rightarrow \Gamma'; \Delta'; \Xi'}
{\begin{array}{ccc}
   \eval{e_1}{v_1} & ... & \eval{e_n}{v_n} \\
   \multicolumn{3}{c}{\m{derive1} \; \Psi ; \Gamma ; \Delta ; \Xi ; \Gamma_1 ; \Delta_1, \fact{name}{v_1}{v_2, ..., v_n} ; \Omega ; C \rightarrow \Gamma'; \Delta' ; \Xi'} \\
   \end{array}
}
\]

\[
\infer[\m{derive \; \bang fact}]
{\m{derive1} \; \Psi ; \Gamma ; \Delta ; \Xi ; \Gamma_1 ; \Delta_1 ; \bang \fact{name}{e_1}{e_2, ..., e_n}, \Omega ; C \rightarrow \Gamma'; \Delta' \Xi'}
{\begin{array}{ccc}
   \eval{e_1}{v_1} & ... & \eval{e_n}{v_n} \\
   \multicolumn{3}{c}{\m{derive1} \; \Psi ; \Gamma ; \Delta ; \Xi; \Gamma_1, \bang\fact{name}{v_1}{v_2, ..., v_n} ; \Delta_1 ; \Omega ; C \rightarrow \Gamma'; \Delta'; \Xi'} \\
   \end{array}
}
\]

For the comprehension, we define a new continuation for the current state of derivation and call $\m{apply}$ in order to attempt applying the comprehension. Comprehension fails in one of the $\m{cont}$ cases. It succeeds when derive reaches the end and a continuation is in place.


\[
\infer[\m{derive \; comprehension}]
{\m{derive1} \; \Psi ; \Gamma; \Delta;\Xi;\Gamma_1;\Delta_1; \comprehension{A}, \Omega; \cdot \rightarrow \Gamma' ;\Delta'; \Xi'}
{\applyl{\Psi}{\Gamma}{\Delta ; [A] ; (\m{derive1}; \Delta ; \Xi; \Gamma_1; \Delta_1; \comprehension{A} ; \Omega)}{\Gamma'}{\Delta'}{\Xi'}}
\]

We first change the aggregate definition. Note that this only happens at this derivation level (no continuation possible).

\[
\infer[\m{derive \; aggregate}]
{\m{derive1} \; \Psi; \Gamma; \Delta; \Xi; \Gamma_1; \Delta_1; \aggregate{Op}{X}{A}{B}, \Omega; \cdot \rightarrow \Gamma' ; \Delta'; \Xi'}
{\aggregatestart{Op}{V} & \m{derive1} \; \Psi;\Gamma;\Delta;\Xi;\Gamma_1;\Delta_1; \aggdef{Op}{V}{(x. A(x))}{(y. B(y))}, \Omega; \cdot \rightarrow \Gamma'; \Delta'; \Xi'}
\]

When unfolding the aggregate and if there is an aggregate continuation (aggregate has already been applied multiple times), we need to change the definition of the aggregate inside the continuation. Note that the derivation context ($\Omega$) must only contain the aggregate.

\fontsize{8}{9.5}\selectfont
\[
\infer[\m{derive \; aggregate \; unfold}]
{\m{derive1} \; \Psi; \Gamma; \Delta; \Xi; \Gamma_1; \Delta_1; \aggdef{Op}{V'}{(x. A(x))}{(y. B(y))} ; (\m{derive} ; \Delta'' ; \Xi''; \Gamma'_1; \Delta'_1; \aggdef{Op}{V}{(x. A(x))}{(y. B(y))} ; \Omega) \rightarrow \Gamma'; \Delta'; \Xi'}
{\begin{array}{c}
   \applyl{\Psi}{\Gamma}{\Delta ; [\forall X'. A(X') \lolli \aggdef{Op}{E}{(x. A(x))}{(y. B(y))}] ; (\m{derive}; \Delta ; \Xi, \Xi''; \Gamma_1, \Gamma'_1; \Delta_1, \Delta'_1; \aggdef{Op}{V'}{(x. A(x))}{(y. B(y))} ; \Omega)}{\Gamma'}{\Delta'}{\Xi'}\\
   \aggregateop{Op}{V'}{X'}{E} \\
      \end{array}
}
\]

\fontsize{10}{9.5}\selectfont
Otherwise, if we get an aggregate without a continuation:

\fontsize{8}{9.5}\selectfont
\[
\infer[\m{derive \; aggregate \; unfold}]
{\m{derive1} \; \Psi; \Gamma; \Delta; \Xi; \Gamma_1; \Delta_1; \aggdef{Op}{V}{(x. A(x))}{(y. B(y))}, \Omega ; \cdot \rightarrow \Gamma'; \Delta'; \Xi'}
{\begin{array}{c}
   \applyl{\Psi}{\Gamma}{\Delta ; [\forall X'. A(X') \lolli \aggdef{Op}{E}{(x. A(x))}{(y. B(y))}] ; (\m{derive}; \Delta ; \Xi; \Gamma_1; \Delta_1; \aggdef{Op}{V}{(x. A(x))}{(y. B(y))} ; \Omega)}{\Gamma'}{\Delta'}{\Xi'}\\
   \aggregateop{Op}{V}{X'}{E} \\
      \end{array}
}
\]

\fontsize{10}{9.5}\selectfont

If $\m{derive}$ ends and there is a continuation, it means that either the aggregate or comprehension can be reused again.

\[
\infer[\m{derive \; comprehension \; end}]
{\m{derive1} \; \Psi ; \Gamma ; \Delta; \Xi; \Gamma_1; \Delta_1 ; \cdot ; (\m{derive}; \Delta'' ; \Xi''; \Gamma'_1; \Delta'_1; \comprehension{A} ; \Omega) \rightarrow \Gamma' ; \Delta' ; \Xi'}
{
   \m{derive1} \; \Psi ; \Gamma ; \Delta ; \Xi, \Xi''; \Gamma_1, \Gamma'_1; \Delta_1, \Delta'_1; \comprehension{A}, \Omega; \cdot \rightarrow \Gamma'; \Delta'; \Xi'}
\]

This is the axiom that wraps everything up. If no rule is applicable in the system, then there is no valid proof derivation.

\[
\infer[\m{derive \; end}]
{\m{derive1} \; \Psi ; \Gamma ; \Delta; \Xi; \Gamma_1; \Delta_1; \cdot ; \cdot \rightarrow \Gamma_1 ; \Delta_1 ; \Xi}
{}
\]

\newcommand{\mz}{\m{m}_0 \;}
\newcommand{\mo}{\m{m}_1 \;}
\newcommand{\dz}{\m{d}_0 \;}
\newcommand{\done}{\m{d}_1 \;}
\newcommand{\az}{\m{a}_0 \;}
\newcommand{\ao}{\m{a}_1 \;}
\newcommand{\doz}{\m{do}_0 \;}
\newcommand{\doo}{\m{do}_1 \;}
\newcommand{\cont}{\m{cont} \;}
\newcommand{\contc}{\m{contc} \;}
\newcommand{\dc}{\m{dc} \;}

\section{Basic Low Level System With Comprehensions}


\subsection{High Level System}

\subsubsection{Match}

\[
\infer[\mz 1]
{\mz \cdot ; 1 \rightarrow 1}
{}
\tab
\infer[\mz p]
{\mz p ; p \rightarrow p}
{}
\]

\[
\infer[\mz \otimes]
{\mz \Delta_1, \Delta_2 ; A \otimes B \rightarrow \Xi_1, \Xi_2}
{\mz \Delta_1 ; A \rightarrow \Xi_1 & \mz \Delta_2 ; B \rightarrow \Xi_2}
\]

\subsubsection{Derive}

\[
\infer[\dz p]
{\dz \Delta ; \Xi ; \Delta_1 ; p, \Omega \rightarrow \Xi' ; \Delta'}
{\dz \Delta ; \Xi ; p, \Delta_1 ; \Omega \rightarrow \Xi' ; \Delta'}
\]

\[
\infer[\dz \otimes]
{\dz \Delta; \Xi; \Delta_1; A \otimes B, \Omega \rightarrow \Xi'; \Delta'}
{\dz \Delta; \Xi; \Delta_1; A, B, \Omega \rightarrow \Xi'; \Delta'}
\tab
\]

\[
\infer[\dz end]
{\dz \Delta; \Xi'; \Delta'; \cdot \rightarrow \Xi';\Delta'}
{}
\]


\[
\infer[\dz comp]
{\dz \Delta ; \Xi; \Delta_1; \m{comp} A \lolli B, \Omega \rightarrow \Xi';\Delta'}
{\dz \Delta; \Xi; \Delta_1; 1 \with (A \lolli B \otimes \m{comp} A \lolli B), \Omega \rightarrow \Xi';\Delta'}
\]

\[
\infer[\dz \lolli]
{\dz \Delta_a, \Delta_b; \Xi; \Delta_1; A \lolli B, \Omega \rightarrow \Xi';\Delta'}
{\mz \Delta_a; A \rightarrow \Delta_a & \dz \Delta_b ; \Xi; \Delta_a; \Delta_1; B, \Omega \rightarrow \Xi'; \Delta'}
\]

\[
\infer[\dz \with L]
{\dz \Delta; \Xi; \Delta_1; A \with B, \Omega \rightarrow \Xi';\Delta'}
{\dz \Delta; \Xi; \Delta_1; A, \Omega \rightarrow \Xi';\Delta'}
\tab
\infer[\dz \with R]
{\dz \Delta; \Xi; \Delta_1; A \with B, \Omega \rightarrow \Xi';\Delta'}
{\dz \Delta; \Xi; \Delta_1; B, \Omega \rightarrow \Xi';\Delta'}
\]

\subsubsection{Apply}

\[
\infer[\az rule]
{\az \Delta, \Delta''; A \lolli B \rightarrow \Xi'; \Delta'}
{\mz \Delta; A \rightarrow \Delta & \dz \Delta''; \Delta; \cdot ; B \rightarrow \Xi'; \Delta'}
\]

\[
\infer[\doz rule]
{\doz \Delta; R, \Phi \rightarrow \Xi';\Delta'}
{\doz \Delta; R \rightarrow \Xi';\Delta'}
\]

\subsection{Low Level System}

\subsubsection{Match}

\[
\infer[ok]
{\mo \Delta, p ; \Xi; p, \Omega; H; C \rightarrow \Xi'; \Delta'}
{\mo \Delta; \Xi, p; \Omega; H; C \rightarrow \Xi'; \Delta'}
\tab
\infer[fail]
{\mo \Delta; \Xi; p, \Omega; H; C \rightarrow \Xi'; \Delta'}
{p \notin \Delta & \cont C ; H; \Xi'; \Delta'}
\]

\[
\infer[\tensor]
{\mo \Delta; \Xi; A \otimes B, \Omega ; H ; C \rightarrow \Xi'; \Delta'}
{\mo \Delta; \Xi; A, B, \Omega; H; C \rightarrow \Xi';\Delta'}
\]

\[
\infer[\m{rule} \; \m{cont}]
{\mo \Delta; \Xi; \cdot ; H; (\Phi; \Delta'') \rightarrow \Xi'; \Delta'}
{\done \Delta; \Xi; \cdot ; H; \cdot \rightarrow \Xi'; \Delta'}
\]

\subsubsection{Derive}

\[
\infer[p]
{\done \Delta; \Xi; \Delta_1; p, \Omega; C \rightarrow \Xi'; \Delta'}
{\done \Delta; \Xi; p, \Delta_1; \Omega; C \rightarrow \Xi'; \Delta'}
\tab
\infer[1]
{\done \Delta; \Xi; \Delta_1; 1, \Omega; C \rightarrow \Xi';\Delta'}
{\done \Delta; \Xi; \Delta_1; \Omega; C \rightarrow \Xi'; \Delta'}
\]

\[
\infer[\otimes]
{\done \Delta; \Xi; \Delta_1; A \otimes B, \Omega; C \rightarrow \Xi'; \Delta'}
{\done \Delta; \Xi; \Delta_1; A, B, \Omega; C \rightarrow \Xi';\Delta'}
\]

\[
\infer[\m{comp}]
{\done \Delta; \Xi; \Delta_1; \m{comp} \; A \lolli B, \Omega; \cdot \rightarrow \Xi'; \Delta'}
{\ao \Delta; A \lolli B; (\done \Delta; \Xi; \Delta_1; \m{comp} \; A \lolli B, \Omega; \cdot) \rightarrow \Xi'; \Delta'}
\]

\[
\infer[\m{comp \; derivation}]
{\mo \Delta; \Xi; \cdot; H; (\done \Delta''; \Xi''; \Delta_1; \m{comp} \; A \lolli B, \Omega; \cdot) \rightarrow \Xi'; \Delta'}
{\done \Delta; \Xi; \cdot ; H; (\done \Delta''; \Xi''; \Delta_1; \m{comp} \; A \lolli B, \Omega; \cdot) \rightarrow \Xi'; \Delta'}
\]

\[
\infer[\m{end}]
{\done \Delta; \Xi; \Delta_1; \cdot; \cdot \rightarrow \Xi; \Delta_1}
{}
\]

\[
\infer[\m{next \; comp}]
{\done \Delta; \Xi; \Delta_1; \cdot; (\done \Delta''; \Xi''; \Delta''_1; \m{comp} \; A \lolli B, \Omega) \rightarrow \Xi'; \Delta'}
{\done \Delta; \Xi, \Xi''; \Delta_1, \Delta''_1; \m{comp} \; A \lolli B, \Omega; \cdot \rightarrow \Xi'; \Delta'}
\]

\subsubsection{Continuations}

\[
\infer[\m{rule \; fail}]
{\cont (\Phi; \Delta); \Xi'; \Delta'}
{\doo \Delta; \Phi \rightarrow \Xi'; \Delta'}
\]

\[
\infer[\m{comp \; done}]
{\cont (\done \Delta''; \Xi''; \Delta_1; \m{comp} \; A \lolli B, \Omega); \Xi'; \Delta'}
{\done \Delta; \Xi''; \Delta_1; \Omega; \cdot \rightarrow \Xi'; \Delta'}
\]

\subsubsection{Apply}

\[
\infer[\m{apply \; rule}]
{\ao \Delta; A \lolli B; C \rightarrow \Xi'; \Delta'}
{\mo \Delta; \cdot; A; B; C \rightarrow \Xi'; \Delta'}
\]

\[
\infer[\m{pick \; rule}]
{\doo \Delta; R, \Phi \rightarrow \Xi'; \Delta'}
{\ao \Delta; R; (\Phi; \Delta) \rightarrow \Xi';\Delta'}
\]

\subsection{Low level comprehension match succeeds or fails}

If $\mo \Delta'', \Delta_1, ..., \Delta_n; \Xi; \Omega; H; (\done \Delta'''; \Xi''; \Delta_1; \m{comp} \; A \lolli B, \Omega') \rightarrow \Xi'; \Delta'$ then either:

\begin{itemize}
\item $\cont (\done \Delta'''; \Xi''; \Delta_1; \m{comp} \; A \lolli B, \Omega'); \Xi'; \Delta'$ or
\item $\mo \Delta''; \Xi, \Delta_1, ..., \Delta_n' \cdot ; H ; (\done ...) \rightarrow \Xi'; \Delta'$ and $\Omega = \Omega_1, ..., \Omega_n$ where $\mz \Delta_1; \Omega_1 \rightarrow \Delta_1$, ..., $\mz \Delta_n ; \Omega_n \rightarrow \Delta_n$ and $\Delta_1, ..., \Delta_n$ is not empty.
\end{itemize}

It's trivial by induction on the assumption, except the case $p, \Omega$ and $A \otimes B, \Omega$.

\subsection{Low level comprehension gives one match}

If $\mo \Delta'', \Xi''; \cdot; A ; H; (\done \Delta'''; \Xi; \Delta_1; \m{comp} \; A \lolli B, \Omega') \rightarrow \Xi'; \Delta'$ then either

\begin{itemize}
\item $\cont (\done \Delta'''; \Xi; \Delta_1; \m{comp} \; A \lolli B, \Omega'); \Xi'; \Delta'$ or
\item $\mo \Delta'' ; \Xi''; \cdot; H; (\done \Delta'''; \Xi; \Delta_1; \m{comp} \; A \lolli B, \Omega') \rightarrow \Xi'; \Delta'$ and $\mz \Xi''; A \rightarrow \Xi''$, where $\Xi''$ is not empty
\end{itemize}

This follows trivially from the previous theorem.

\subsection{Comprehension head is another derivation theorem}

If $\done \Delta; \Xi; \Delta_1; \Omega'; (\done \Delta''; \Xi''; \Delta'_1; \m{comp} \; A \lolli B, \Omega) \rightarrow \Xi'; \Delta'$ \\ then \\ $\done \Delta; \Xi, \Xi''; \Delta_1, \Delta'_1; \Omega', \m{comp} \; A \lolli B, \Omega; \cdot \rightarrow \Xi'; \Delta'$.

\begin{itemize}
\item $p$

$\done \Delta; \Xi; \Delta_1; p, \Omega''; (\done ...) \rightarrow \Xi'; \Delta'$ \hfill (1) assumption \\
$\Omega' = p, \Omega''$ \hfill (2) from (1) \\
$\done \Delta; \Xi; p, \Delta_1; \Omega''; (\done ...) \rightarrow \Xi'; \Delta'$ \hfill (3) inversion of (1) \\
$\done \Delta; \Xi, \Xi''; p, \Delta_1, \Delta'_1; \Omega'', \m{comp} \; A \lolli B, \Omega; \cdot \rightarrow \Xi'; \Delta'$ \hfill (4) i.h. on (3) \\
$\done \Delta; \Xi, \Xi''; \Delta_1, \Delta'_1; p, \Omega'', \m{comp} \; A \lolli B, \Omega; \cdot \rightarrow \Xi'; \Delta'$ \hfill (5) apply rule on (4) \\

\item $A \otimes B$

$\done \Delta; \Xi, \Xi''; \Delta_1, \Delta'_1; A, B, \Omega''; \m{comp} \; A \lolli B, \Omega; \cdot \rightarrow \Xi'; \Delta'$ \hfill (1) by i.h. \\
$\done \Delta; \Xi, \Xi''; \Delta_1, \Delta'_1; A \otimes B, \Omega''; \m{comp} \; A \lolli B, \Omega; \cdot \rightarrow \Xi'; \Delta'$ \hfill (2) rule application on (1) \\

\item $\cdot$

$\done \Delta; \Xi; \Delta_1; \cdot; (\done \Delta''; \Xi''; \Delta'_1; \m{comp} \; A \lolli B, \Omega) \rightarrow \Xi'; \Delta'$ \hfill (1) assumption \\
$\done \Delta; \Xi, \Xi''; \Delta_1, \Delta'_1; \m{comp} \; A \lolli B, \Omega; \cdot \rightarrow \Xi'; \Delta'$ \hfill (2) inversion of (1) \\

\end{itemize}

\subsection{Low level matching gives high level matching theorem}

If \\
$\mo \Delta, \Delta_1, ..., \Delta_n; \Xi; A_1, ..., A_n; H; \cdot \rightarrow \Xi'; \Delta'$ \\
then \\
$\mz \Delta_1; A_1 \rightarrow \Delta_1$ through $\mz \Delta_n; A_n \rightarrow \Delta_n$ and \\
$\mo \Delta; \Xi, \Delta_1, ..., \Delta_n; \cdot; H; \cdot \rightarrow \Xi'; \Delta'$ and \\
$\Delta_1, ..., \Delta_n$ is not empty if $A_1, ..., A_n$ is not $\cdot$.

Induction on $\Omega = A_1, ..., A_n$.

\begin{itemize}
\item $p, \Omega$ and $p \notin \Delta$

Not applicable.

\item $p, \Omega$

$\mo \Delta, \Delta_1, ..., \Delta_n, p; \Xi; p, A_1, ..., A_n; H; \cdot \rightarrow \Xi'; \Delta'$ \hfill (1) assumption \\
$\mo \Delta, \Delta_1, ..., \Delta_n; \Xi, p; A_1, ..., A_n; H; \cdot \rightarrow \Xi'; \Delta'$ \hfill (2) inversion of (1) \\
$\mz \Delta_1 ; A_1 \rightarrow \Delta_1$ ... $\mz \Delta_n ; A_n \rightarrow \Delta_n$ \hfill (3) induction on (2) \\
$\mo \Delta; \Xi, p, \Delta_1, ..., \Delta_n; \cdot ; H ; \cdot \rightarrow \Xi'; \Delta'$ \hfill (4) induction on (2) \\
$\mz p; p \rightarrow p$ \hfill (5) axiom \\
$p, \Delta_1, ..., \Delta_n$ is not empty\\

\item $A \otimes B, \Omega$

$\mo \Delta, \Delta_1, ..., \Delta_n; \Xi; A \otimes B, A_1, ..., A_{n-2}; H; \cdot \rightarrow \Xi'; \Delta'$ \hfill (1) assumption \\
$\mo \Delta, \Delta_1, ..., \Delta_n; \Xi; A, B, A_1, ..., A_{n-2}; H; \cdot \rightarrow \Xi'; \Delta'$ \hfill (2) inversion of (1) \\
$\mo \Delta; \Xi, \Delta_1, ..., \Delta_n; \cdot ; H; \cdot \rightarrow \Xi'; \Delta'$ \hfill (3) induction on (2) \\
$\mz \Delta_1 ; A \rightarrow \Delta_1$, \; $\mz \Delta_2; B \rightarrow \Delta_2$, \; ... \; $\mz \Delta_n; A_{n-2} \rightarrow \Delta_n$ \hfill (4) induction on (2) \\
$\mz \Delta_1, \Delta_2; A \otimes B \rightarrow \Delta_1, \Delta_2$ \hfill (5) rule on (4) \\
$\mz \Delta_1, \Delta_2; A \otimes B \rightarrow \Delta_1, \Delta_2$ \; ... \; $\mz \Delta_n; A_{n-2} \rightarrow \Delta_n$ \hfill (6) from (5) \\
$\Delta_1, ..., \Delta_n$ is not empty \hfill (7) from induction on (2) \\

\item $\cdot$

$\mo \Delta ; \Xi; \cdot ; H; (\cdot ; \Delta'') \rightarrow \Xi'; \Delta'$ \hfill (1) assumption \\
$n = 0$ \hfill since $\Omega = \cdot$ \\

\end{itemize}

\subsection{Derive soundness}

If $\done \Delta; \Xi; \Delta_1; \Omega; \cdot \rightarrow \Xi'; \Delta'$ then \\
$\dz \Delta; \Xi; \Delta_1; \Omega \rightarrow \Xi'; \Delta'$

By induction first on $\Omega$ and then on the size of $\Delta$.

\begin{itemize}
\item $p, \Omega$

$\done \Delta; \Xi; \Delta_1; p, \Omega; \cdot \rightarrow \Xi'; \Delta'$ \hfill (1) assumption \\
$\done \Delta; \Xi; \Delta_1, p; \Omega; \cdot \rightarrow \Xi'; \Delta'$ \hfill (2) inversion of (1) \\
$\dz \Delta; \Xi; \Delta_1, p; \Omega \rightarrow \Xi'; \Delta'$ \hfill (3) by induction on (2) \\
$\dz \Delta; \Xi; \Delta_1; p, \Omega \rightarrow \Xi'; \Delta'$ \hfill (4) rule application on (3) \\

\item $1, \Omega$

Same as before.

\item $A \otimes B, \Omega$

Same as before.

\item $\comp A \lolli B, \Omega$

$\done \Delta; \Xi; \Delta_1; \comp A \lolli B, \Omega; \cdot \rightarrow \Xi'; \Delta'$ \hfill (1) assumption \\
$\ao \Delta; A \lolli B; (\done \Delta; \Xi; \Delta_1; \comp A \lolli B, \Omega) \rightarrow \Xi'; \Delta'$ \hfill (2) inversion of (1) \\
$\mo \Delta; \cdot; A; B ; (\done \Delta; \Xi; \Delta_1; \comp A \lolli B, \Omega) \rightarrow \Xi'; \Delta'$ \hfill (3) inversion of (2) \\
Using (3) on theorem "Low level comprehension gives one match" we get two subcases:

\begin{itemize}
\item Comprehension fails:

$\cont (\done \Delta; \Xi; \Delta_1; \comp A \lolli B, \Omega); \Xi' \Delta'$ \hfill (4) from theorem \\
$\done \Delta; \Xi; \Delta_1; \Omega; \cdot \rightarrow \Xi'; \Delta'$ \hfill (5) inversion of (4) \\
$\dz \Delta; \Xi; \Delta_1; \Omega \rightarrow \Xi'; \Delta'$ \hfill (6) induction on (5) \\
$\dz \Delta; \Xi; \Delta_1; 1, \Omega \rightarrow \Xi';\Delta'$ \hfill (7) rule on (6) \\
$\dz \Delta; \Xi; \Delta_1; 1 \with (A \lolli B \otimes \comp A \lolli B), \Omega \rightarrow \Xi'; \Delta'$ \hfill (8) rule application on (7) \\
$\dz \Delta; \Xi; \Delta_1; \comp A \lolli B, \Omega \rightarrow \Xi'; \Delta'$ (9) rule application on (8) \\
\item Comprehension succeeds:
$\Delta = \Delta'', \Xi''$ \hfill (4) from theorem \\
$\mo \Delta''; \Xi''; \cdot ; B; (\done \Delta; \Xi; \Delta_1; \comp A \lolli B, \Omega) \rightarrow \Xi'; \Delta'$ \hfill (5) from theorem "Low level comprehension gives one match" \\
$\mz \Xi''; A \rightarrow \Xi''$ \hfill (6) from the same theorem \\
$\Delta''$ is smaller than $\Delta$ \hfill (7) from the same theorem \\
$\done \Delta''; \Xi''; \cdot; B; (\done \Delta; \Xi; \Delta_1; \comp A \lolli B, \Omega) \rightarrow \Xi'; \Delta'$ \hfill (8) inversion of (5) \\
$\done \Delta''; \Xi'', \Xi; \Delta_1; B, \comp A \lolli B, \Omega; \cdot \rightarrow \Xi'; \Delta'$ \hfill (9) using theorem "Comprehension head is another derivation theorem" on (8) \\
$\dz \Delta''; \Xi''; \Xi; \Delta_1; B, \comp A \lolli B, \Omega \rightarrow \Xi'; \Delta'$ \hfill (10) by i.h. on (9) because of (7) \\
$\dz \Xi'', \Delta''; \Xi'; \Delta_1; A \lolli B, \comp A \lolli B, \Omega \rightarrow \Xi'; \Delta'$ \hfill (11) using rule on (10) and (6) \\
$\dz \Xi'', \Delta''; \Xi'; \Delta_1; (A \lolli B) \otimes (\comp A \lolli B), \Omega \rightarrow \Xi'; \Delta'$ \hfill (12) using rule on (11) \\
$\dz \Xi'', \Delta''; \Xi'; \Delta_1; 1 \with ((A \lolli B) \otimes (\comp A \lolli B)), \Omega \rightarrow \Xi'; \Delta'$ \hfill (13) rule on (12) \\
$\dz \Xi'', \Delta''; \Xi'; \Delta_1; \comp A \lolli B, \Omega \rightarrow \Xi'; \Delta'$ \hfill (14) rule on (13) \\

\end{itemize}
\end{itemize}

\subsection{Apply Soundness Theorem}

If $\ao \Delta; R; (\cdot; \Delta) \rightarrow \Xi'; \Delta'$ then \\
      $\az \Delta; R \rightarrow \Xi'; \Delta'$

Case by case analysis:

\begin{itemize}
\item $R = A \lolli B$

$\ao \Delta; A \lolli B ; (\cdot ; \Delta) \rightarrow \Xi'; \Delta'$ \hfill (1) assumption \\
$\mo \Delta_1, \Delta_2; \cdot; A ; B ; (\cdot ; \Delta) \rightarrow \Xi'; \Delta'$ \hfill (2) inversion of (1) \\
$\mo \Delta_2; \Delta_1; \cdot ; B ; (\cdot ; \Delta) \rightarrow \Xi'; \Delta'$ \hfill (3) using theorem "Low level matching gives high level matching theorem" on (2) \\
$\mz \Delta_1 ; A \rightarrow \Delta_1$ \hfill (4) from same theorem on (2) \\
$\do \Delta_2; \Delta_1; \cdot ; B ; (\cdot ; \Delta) \rightarrow \Xi'; \Delta'$ \hfill (5) inversion of (3) \\
$\dz \Delta_2; \Delta_1; \cdot ; B \rightarrow \Xi'; \Delta'$ \hfill (6) derive soundness on (5) \\
$\az \Delta_1, \Delta_2; A \lolli B \rightarrow \Xi'; \Delta'$ \hfill (7) using rule on (6) and (4) \\
\end{itemize}

\subsection{Success or continuation}

If $\mo \Delta; \Xi; \Delta_1; \Omega; H; (\Phi; \Delta'') \rightarrow \Xi'; \Delta'$ then either:

\begin{itemize}
   \item $\cont \; (\Phi; \Delta''); \Xi'; \Delta'$
   \item $\mo \Delta; \Xi; \Delta_1; \Omega; H; (\cdot; \Delta'') \rightarrow \Xi'; \Delta'$
\end{itemize}

Induction on $\Omega$:

\begin{itemize}
\item $p, \Omega$ and $p \in \Omega$

$\mo \Delta, p; \Xi; p, \Omega; H; (\Phi; \Delta'') \rightarrow \Xi'; \Delta'$ \hfill (1) assumption\\
$\mo \Delta; \Xi, p; \Omega; H; (\Phi; \Delta'') \rightarrow \Xi'; \Delta'$ \hfill (2) inversion of (1)\\
$\cont \; (\Phi; \Delta'') ; \Xi'; \Delta'$ or $\mo \Delta; \Xi, p; \Omega; H; (\cdot; \Delta'') \rightarrow \Xi'; \Delta'$ \hfill (3) induction of (2) \\
$\cont \; (\Phi; \Delta'') ; \Xi'; \Delta'$ or $\mo \Delta, p; \Xi; p, \Omega; H; (\cdot; \Delta'') \rightarrow \Xi'; \Delta'$ \hfill (4) rule application on (3) \\

\item $p, \Omega$ and $p \notin \Delta$

$\mo \Delta; \Xi; p, \Omega; H; (\Phi; \Delta'') \rightarrow \Xi'; \Delta'$ \hfill (1) assumption \\
$\cont \; (\Phi; \Delta''); \Xi'; \Delta'$ \hfill (2) inversion of (1) \\

\item $A \otimes B, \Omega$

$\mo \Delta; \Xi; A \otimes B, \Omega; H; (\Phi; \Delta'') \rightarrow \Xi'; \Delta'$ \hfill (1) assumption \\
$\mo \Delta; \Xi; A, B, \Omega; H; (\Phi; \Delta'') \rightarrow \Xi'; \Delta'$ \hfill (2) inversion of (1) \\
$\cont \; (\Phi; \Delta''); \Xi'; \Delta'$ or $\mo \Delta; \Xi; A, B, \Omega; H; (\cdot; \Delta'') \rightarrow \Xi'; \Delta'$ \hfill (3) i.h. on (2) \\
$\cont \; (\Phi; \Delta''); \Xi'; \Delta'$ or $\mo \Delta; \Xi; A \otimes B, \Omega; H; (\cdot; \Delta'') \rightarrow \Xi'; \Delta'$ \hfill (4) rule on (3) \\

\item $\cdot$

$\mo \Delta; \Xi; \cdot; H; (\Phi; \Delta'') \rightarrow \Xi'; \Delta'$ \hfill (1) assumption \\
$\done \Delta; \Xi; \cdot; H; \cdot \rightarrow \Xi'; \Delta'$ \hfill (2) inversion of (1) \\
$\mo \Delta; \Xi; \cdot; H; (\cdot; \Delta'') \rightarrow \Xi'; \Delta'$ \hfill (3) rule application on (2) \\
\end{itemize}

\subsection{One Rule Theorem}

If $\doo \Delta; \Phi \rightarrow \Xi'; \Delta'$ then $\exists R \in \Phi. \doo \Delta; R \rightarrow \Xi'; \Delta'$.

Induction on the size of $\Phi$.

\begin{itemize}
\item $\Phi = \cdot$

Not applicable.

\item $\Phi = R', \Phi'$

$\doo \Delta; R', \Phi' \rightarrow \Xi' \Delta'$ \hfill (1) assumption \\
$\ao \Delta; R'; (\Phi'; \Delta) \rightarrow \Xi'; \Delta'$ \hfill (2) inversion of (1) \\
$\mo \Delta; \cdot; A; B; (\Phi'; \Delta) \rightarrow \Xi'; \Delta'$ \hfill (3) inversion of (2) where $R' = A \lolli B$\\

From theorem "Success or continuation" we have two cases:

   \begin{itemize}
   \item Match Success \\
   $\mo \Delta; \cdot; A; B; (\cdot; \Delta) \rightarrow \Xi'; \Delta'$ \hfill (4) from theorem \\
   $\ao \Delta; A \lolli B; (\cdot; \Delta) \rightarrow \Xi'; \Delta'$ \hfill (5) rule application on (4) \\
   $\doo \Delta; A \lolli B \rightarrow \Xi'; \Delta'$ \hfill (6) rule application on (5) \\
   $\doo \Delta; R \rightarrow \Xi'; \Delta'$ \hfill (7) rewrite of (6) \\

   \item Continuation \\
   $\cont (\Phi'; \Delta); \Xi'; \Delta'$ \hfill (4) from theorem \\
   $\doo \Delta; \Phi' \rightarrow \Xi'; \Delta'$ \hfill (5) inversion of (4) \\
   $\exists R \in \Phi'. \doo \Delta; R \rightarrow \Xi'; \Delta'$ \hfill (6) i.h. on (5) \\
   $\exists R \in \Phi.$ \hfill (7) from (6) \\
   \end{itemize}

\end{itemize}

\section{Low Level System With Matching Continuations}


\subsection{Low Level System}

For this system, we include only linear facts but we use a continuation stack to match facts.

\subsubsection{Match}

\[
\infer[\mo ok \; first]
{\mo \Delta, p_1, \Delta'' ; \Xi; p, \Omega; H; \cdot; R \rightarrow \Xi'; \Delta'}
{\mo \Delta, \Delta''; \Xi, p_1; \Omega; H; (\Delta, p_1; \Delta''; p; \Omega; \Xi; \cdot); R \rightarrow \Xi'; \Delta'}
\]

\[
\infer[\mo ok \; other]
{\mo \Delta, p_1, \Delta'' ; \Xi; p, \Omega; H; C_1, C; R \rightarrow \Xi'; \Delta'}
{\mo \Delta, \Delta''; \Xi, p_1; \Omega; H; (\Delta, p_1; \Delta''; p ; \Omega; \Xi; q, \Lambda), C_1, C; R \rightarrow \Xi'; \Delta' & C_1 = (\Delta_{old}; \Delta'_{old}; \Xi_{old}; q; \Omega_{old}; \Lambda)}
\]

\[
\infer[\mo fail]
{\mo \Delta; \Xi; p, \Omega; H; C; R \rightarrow \Xi'; \Delta'}
{p \notin \Delta & \cont C ; H; R; \Xi'; \Delta'}
\]

\[
\infer[\mo \otimes]
{\mo \Delta; \Xi; A \otimes B, \Omega ; H ; C; R \rightarrow \Xi'; \Delta'}
{\mo \Delta; \Xi; A, B, \Omega; H; C; R \rightarrow \Xi';\Delta'}
\]

\[
\infer[\mo end]
{\mo \Delta; \Xi; \cdot ; H; C; R \rightarrow \Xi'; \Delta'}
{\done \Delta; \Xi; \cdot ; H; \cdot \rightarrow \Xi'; \Delta'}
\]

\subsubsection{Derive}

\[
\infer[\done p]
{\done \Delta; \Xi; \Delta_1; p, \Omega; C \rightarrow \Xi'; \Delta'}
{\done \Delta; \Xi; p, \Delta_1; \Omega; C \rightarrow \Xi'; \Delta'}
\tab
\infer[\done 1]
{\done \Delta; \Xi; \Delta_1; 1, \Omega; C \rightarrow \Xi';\Delta'}
{\done \Delta; \Xi; \Delta_1; \Omega; C \rightarrow \Xi'; \Delta'}
\]

\[
\infer[\done \otimes]
{\done \Delta; \Xi; \Delta_1; A \otimes B, \Omega; C \rightarrow \Xi'; \Delta'}
{\done \Delta; \Xi; \Delta_1; A, B, \Omega; C \rightarrow \Xi';\Delta'}
\]

\[
\infer[\done end]
{\done \Delta; \Xi; \Delta_1; \cdot \rightarrow \Xi; \Delta_1}
{}
\]

\subsubsection{Continuation}

\[
\infer[\cont next \; rule]
{\cont \cdot; H; (\Phi; \Delta); \Xi'; \Delta'}
{\doo \Delta; \Phi \rightarrow \Xi'; \Delta'}
\]

\[
\infer[\cont next]
{\cont (\Delta; p_1, \Delta''; p; \Omega; \Xi; \Lambda), C; H; R; \Xi'; \Delta'}
{\mo \Delta, \Delta''; \Xi, p_1; \Omega; H; (\Delta, p_1; \Delta''; p ; \Omega; \Xi; \Lambda), C; R \rightarrow \Xi'; \Delta'}
\]

\[
\infer[\cont no \; more]
{\cont (\Delta; \cdot; p; \Omega; \Xi; \Lambda), C; H; R; \Xi'; \Delta'}
{\cont C; H; R; \Xi'; \Delta'}
\]

\subsubsection{Apply}

\[
\infer[\ao start]
{\ao \Delta; A \lolli B; R \rightarrow \Xi'; \Delta'}
{\mo \Delta; \cdot; A; B; \cdot; R \rightarrow \Xi'; \Delta'}
\]

\[
\infer[\doo rule]
{\doo \Delta; R, \Phi \rightarrow \Xi'; \Delta'}
{\ao \Delta; R; (\Phi; \Delta) \rightarrow \Xi';\Delta'}
\]


\subsection{High Level System}

\subsubsection{Match}

\[
\infer[\mz 1]
{\mz \cdot ; 1 \rightarrow 1}
{}
\tab
\infer[\mz p]
{\mz p ; p \rightarrow p}
{}
\]

\[
\infer[\mz \otimes]
{\mz \Delta_1, \Delta_2 ; A \otimes B \rightarrow \Xi_1, \Xi_2}
{\mz \Delta_1 ; A \rightarrow \Xi_1 & \mz \Delta_2 ; B \rightarrow \Xi_2}
\]

\subsubsection{Derive}

\[
\infer[\dz p]
{\dz \Delta ; \Xi ; \Delta_1 ; p, \Omega \rightarrow \Xi' ; \Delta'}
{\dz \Delta ; \Xi ; p, \Delta_1 ; \Omega \rightarrow \Xi' ; \Delta'}
\]

\[
\infer[\dz \otimes]
{\dz \Delta; \Xi; \Delta_1; A \otimes B, \Omega \rightarrow \Xi'; \Delta'}
{\dz \Delta; \Xi; \Delta_1; A, B, \Omega \rightarrow \Xi'; \Delta'}
\tab
\]

\[
\infer[\dz end]
{\dz \Delta; \Xi'; \Delta'; \cdot \rightarrow \Xi';\Delta'}
{}
\]

\subsubsection{Apply}

\[
\infer[\az]
{\az \Delta, \Delta''; A \lolli B \rightarrow \Xi'; \Delta'}
{\mz \Delta; A \rightarrow \Delta & \dz \Delta''; \Delta; \cdot ; B \rightarrow \Xi'; \Delta'}
\]

\[
\infer[\doz rule]
{\doz \Delta; R, \Phi \rightarrow \Xi';\Delta'}
{\doz \Delta; R \rightarrow \Xi';\Delta'}
\]

\subsection{Definitions}

\subsubsection{Well Formed Frame}

Given a frame $f = (\Delta; \Delta'; \Xi; p; \Omega_1, ..., \Omega_n; \Lambda_1, ..., \Lambda_m)$ and a body term $A$ and a context $\Delta_{inv}$, we say that $f$ is well formed iff:

\begin{enumerate}
   \item $\Lambda_1, ..., \Lambda_m$ are atomic terms $p_i$.
   \item $\Xi = \Xi_1, ..., \Xi_m$
   \item $\feq{p, \Omega_1, ..., \Omega_n, \Lambda_1, ..., \Lambda_m}{A}$
   \item $\mz \Xi \rightarrow \Lambda_1 \otimes ... \otimes \Lambda_m$.
   \item $\Delta, \Delta', \Xi = \Delta_{inv}$.
\end{enumerate}

\subsubsection{Well Formed Stack}

A continuation stack $C$ is well formed iff every frame is well formed.

\subsubsection{Related Match}\label{sec:related_match}

$\mo \Delta; \Xi; \Omega; H; C; R \rightarrow \Xi'; \Delta'$ is related to a term $A$ and a context $\Delta_{inv}$ iff:

\begin{itemize}
   \item $C$ is well formed in relation to $A$ and $\Delta_{inv}$;
   \item $\Delta, \Xi = \Delta_{inv}$
   \item Either:
   \begin{itemize}
      \item $C = \cdot$
   
      $\feq{\Omega}{A}$
   
      \item $C = (\Delta_a; \Delta_b; \Xi''; p; \Omega'; \Lambda_1, ..., \Lambda_m), C'$
   
      $\feq{A}{\Omega, p, \Omega', \Lambda_1, ..., \Lambda_n}$ \\
      $\feq{\Omega'}{\Omega}$ \\ 
      $p_1 \in \Xi$ and $\mz p_1 \rightarrow p$ \\
      $\Xi = \Xi'', p_1$ \\
   \end{itemize}
\end{itemize}

\subsection{Theorems}

\subsubsection{Lexicographical Ordering}

\begin{enumerate}
   \item $\cont C; H; R; \Xi'; \Delta' \prec \cont C', C; H; R; \Xi'; \Delta'$
   \item $\cont C', (\Delta, \Delta_1; \Delta_2), C; H; R; \Xi'; \Delta' \prec \cont C'', (\Delta; \Delta_1, \Delta_2), C; H; R; \Xi'; \Delta'$
   \item $\cont C'', C; H; R; \Xi'; \Delta' \prec \mo \Delta; \Xi; \Omega; H; C', C; R \rightarrow \Xi'; \Delta'$ as long as $C'' \prec C$
   \item $\mo \Delta''; \Xi''; \Omega'; H; C', C; R \rightarrow \Xi'; \Delta' \prec \mo \Delta; \Xi; \Omega; H; C; R \rightarrow \Xi'; \Delta'$ as long as $\Omega' \prec \Omega$
   \item $\mo \Delta; \Xi; \Omega; H; C; R \rightarrow \Xi'; \Delta' \prec \cont C', C; H; R; \Xi'; \Delta'$
   \item $\mo \Delta; \Xi; \Omega; H; C'', (\Delta, \Delta_1; \Delta_2), C; R \rightarrow \Xi'; \Delta' \prec \mo \Delta''; \Xi''; \Omega'; C', (\Delta; \Delta_1, \Delta_2), C; R \rightarrow \Xi'; \Delta'$
\end{enumerate}

\subsubsection{Match soundness theorem}\label{thm:match_soundness_basic}

If a match $\mo \Delta_1, \Delta_2; \Xi; \Omega; H; C; R \rightarrow \Xi'; \Delta'$ is related to term $A$ and context $\Delta_1, \Delta_2, \Xi$ then either:\\
\begin{enumerate}
   \item $\cont \cdot; H; R; \Xi'; \Delta'$ or
   \item $\mz \Delta_x \rightarrow A$ (where $\Delta_x$ is a subset of $\Delta_1, \Delta_2, \Xi$) and one of the two subcases are true:
      \begin{enumerate}
         \item $\mo \Delta_1; \Xi, \Delta_2; \cdot; H; C'', C; R \rightarrow \Xi'; \Delta'$ (related) and $\Delta_x = \Xi, \Delta_2$
         \item $\exists f = (\Delta_a; \Delta_{b_1}, p_2, \Delta_{b_2}; \Omega_1, ..., \Omega_k; \Xi_1, ..., \Xi_m; \Lambda_1, ..., \Lambda_m) \in C$ where $C = C', f, C''$ and $f$ turns into some $f' = (\Delta_a, \Delta_{b_1}, p_2; \Delta_{b_2}; p; \Omega_1, ..., \Omega_k; \Xi_1, ..., \Xi_m; \Lambda_1, ..., \Lambda_m)$ such that:
            \begin{itemize}
               \item $\mo \Delta_c; \Xi_1, ..., \Xi_m, p_2, \Xi_c; \cdot; H; C''', f', C''; R \rightarrow \Xi'; \Delta'$ (related)
            \end{itemize}
      \end{enumerate}
\end{enumerate}

If $\cont C; H; R; \Xi'; \Delta'$ and $C$ is well formed in relation to $A$ and $\Delta_1, \Delta_2, \Xi$ then either:\\
\begin{enumerate}
   \item $\cont \cdot; H; R; \Xi'; \Delta'$
   \item $\mz \Delta_x \rightarrow A$ (where $\Delta_x$ is a subset of $\Delta_1, \Delta_2, \Xi$) where:
   \begin{enumerate}
      \item $\exists f = (\Delta_a; \Delta_{b_1}, p_2, \Delta_{b_2}; \Omega_1, ..., \Omega_k; \Xi_1, ..., \Xi_m; \Lambda_1, ..., \Lambda_m) \in C$ where $C = C', f, C''$ and $f$ turns into some $f' = (\Delta_a, \Delta_{b_1}, p_2; \Delta_{b_2}; p; \Omega_1, ..., \Omega_k; \Xi_1, ..., \Xi_m; \Lambda_1, ..., \Lambda_m)$ such that:
         \begin{itemize}
            \item $\mo \Delta_c; \Xi_1, ..., \Xi_m, p_2, \Xi_c; \cdot; H; C''', f', C''; R \rightarrow \Xi'; \Delta'$ (related)
         \end{itemize}
   \end{enumerate}
\end{enumerate}

Proof by mutual induction. In $\mo$ on the size of $\Omega$ and on $\cont$, first on the size of $\Delta''$ and then on the size of $C$. Use the stack constraints and the Match Equivalence Theorem to prove $\mz \Delta_x \rightarrow A$.

\begin{itemize}
   \item $\mo ok \; first$
   
   First prove that the inverted match is related and then use induction.
   
   \item $\mo ok \; other$
   
   First we prove that the inverted match is related. We know that $q \in \Xi$ due to our assumption, so that proves $\mz q \rightarrow q$. \\
   Second, we prove that the new stack frame is related and then apply induction.
   
   \item $\mo fail$
   
   $\mo \Delta; \Xi; p, \Omega; H; C; R \rightarrow \Xi'; \Delta'$ \hfill (1) assumption \\
   $\cont C; H; R; \Xi'; \Delta'$ \hfill (2) inversion of (1) \\
   Apply i.h. on (2) to get $\cont \cdot; H; R; \Xi'; \Delta'$ or the $\mz \Delta_x \rightarrow A$.
   
   \item $\mo \otimes$
   
   By induction.
   
   \item $\mo end$
   
   $\mo \Delta; \Xi; \cdot; H; C; R \rightarrow \Xi'; \Delta'$ \hfill (1) assumption.\\
   Our stack must have some frames, thus $\feq{p, \Lambda_c}{A}$ ($p$ and $\Lambda_c$ both arguments of the last frame). We also know that $p \in \Xi$ due to our assumption and that $\mz \Xi_c, p \rightarrow \Lambda \otimes p$ is true. Therefore $\Xi = \Xi_c, p$ and thus $\mz \Xi \rightarrow A$.
   
   \item $\cont next \; rule$
   
   $\cont \cdot; H; (\Phi; \Delta); \Xi'; \Delta'$ \hfill (1) assumption \\
   
   \item $\cont next$
   
   $\cont (\Delta; p_1, \Delta''; p, \Omega; \Xi; \Lambda), C; H; R; \Xi'; \Delta'$ \hfill (1) assumption \\
   $\mo \Delta, \Delta''; \Xi, p_1;  \Omega; (\Delta, p_1; \Delta''; p, \Omega; H; \Xi; \Lambda), C; R \rightarrow \Xi'; \Delta'$ \hfill (2) inversion of (1)\\
   Since the frame we push into the match makes the match related, we can use induction (2):
   
   \begin{enumerate}
      \item $\cont \cdot; H; R; \Xi'; \Delta'$
      
      \item $\mz \Delta_x \rightarrow A$
      \begin{itemize}
         \item $\mo \Delta_a; \Xi, p_1, \Delta_1, ..., \Delta_k; \cdot; H; C''; R \rightarrow \Xi'; \Delta'$ 
      
         where $\Delta_a, \Delta_1, ..., \Delta_k = \Delta, \Delta'$ \\
      
         $\exists f = (\Delta; p_1; \Delta''; p, \Omega; H; \Xi; \Lambda)$
      
         \item $\exists f \in (\Delta, p_1; \Delta''; p, \Omega; H; \Xi; \Lambda), C$
      
         $f$ can be $(\Delta, p_1; \Delta''; p, \Omega; H; \Xi; \Lambda)$ (which is contained in the original $\cont$)\\
         or $f \in C$\\
      \end{itemize}
   \end{enumerate}
   
   \item $\cont no \; more$
   
   $\cont (\Delta; \cdot; p, \Omega; \Xi), C; H; R; \Xi'; \Delta'$ \hfill (1) assumption \\
   $\cont C; H; R; \Xi'; \Delta'$ \hfill (2) inversion of (1) \\
   
   Apply induction.
\end{itemize}

\subsection{Match Soundness Lemma}

If $\mo \Delta, \Delta''; \cdot; \Omega; H; \cdot; R \rightarrow \Xi'; \Delta'$ then either:\\
\begin{enumerate}
   \item $\cont \cdot; R; \Xi'; \Delta'$ or;
   \item $\mo \Delta; \Delta''; \cdot; H; C'; R \rightarrow \Xi'; \Delta'$ and $\mz \Delta'' \rightarrow \Omega$
\end{enumerate}

\begin{proof}
By direct use of the match soundness theorem.
\end{proof}

\subsection{Derive Soundness Theorem}

If $\done \Delta; \Xi; \Delta_1; \Omega; C \rightarrow \Xi'; \Delta'$ then $\dz \Delta; \Xi; \Delta_1; \Omega \rightarrow \Xi'; \Delta'$.

By simple induction on $\Omega$.

\begin{itemize}
   \item $\done p$
   
   By induction.
   
   \item $\done 1$
   
   By induction.
   
   \item $\done \otimes$
   
   By induction.
   
   \item $\done end$
   
   Use axioms.
   
\end{itemize}

\section{Low Level System With Comprehensions}



\subsection{High Level System}

\subsubsection{Match}

\[
\infer[\mz 1]
{\mz \cdot ; 1 \rightarrow 1}
{}
\tab
\infer[\mz p]
{\mz p ; p \rightarrow p}
{}
\]

\[
\infer[\mz \otimes]
{\mz \Delta_1, \Delta_2 ; A \otimes B \rightarrow \Xi_1, \Xi_2}
{\mz \Delta_1 ; A \rightarrow \Xi_1 & \mz \Delta_2 ; B \rightarrow \Xi_2}
\]

\subsubsection{Derive}

\[
\infer[\dz p]
{\dz \Delta ; \Xi ; \Delta_1 ; p, \Omega \rightarrow \Xi' ; \Delta'}
{\dz \Delta ; \Xi ; p, \Delta_1 ; \Omega \rightarrow \Xi' ; \Delta'}
\]

\[
\infer[\dz \otimes]
{\dz \Delta; \Xi; \Delta_1; A \otimes B, \Omega \rightarrow \Xi'; \Delta'}
{\dz \Delta; \Xi; \Delta_1; A, B, \Omega \rightarrow \Xi'; \Delta'}
\tab
\]

\[
\infer[\dz end]
{\dz \Delta; \Xi'; \Delta'; \cdot \rightarrow \Xi';\Delta'}
{}
\]


\[
\infer[\dz comp]
{\dz \Delta ; \Xi; \Delta_1; \m{comp} A \lolli B, \Omega \rightarrow \Xi';\Delta'}
{\dz \Delta; \Xi; \Delta_1; 1 \with (A \lolli B \otimes \m{comp} A \lolli B), \Omega \rightarrow \Xi';\Delta'}
\]

\[
\infer[\dz \lolli]
{\dz \Delta_a, \Delta_b; \Xi; \Delta_1; A \lolli B, \Omega \rightarrow \Xi';\Delta'}
{\mz \Delta_a; A \rightarrow \Delta_a & \dz \Delta_b ; \Xi; \Delta_a; \Delta_1; B, \Omega \rightarrow \Xi'; \Delta'}
\]

\[
\infer[\dz \with L]
{\dz \Delta; \Xi; \Delta_1; A \with B, \Omega \rightarrow \Xi';\Delta'}
{\dz \Delta; \Xi; \Delta_1; A, \Omega \rightarrow \Xi';\Delta'}
\tab
\infer[\dz \with R]
{\dz \Delta; \Xi; \Delta_1; A \with B, \Omega \rightarrow \Xi';\Delta'}
{\dz \Delta; \Xi; \Delta_1; B, \Omega \rightarrow \Xi';\Delta'}
\]

\subsubsection{Apply}

\[
\infer[\az rule]
{\az \Delta, \Delta''; A \lolli B \rightarrow \Xi'; \Delta'}
{\mz \Delta; A \rightarrow \Delta & \dz \Delta''; \Delta; \cdot ; B \rightarrow \Xi'; \Delta'}
\]

\[
\infer[\doz rule]
{\doz \Delta; R, \Phi \rightarrow \Xi';\Delta'}
{\doz \Delta; R \rightarrow \Xi';\Delta'}
\]

\subsection{Low Level System}

We extend the previous system with comprehensions.

\subsubsection{Match}

\[
\infer[\mo ok]
{\mo \Delta, p_1, \Delta'' ; \Xi; p, \Omega; H; C; R \rightarrow \Xi'; \Delta'}
{\mo \Delta, \Delta''; \Xi, p_1; \Omega; H; (\Delta, p_1; \Delta''; p, \Omega; H; \Xi), C; R \rightarrow \Xi'; \Delta'}
\tab
\infer[\mo fail]
{\mo \Delta; \Xi; p, \Omega; H; C; R \rightarrow \Xi'; \Delta'}
{p \notin \Delta & \cont C ; R; \Xi'; \Delta'}
\]

\[
\infer[\mo \otimes]
{\mo \Delta; \Xi; A \otimes B, \Omega ; H ; C; R \rightarrow \Xi'; \Delta'}
{\mo \Delta; \Xi; A, B, \Omega; H; C; R \rightarrow \Xi';\Delta'}
\]

\[
\infer[\mo end]
{\mo \Delta; \Xi; \cdot ; H; C; R \rightarrow \Xi'; \Delta'}
{\done \Delta; \Xi; \cdot ; H; \cdot \rightarrow \Xi'; \Delta'}
\]

\subsubsection{Derive}

\newcommand{\mc}[0]{\m{mc} \; }
\newcommand{\dall}[0]{\m{dall} \; }

\[
\infer[\done p]
{\done \Delta; \Xi; \Delta_1; p, \Omega; C \rightarrow \Xi'; \Delta'}
{\done \Delta; \Xi; p, \Delta_1; \Omega; C \rightarrow \Xi'; \Delta'}
\tab
\infer[\done 1]
{\done \Delta; \Xi; \Delta_1; 1, \Omega; C \rightarrow \Xi';\Delta'}
{\done \Delta; \Xi; \Delta_1; \Omega; C \rightarrow \Xi'; \Delta'}
\]

\[
\infer[\done \otimes]
{\done \Delta; \Xi; \Delta_1; A \otimes B, \Omega; C \rightarrow \Xi'; \Delta'}
{\done \Delta; \Xi; \Delta_1; A, B, \Omega; C \rightarrow \Xi';\Delta'}
\]

\[
\infer[\done end]
{\done \Delta; \Xi; \Delta_1; \cdot \rightarrow \Xi; \Delta_1}
{}
\]

\[
\infer[\done comp]
{\done \Delta ; \Xi; \Delta_1; \comp A \lolli B, \Omega \rightarrow \Xi';\Delta'}
{\mc \Delta; \Xi; \Delta_1; \cdot; A ; B ; \cdot; \comp A \lolli B; \Omega; \Delta \rightarrow \Xi';\Delta'}
\]

\subsubsection{Continuation}

\[
\infer[\cont next \; rule]
{\cont \cdot; (\Phi; \Delta); \Xi'; \Delta'}
{\doo \Delta; \Phi \rightarrow \Xi'; \Delta'}
\]

\[
\infer[\cont next]
{\cont (\Delta; p_1, \Delta''; p, \Omega; H; \Xi), C; R; \Xi'; \Delta'}
{\mo \Delta, \Delta''; \Xi, p_1; \Omega; H; (\Delta, p_1; \Delta''; p, \Omega; H; \Xi), C; R \rightarrow \Xi'; \Delta'}
\]

\[
\infer[\cont no \; more]
{\cont (\Delta; \cdot; p, \Omega; H; \Xi), C; R; \Xi'; \Delta'}
{\cont C; R; \Xi'; \Delta'}
\]

\subsubsection{Match Comprehension}

\[
\infer[\mc p \; ok \; first]
{\mc \Delta, p_1, \Delta''; \Xi_N; \Delta_{N1}; \Xi; p, \Omega; \cdot; AB; \Omega_N; \Delta_N \rightarrow \Xi'; \Delta'}
{\mc \Delta, \Delta''; \Xi_N; \Delta_{N1}; \Xi, p_1; \Omega; (\Delta, p_1; \Delta''; \Xi; p; \Omega; \cdot); AB; \Omega_N; \Delta_N \rightarrow \Xi'; \Delta'}
\]

\[
\infer[\mc p \; ok \; other]
{\mc \Delta, p_1, \Delta''; \Xi_N; \Delta_{N1}; \Xi; p, \Omega; C_1, C; AB; \Omega_N; \Delta_N \rightarrow \Xi'; \Delta'}
{\mc \Delta, \Delta''; \Xi_N; \Delta_{N1}; \Xi, p_1; \Omega; (\Delta, p_1; \Delta''; \Xi; p; \Omega; q, \Lambda), C_1, C; AB; \Omega_N; \Delta_N \rightarrow \Xi'; \Delta' & C1 = (\Delta_{old}; \Delta'_{old}; \Xi_{old}; q; \Omega_{old}; \Lambda)}
\]


\[
\infer[\mc p \; fail]
{\mc \Delta; \Xi_N; \Delta_{N1}; \Xi; p, \Omega; C; \comp A \lolli B; \Omega_N; \Delta_N \rightarrow \Xi'; \Delta'}
{\contc \Delta_N; \Xi_N; \Delta_{N1}; C; \comp A \lolli B; \Omega_N \rightarrow \Xi'; \Delta'}
\]

\[
\infer[\mc \otimes]
{\mc \Delta; \Xi_N; \Delta_{N1}; \Xi; X \otimes Y, \Omega; C; \comp A \lolli B; \Omega_N; \Delta_N \rightarrow \Xi'; \Delta'}
{\mc \Delta; \Xi_N; \Delta_{N1}; \Xi; X, Y, \Omega; C; \comp A \lolli B; \Omega_N; \Delta_N \rightarrow \Xi'; \Delta'}
\]

\[
\infer[\mc end]
{\mc \Delta; \Xi_N; \Delta_{N1}; \Xi; \cdot; C; \comp A \lolli B; \Omega_N; \Delta_N \rightarrow \Xi'; \Delta'}
{\dall \Xi_N; \Delta_{N1}; \Xi; C; \comp A \lolli B; \Omega_N; \Delta_N \rightarrow \Xi'; \Delta'}
\]

\[
\infer[\dall end]
{\dall \Xi_N; \Delta_{N1}; \Xi; (\Delta_x; \Delta''; \cdot; p, \Omega; \cdot); \comp A \lolli B; \Omega_N; \Delta_N \rightarrow \Xi'; \Delta'}
{\dc \Xi_N, \Xi; \Delta_{N1}; (\Delta_x - \Xi; \Delta'' - \Xi; \cdot; p; \Omega; \cdot) ; \comp A \lolli B; \Omega_N; (\Delta_N - \Xi) \rightarrow \Xi'; \Delta'}
\]

\[
\infer[\dall more]
{\dall \Xi_N; \Delta_{N1}; \Xi; \_, X, C; \comp A \lolli B; \Omega_N; \Delta_N \rightarrow \Xi'; \Delta'}
{\dall \Xi_N; \Delta_{N1}; \Xi; X, C; \comp A \lolli B; \Omega_N; \Delta_N \rightarrow \Xi'; \Delta'}
\]

\[
\infer[\dc p]
{\dc \Xi; \Delta_1; p, \Omega; C; \comp A \lolli B; \Omega_N; \Delta_N \rightarrow \Xi'; \Delta'}
{\dc \Xi; \Delta_1, p; \Omega; C; \comp A \lolli B; \Omega_N; \Delta_N \rightarrow \Xi'; \Delta'}
\]

\[
\infer[\dc \otimes]
{\dc \Xi; \Delta_1; A \otimes B, \Omega; C; \comp A \lolli B; \Omega_N; \Delta_N \rightarrow \Xi'; \Delta'}
{\dc \Xi; \Delta_1; A, B, \Omega; C; \comp A \lolli B; \Omega_N; \Delta_N \rightarrow \Xi'; \Delta'}
\]

\[
\infer[\dc end]
{\dc \Xi; \Delta_1; \cdot; C; \comp A \lolli B; \Omega_N; \Delta_N \rightarrow \Xi'; \Delta'}
{\contc \Delta_N; \Xi; \Delta_1; C; \comp A \lolli B; \Omega_N \rightarrow \Xi'; \Delta'}
\]

\subsubsection{Match Comprehension Continuation}

\[
\infer[\contc end]
{\contc \Delta_N; \Xi_N; \Delta_{N1}; \cdot; \comp A \lolli B; \Omega \rightarrow \Xi'; \Delta'}
{\done \Delta_N; \Xi_N; \Delta_{N1}; \Omega \rightarrow \Xi'; \Delta'}
\]

\[
\infer[\contc next]
{\contc \Delta_N; \Xi_N; \Delta_{N1}; (\Delta; p_1, \Delta''; \Xi; p; \Omega; \Lambda), C; AB; \Omega_N \rightarrow \Xi'; \Delta'}
{\mc \Delta; \Xi_N; \Delta_{N1}; \Xi; \Omega; (\Delta, p_1; \Delta''; \Xi; p; \Omega; \Lambda), C; AB; \Omega_N; \Delta_N \rightarrow \Xi'; \Delta'}
\]

\[
\infer[\contc next \; empty]
{\contc \Delta_N; \Xi_N; \Delta_{N1}; (\Delta; \cdot; \Xi; p; \Omega; \Lambda), C; AB; \Omega_N \rightarrow \Xi'; \Delta'}
{\contc \Delta_N; \Xi_N; \Delta_{N1}; C; AB; \Omega_N \rightarrow \Xi'; \Delta'}
\]

\subsubsection{Apply}

\[
\infer[\ao start \; matching]
{\ao \Delta; A \lolli B; R \rightarrow \Xi'; \Delta'}
{\mo \Delta; \cdot; A; B; \cdot; R \rightarrow \Xi'; \Delta'}
\]

\[
\infer[\doo rule]
{\doo \Delta; R, \Phi \rightarrow \Xi'; \Delta'}
{\ao \Delta; R; (\Phi; \Delta) \rightarrow \Xi';\Delta'}
\]

\subsection{Flat Equivalence}

\newcommand{\flatten}[0]{\m{flatten} \;}

\newcommand{\feq}[2]{#1 \equiv #2}

\[
\infer[\equiv p]
{\feq{p, A}{p, B}}
{\feq{A}{B}}
\]

\[
\infer[\equiv 1 \; left]
{\feq{1, A}{B}}
{\feq{A}{B}}
\tab
\infer[\equiv 1 \; right]
{\feq{A}{1, B}}
{\feq{A}{B}}
\]

\[
\infer[\equiv \cdot]
{\feq{\cdot}{\cdot}}
{}
\]

\[
\infer[\equiv left]
{\feq{A \otimes B, C}{D}}
{\feq{A, B, C}{D}}
\tab
\infer[\equiv right]
{\feq{A}{B \otimes C, D}}
{\feq{A}{B, C, D}}
\]

\subsubsection{Equivalence commutativity}

If $\feq{A}{B}$ then $\feq{B}{A}$.

\begin{proof}
   By induction on the structure of the assumption.
   
   \begin{itemize}
      \item $p$
      
      $\feq{p, A}{p, B}$ \hfill (1) assumption \\
      $\feq{A}{B}$ \hfill (2) inversion of (1) \\
      $\feq{B}{A}$ \hfill (3) i.h. on (2) \\
      $\feq{p, B}{p, A}$ \hfill (4) rule on (3) \\
      
      \item $1 \; left$
      
      $\feq{1, A}{B}$ \hfill (1) assumption \\
      $\feq{A}{B}$ \hfill (2) assumption \\
      $\feq{B}{A}$ \hfill (3) i.h. on (2) \\
      $\feq{B}{1, A}$ \hfill (4) apply right rule on (3) \\
      
      \item $1 \; right$
      
      Same thing as before.
      
      \item $\cdot$
      
      Immediate.
      
      \item $left$
      
      $\feq{A \otimes B, C}{D}$ \hfill (1) assumption \\
      $\feq{A, B, C}{D}$ \hfill (2) inversion of (1) \\
      $\feq{D}{A, B, C}$ \hfill (3) i.h. on (2) \\
      $\feq{D}{A \otimes B, C}$ \hfill (4) right rule on (3) \\
      
      \item $right$
      
      Same thing as the last case.
   \end{itemize}
\end{proof}

\subsubsection{Match Equivalence Theorem}

If $\feq{A_1, ..., A_n}{B_1, ..., B_m}$ and $\mz \Delta; A_1 \otimes ... \otimes A_n \rightarrow \Delta$ then $\mz \Delta; B_1 \otimes ... \otimes B_m \rightarrow \Delta$.

\begin{proof}
   By induction on the structure of the assumption.
   
   \begin{itemize}
      \item $p$
      
      $\feq{p, A}{p, B}$ \hfill (1) assumption \\
      $\feq{A}{B}$ \hfill (2) assumption \\
      $\mz p; p \rightarrow p$ \hfill (3) axiom \\
      $\mz p, \Delta; p \otimes A \rightarrow p, \Delta$ \hfill (4) assumption \\
      $\mz \Delta; A \rightarrow \Delta$ \hfill (5) inversion of (4) \\
      $\mz \Delta; B \rightarrow \Delta$ \hfill (6) i.h. on (5) and (2) \\
      $\mz p, \Delta; p \otimes B \rightarrow p, \Delta$ \hfill (7) rule application on (6) and (3) \\
      
      \item $1 \; left$
      
      $\feq{1, A}{B}$ \hfill (1) assumption \\
      $\feq{A}{B}$ \hfill (2) assumption \\
      $\mz \Delta; 1 \otimes A \rightarrow \Delta$ \hfill (3) assumption \\
      $\mz \Delta; A \rightarrow \Delta$ \hfill (4) inversion of (4) \\
      $\mz \cdot; 1 \rightarrow \cdot$ \hfill (5) axiom \\
      $\mz \Delta; B \rightarrow \Delta$ \hfill (6) i.h. on (2) and (4) \\
      $\mz \Delta; 1 \otimes B \rightarrow \Delta$ \hfill (7) rule application on (6) and (5) \\
      
      \item $1 \; right$
      
      Same thing as before.
      
      \item $\cdot$
      
      Immediate since $\mz$ fails.
      
      \item $\otimes \; left$
      
      $\feq{A \otimes B, C}{D}$ \hfill (1) assumption \\
      $\feq{A, B, C}{D}$ \hfill (2) inversion of (1) \\
      $\mz \Delta_1, \Delta_2, \Delta_3; (A \otimes B) \otimes C \rightarrow \Delta_1, \Delta_2, \Delta_3$ \hfill (3) assumption \\
      $\mz \Delta_1, \Delta_2; A \otimes B \rightarrow \Delta_1, \Delta_2$ \hfill (4) inversion of (3) \\
      $\mz \Delta_3; C \rightarrow \Delta_3$ \hfill (5) inversion of (3) \\
      $\mz \Delta_1; A \rightarrow \Delta_1$ and $\mz \Delta_2; B \rightarrow \Delta_2$ \hfill (6) inversion of (4) \\
      $\mz \Delta_1, \Delta_2, \Delta_3; A \otimes B \otimes C \rightarrow \Delta_1, \Delta_2, \Delta_3$ \hfill (7) apply (5) with (6) \\
      $\mz \Delta_1, \Delta_2, \Delta_3; D \rightarrow \Delta_1, \Delta_2, \Delta_3$ \hfill (8) i.h. on (7) and (2)\\
      
      \item $\otimes \; right$
      
      Apply equivalence commutativity theorem and follow the previous case.
   \end{itemize}
\end{proof}

\subsection{Continuation frame and stack properties}

\subsubsection{Flat Property}

Given a frame $f = (\Delta; \Delta''; \Xi; p; \Omega_1, ..., \Omega_n; \Lambda_1, ..., \Lambda_m)$ and a body term $A$, we say that $f$ follows the flat property if $\feq{p, \Omega_1, ..., \Omega_n, \Lambda_1, ..., \Lambda_m}{A}$.

\subsubsection{Flat well-formed}

A continuation stack $C$ is "flat" well-formed if every frame follows the flat property.

\subsubsection{Resource Invariant}

A frame $f = (\Delta'; \Delta''; \Xi; p; \Omega_1, ..., \Omega_n; \Lambda_1, ..., \Lambda_m)$ is resource invariant in relation to a $\Delta$ context if $\Delta', \Delta'', \Xi = \Delta$.

A continuation stack $C$ is resource invariant in relation to a $\Delta$ context if $\forall f \in C. f$ is resource invariant in relation to $\Delta$.

\subsection{Theorems}

\subsubsection{Body Match Soundness}

If $\mo \Delta, \Delta_1, ..., \Delta_n; \Xi; \Omega_1, ..., \Omega_n; H; C; R \rightarrow \Xi'; \Delta'$ then either:\\
1. \hspace{1cm} $\cont \cdot; R; \Xi'; \Delta'$ or \\
2. \hspace{1cm} $\mo \Delta; \Xi, \Delta_1, ..., \Delta_n; \cdot; H; C''; R \rightarrow \Xi'; \Delta'$ and $\mz \Delta_1; \Omega_1 \rightarrow \Delta_1$ ... $\mz \Delta_n; \Omega_n \rightarrow \Delta_n$ or \\
3. \hspace{1cm} $\exists f = (\Delta_a; \Delta_b; \Omega_1, ..., \Omega_n; H; \Xi_a) \in C$ such that:\\
3.1 \hspace{2cm} $\Delta_a, \Delta_b = \Delta_c, \Delta_1, ..., \Delta_n$ and \\
3.2 \hspace{2cm} $\mo \Delta_c; \Xi_a, \Delta_1, ..., \Delta_n; \cdot; H; C''; R \rightarrow \Xi'; \Delta'$ and \\
3.3 \hspace{2cm} $\mz \Delta_1; \Omega_1 \rightarrow \Delta_1$ ... $\mz \Delta_n; \Omega_n \rightarrow \Delta_n$.\\

If $\cont C; R; \Xi'; \Delta'$ then either:\\
1. \hspace{1cm} $\cont \cdot; R; \Xi'; \Delta'$ or \\
2. \hspace{1cm} $\exists f = (\Delta_a; \Delta_b; \Omega_1, ..., \Omega_n; H; \Xi_a) \in C$ such that:\\
2.1 \hspace{2cm} $\Delta_a, \Delta_b = \Delta_c, \Delta_1, ..., \Delta_n$ and \\
2.2 \hspace{2cm} $\mo \Delta_c; \Xi_a, \Delta_1, ..., \Delta_n; \cdot; H; C''; R \rightarrow \Xi'; \Delta'$ and \\
2.3 \hspace{2cm} $\mz \Delta_1; \Omega_1 \rightarrow \Delta_1$ ... $\mz \Delta_n; \Omega_n \rightarrow \Delta_n$.\\

\begin{proof}
Proved in Section~\ref{thm:match_soundness_basic}.
\end{proof}

\subsubsection{Comprehension soundness theorem}

\begin{itemize}
   \item Match sub-theorem

If $\mc \Delta, \Delta_1, ..., \Delta_n; \Xi_N; \Delta_{N1}; \Xi; \Omega_1, ..., \Omega_n; C; \comp A \lolli B; \Omega_N; \Delta_N \rightarrow \Xi'; \Delta'$  and $C$ is flat well-formed in respect to $A$ and $C$ is resource invariant in relation to $\Delta, \Delta_1, ..., \Delta_n, \Xi$ and $\Delta, \Delta_1, ..., \Delta_n, \Xi = \Delta_N$:\\

The matching will fail (1), succeed without using any continuation frame in $C$ (2) or it needs to backtrack to a frame in $C$ (3):\\
1. \hspace{1cm} $\done \Delta_N; \Xi_N; \Delta_{N1}; \Omega_N \rightarrow \Xi'; \Delta'$; \\
2. \hspace{1cm} $\mc \Delta; \Xi_N; \Delta_{N1}; \Xi, \Delta_1, ..., \Delta_n; \cdot; C', C; \comp A \lolli B; \Omega_N; \Delta_N \rightarrow \Xi'; \Delta'$ and\\
2.1 \hspace{2cm} $\mz \Delta_1; \Omega_1 \rightarrow \Delta_1$ ... $\mz \Delta_n; \Omega_n \rightarrow \Delta_n$\\
2.2 \hspace{2cm} $C'$ is flat well-formed in relation to $A$. \\
2.3 \hspace{2cm} $C'$ is resource invariant in relation to $\Delta, \Delta_1, ..., \Delta_n, \Xi$.\\
3. \hspace{1cm} $C = C_1, f, C_2$ and $f = (\Delta_a; \Delta_{b_1}, \Delta_{b_2}; \Xi_1, ..., \Xi_m; p; \Omega'_1, ..., \Omega'_k; \Lambda_1, ..., \Lambda_m)$ turns into $f' = (\Delta_a, \Delta_{b_1}; \Delta_{b_2}; \Xi_1, ..., \Xi_m; p; \Omega'_1, ..., \Omega'_k; \Lambda_1, ..., \Lambda_m)$ and\\
3.1 \hspace{2cm} $\Delta_a, \Delta_{b1}, \Delta_{b2} = \Delta_c, p, \Delta_1, ..., \Delta_k$ \\
3.2 \hspace{2cm} $\mc \Delta_c; \Xi_N; \Delta_{N1}; \Xi_1, ..., \Xi_m, \Delta_1, ..., \Delta_k; \cdot; C'_1, f', C_2; \comp A \lolli B; \Omega_N; \Delta_N \rightarrow \Xi'; \Delta'$.\\
3.3 \hspace{2cm} $C'_1, f'$ are flat well-formed in relation to $A$. \\
3.4 \hspace{2cm} $C'_1, f'$ are resource invariant in relation to $\Delta, \Delta_1, ..., \Delta_n, \Xi$ \\
3.5 \hspace{2cm} $\mz p; p \rightarrow p$, $\mz \Delta_1; \Omega'_1 \rightarrow \Delta_1$ ... $\mz \Delta_k; \Omega'_k \rightarrow \Delta_k$ and $\mz \Xi_1; \Lambda_1 \rightarrow \Xi_1$ ... $\mz \Xi_m; \Lambda_m \rightarrow \Xi_m$\\

We assume the following "context constraint" to be true:\\

1. If $C = \cdot$ then $\feq{A}{\Omega_1, ..., \Omega_n}$.\\
2. If $C = (\Delta_a; \Delta_b; \Xi; p; \Omega; \Lambda_1, ..., \Lambda_m), C'$ then $\feq{A}{p, \Omega_1, ..., \Omega_n, \Lambda_1, ..., \Lambda_m}$ and $\feq{\Omega_1, ..., \Omega_n}{\Omega}$.

\item Continuation sub-theorem

If $\contc \Delta_N; \Xi_N; \Delta_{N1}; C; \comp A \lolli B; \Omega_N \rightarrow \Xi'; \Delta'$ and $C$ is flat well-formed in respect to $A$ and resource invariant in relation to $\Delta_N$: \\
1. \hspace{1cm} $\done \Delta_N; \Xi_N; \Delta_{N1}; \Omega_N \rightarrow \Xi'; \Delta'$ or \\
2. \hspace{1cm} $C = C_1, f, C_2$ and $f = (\Delta_a; \Delta_{b_1}, \Delta_{b_2}; \Xi_1, ..., \Xi_m; p; \Omega'_1, ..., \Omega'_k; \Lambda_1, ..., \Lambda_m)$ turns into $f' = (\Delta_a, \Delta_{b_1}; \Delta_{b_2}; \Xi_1, ..., \Xi_m; p; \Omega'_1, ..., \Omega'_k; \Lambda_1, ..., \Lambda_m)$ and\\
2.1 \hspace{2cm} $\Delta_a, \Delta_{b1}, \Delta_{b2} = \Delta_c, p, \Delta_1, ..., \Delta_k$. \\
2.2 \hspace{2cm} $\mc \Delta_c; \Xi_N; \Delta_{N1}; \Xi_1, ..., \Xi_m, \Delta_1, ..., \Delta_k; \cdot; C'_1, f', C_2; \comp A \lolli B; \Omega_N; \Delta_N \rightarrow \Xi'; \Delta'$. \\
2.3 \hspace{2cm} $C'_1, f'$ are flat well-formed in relation to $A$. \\
2.4 \hspace{2cm} $C'_1, f'$ are resource invariant in relation to $\Delta, \Delta_1, ..., \Delta_n, \Xi$ \\
2.5 \hspace{2cm} $\mz p; p \rightarrow p$, $\mz \Delta_1; \Omega'_1 \rightarrow \Delta_1$ ... $\mz \Delta_k; \Omega'_k \rightarrow \Delta_k$ and $\mz \Xi_1; \Lambda_1 \rightarrow \Xi_1$ ... $\mz \Xi_m; \Lambda_m \rightarrow \Xi_m$\\

\end{itemize}

\begin{proof}
   By nested induction. In $\mc$ on the size of $\Omega = \Omega_1, ..., \Omega_n$. In $\contc$, first on the size of $\Delta''$ of the continuation frame and then on the continuation stack $C$.
   
   \begin{itemize}
      \item $\mc p \; ok \; first$
      
      By induction on $\Omega$.\\
      Stack frame $(\Delta, p_1; \Delta''; \Xi; p; \Omega; \cdot)$ is flat well-formed. \\
      
      \item $\mc p \; ok \; other$
      
      By induction on $\Omega$.\\
      Stack frame $(\Delta, p_1; \Delta''; \Xi; p; \Omega; q, \Lambda)$ is flat well-formed.
      
      \item $\mc p \; fail$
      
      By mutual induction on $\contc$.
      
      \item $\mc \otimes$
      
      By induction on $\Omega$. \\
      
      \item $\mc end$
      
      Select case 2 with assumption.
      
      \item $\contc end$
      
      Select case 1 with assumption.
      
      \item $\contc next$
      
      By induction on $\Delta''$.\\
      
      \item $\contc next \; empty$
      
      By induction on the size of $C$.
      
   \end{itemize}
\end{proof}

\subsubsection{Comprehension soundness lemma}

If $\mc \Delta, \Xi; \Xi_N; \Delta_{N1}; \cdot; A; \cdot; \comp A \lolli B; \Omega_N; \Delta_N \rightarrow \Xi'; \Delta'$ where $\Delta, \Xi = \Delta_N$ then either:\\
1. \hspace{1cm} $\done \Delta_N; \Xi_N; \Delta_{N1}; \Omega_N \rightarrow \Xi'; \Delta'$ or \\
2. \hspace{1cm} $\mc \Delta; \Xi_N; \Delta_{N1}; \Xi; \cdot; C'; \comp A \lolli B; \Omega_N; \Delta_N \rightarrow \Xi'; \Delta'$ and \\
2.1 \hspace{2cm} $\mz \Delta_1; A \rightarrow \Delta_1$ \\
2.2 \hspace{2cm} $C'$ is flat well-formed in relation to $A$. \\
2.3 \hspace{2cm} $C'$ is resource invariant in relation to $\Delta, \Xi$ \\

\begin{proof}
Direct application of the previous theorem.
\end{proof}

\subsubsection{Dall transformation theorem}

If $\dall \Xi_N; \Delta_{N1}; \Xi; C, (\Delta_a; \Delta_b; \cdot; \Omega; \cdot); \comp A \lolli B; \Omega_N; \Delta_N \rightarrow \Xi'; \Delta'$ then\\
$\dc \Xi_N, \Xi; \Delta_{N1}; B; (\Delta_a - \Xi; \Delta_b - \Xi; \cdot; \Omega; \cdot); \comp A \lolli B; \Omega_N; (\Delta_N - \Xi) \rightarrow \Xi'; \Delta'$.

\begin{proof}
   By induction on the size of the continuation stack $C$.
   
   \begin{itemize}
      \item $C = \_, X, C'$
      
      $\dall \Xi_N; \Delta_{N1}; \Xi; \_, X, C'; \comp A \lolli B; \Omega_N; \Delta_N \rightarrow \Xi'; \Delta'$ \hfill (1) assumption \\
      $\dall \Xi_N; \Delta_{N1}; \Xi; X, C'; \comp A \lolli B; \Omega_N; \Delta_N \rightarrow \Xi'; \Delta'$ \hfill (2) inversion of (1) \\
      $\dc \Xi_N, \Xi; \Delta_{N1}; B; (\Delta_a - \Xi; \Delta_b - \Xi; \cdot; \Omega; \cdot); \comp A \lolli B; \Omega_N; (\Delta_N - \Xi) \rightarrow \Xi'; \Delta'$ \hfill (3) i.h. on (2) \\
      
      \item $C = (\Delta_a - \Xi; \Delta_b - \Xi; \cdot; \Omega; B)$
      
      $\dall \Xi_N; \Delta_{N1}; \Xi; (\Delta_a; \Delta_b; \cdot; \Omega; \cdot); \comp A \lolli B; \Omega_N; \Delta_N \rightarrow \Xi'; \Delta'$ \hfill (1) assumption \\
      $\dc \Xi_N, \Xi; \Delta_{N1}; B; (\Delta_a - \Xi; \Delta_b - \Xi; \cdot; \Omega; B); \comp A \lolli B; \Omega_N; (\Delta_N - \Xi) \rightarrow \Xi'; \Delta'$ \hfill (2) inversion of (1) \\
   \end{itemize}
\end{proof}

\subsubsection{Successful comprehension match gives derivation}

If $\mc \Delta; \Xi_N; \Delta_{N1}; \Xi; \cdot; B; C, (\Delta_a; \Delta_b; \cdot; p; \Omega; \cdot); \comp A \lolli B; \Omega_N; \Delta_N \rightarrow \Xi'; \Delta'$ then:\\
\hspace{1cm} $\dc \Xi_N, \Xi; \Delta_{N1}; B; (\Delta_a - \Xi; \Delta_b - \Xi; \cdot; p; \Omega; \cdot); \comp A \lolli B; \Omega_N; (\Delta_N - \Xi) \rightarrow \Xi'; \Delta'$.

\begin{proof}
   Invert the assumption and then apply dall transformation theorem.
\end{proof}

\subsubsection{Continuation derivation theorem}

If $\dc \Xi_N; \Delta_{N1}; \Omega_1, ..., \Omega_n; C; \comp A \lolli B; \Omega_N; \Delta_N \rightarrow \Xi'; \Delta'$ then:\\
1. \hspace{1cm} $\dc \Xi_N; \Delta_{N1}, \Delta_1, ..., \Delta_n; \cdot; C; \comp A \lolli B; \Omega_N; \Delta_N \rightarrow \Xi'; \Delta'$ and \\
2. \hspace{1cm} If $\dz \Delta; \Xi_N; \Delta_{N1}, \Delta_1, ..., \Delta_n; \Omega \rightarrow \Xi'; \Delta'$ then $\dz \Delta; \Xi_N; \Delta_{N1}; \Omega_1, ..., \Omega_n, \Omega \rightarrow \Xi'; \Delta'$

\begin{proof}
   By induction on the size of $\Omega_1, ..., \Omega_n$.
   
   \begin{itemize}
      \item $p, \Omega$
      
      $\dc \Xi_N; \Delta_{N1}; p, \Omega_2, ..., \Omega_n; C; \comp A \lolli B; \Omega_N; \Delta_N \rightarrow \Xi'; \Delta'$ \hfill (1) assumption \\
      $\dc \Xi_N; \Delta_{N1}, p; \Omega_2, ..., \Omega_n; C; \comp A \lolli B; \Omega_N; \Delta_N \rightarrow \Xi'; \Delta'$ \hfill (2) inversion of (1) \\
      $\dc \Xi_N; \Delta_{N1}, p, \Delta_2, ..., \Delta_n; \cdot; C; \comp A \lolli B; \Omega_N; \Delta_N \rightarrow \Xi'; \Delta'$ \hfill (3) i.h. on (2) \\
      if $\dz \Delta; \Xi_N; \Delta_{N1}, p, \Delta_2, ..., \Delta_n; \Omega \rightarrow \Xi'; \Delta'$ then $\dz \Delta; \Xi_N; \Delta_{N1}; p, \Omega_2, ..., \Omega_n, \Omega \rightarrow \Xi'; \Delta'$ \hfill (4) i.h. on (2) \\
      $\dz \Delta; \Xi_N; \Delta_{N1}, p, \Delta_2, ..., \Delta_n; \Omega \rightarrow \Xi'; \Delta'$ \hfill (5) from (4) \\
      $\dz \Delta; \Xi_N; \Delta_{N1}; p, \Omega_2, ..., \Omega_n, \Omega \rightarrow \Xi'; \Delta'$ \hfill (6) using (5) on (4) \\
      
      \item $1, \Omega$
      
      By induction.
      
      \item $A \otimes B, \Omega$
      
      $\dc \Xi_N; \Delta_{N1}; A \otimes B, \Omega_2, ..., \Omega_n; C; \comp A \lolli B; \Omega_N; \Delta_N \rightarrow \Xi'; \Delta'$ \hfill (1) assumption \\
      $\dc \Xi_N; \Delta_{N1}; A, B, \Omega_2, ..., \Omega_n; C; \comp A \lolli B; \Omega_N; \Delta_N \rightarrow \Xi'; \Delta'$ \hfill (2) inversion of (1) \\
      Solve by induction on (2) \\
   \end{itemize}
\end{proof}

\subsubsection{Continuation derivation lemma}

If $\dc \Xi_N; \Delta_{N1}; \Omega_x; C; \comp A \lolli B; \Omega_N; \Delta_N \rightarrow \Xi'; \Delta'$ then: \\
1. \hspace{1cm} $\dc \Xi_N; \Delta_{N1}, \Delta_x; \cdot; C; \comp A \lolli B; \Omega_N; \Delta_N \rightarrow \Xi'; \Delta'$ and \\
2. \hspace{1cm} If $\dz \Delta; \Xi_N; \Delta_{N1}, \Delta_x; \Omega \rightarrow \Xi'; \Delta'$ then $\dz \Delta; \Xi_N; \Delta_{N1}; \Omega_x, \Omega \rightarrow \Xi'; \Delta'$

\begin{proof}
   By direct application of the continuation derivation theorem.
\end{proof}

\subsubsection{Comprehension Theorem}

If $\mc \Delta_a, \Delta'_b; \Xi_N; \Delta_{N1}; p_1; \Omega; (\Delta_a, p_1; \Delta'_b; \cdot; p; \Omega; \cdot); \comp A \lolli B; \Omega_N; \Delta, \Xi_1, ..., \Xi_n \rightarrow \Xi'; \Delta'$ and $\Delta_a, \Delta_b = \Delta, \Xi_1, ..., \Xi_n$ and $\feq{p, \Omega}{A}$ then $\exists n \geq 0$ such that: \\
1. \hspace{1cm} $\done \Delta; \Xi_N, \Xi_1, ..., \Xi_n; \Delta_{N1}, \Delta_1, ..., \Delta_n; \Omega_N \rightarrow \Xi'; \Delta'$\\
2. \hspace{1cm} $\mz \Xi_1; A \rightarrow \Xi_1$ ... $\mz \Xi_n; A \rightarrow \Xi_n$ \\
3. \hspace{1cm} $n$ implications from $1 ... i ... n$ such that: $\forall \Omega_x, \Delta_x.$ if $\dz \Delta_x; \Xi_N, \Xi_1, ..., \Xi_i; \Delta_{N1}, \Delta_1, ..., \Delta_i; \Omega_x \rightarrow \Xi'; \Delta'$ then $\dz \Delta_x; \Xi_N, \Xi_1, ..., \Xi_i; \Delta_{N1}, \Delta_1, ..., \Delta_{i-1}; B, \Omega_x \rightarrow \Xi'; \Delta'$.

\begin{proof}
   By induction on the size of $\Delta_b$.\\

   $\Delta_b = p_1, \Delta'_b$ \hfill (1) from assumption \\
   $\Delta_a, p_1, \Delta'_b = \Delta, p, \Xi'_1, ..., \Xi_n$ \hfill (2) from assumption \\
   By applying the comprehension soundness theorem on the assumption:\\
      
   \begin{itemize}
      \item Failure:
         
      $\done \Delta, \Xi_1, ..., \Xi_n; \Xi_N; \Delta_{N1}; \Omega_N \rightarrow \Xi'; \Delta'$ \hfill (3) assumption (so $n = 0$)\\
         
      \item Success:
         
      $\mc \Delta, \Xi_2, ..., \Xi_n; \Xi_N; \Delta_{N1}; p, \Xi'_1; \cdot; C', (\Delta_a, p_1; \Delta'_b; \cdot; p; \Omega; \cdot); \comp A \lolli B; \Omega_N; \Delta, \Xi_1, ..., \Xi_n \rightarrow \Xi'; \Delta'$ \hfill (3) assumption \\
      $\mz p_1; p \rightarrow p_1$ and $\mz \Xi'_1; \Omega \rightarrow \Xi'_1$ \hfill (4) assumption \\
      $C'$ is flat well-formed \hfill (5) assumption \\
      $\mz p_1, \Xi'_1; p \otimes \Omega \rightarrow p_1, \Xi'_1$ \hfill (6) from (4) \\
      $\feq{p, \Omega}{A}$ \hfill (7) assumption about the original continuation stack \\
      $\mz p_1, \Xi'_1; A \rightarrow p_1, \Xi'_1$ \hfill (8) from match equivalence theorem on (6) and (7) \\
      $\dc \Xi_N, p, \Xi'_1; \Delta_{N1}; B; (\Delta_a, p_1 - (p_1, \Xi'_1); \Delta'_b - (p_1, \Xi'_1); \cdot; p; \Omega; \cdot); \comp A \lolli B; \Omega_N; (\Delta, \Xi_1, ..., \Xi_n) - \Xi_1 \rightarrow \Xi'; \Delta'$ \hfill (9) apply successful comprehension matches gives derivation lemma to (3) \\
      $\dc \Xi_N, p, \Xi'_1; \Delta_{N1}; B; (\Delta_a - \Xi'_1; \Delta'_b - \Xi'_1; \cdot; p; \Omega; \cdot); \comp A \lolli B; \Omega_N; \Delta, \Xi_2, ..., \Xi_n \rightarrow \Xi'; \Delta'$ \hfill (10) simplification of (9) \\
      $\dc \Xi_N, p, \Xi'_1; \Delta_{N1}, \Delta_1; \cdot; (\Delta_a - \Xi'_1; \Delta'_b - \Xi'_1; \cdot; p; \Omega; \cdot); \comp A \lolli B; \Omega_N; \Delta, \Xi_2, ..., \Xi_n \rightarrow \Xi'; \Delta'$ \hfill (11) from continuation derivation lemma on (10)\\
      if $\forall \Omega_x, \Delta_x. \dz \Delta_x; \Xi_N, p, \Xi'_1; \Delta_{N1}, \Delta_1; \Omega_x \rightarrow \Xi'; \Delta'$ then $\dz \Delta_x; \Xi_N, p, \Xi'_1; \Delta_{N1}; B, \Omega_x \rightarrow \Xi'; \Delta'$ \hfill (12) using the same lemma \\
      $\contc \Delta, \Xi_2, ..., \Xi_n; \Xi_N, p, \Xi'_1; \Delta_{N1}, \Delta_1; (\Delta_a - \Xi'_1; \Delta'_b - \Xi'_1; \cdot; p; \Omega; \cdot); \comp A \lolli B; \Omega_N \rightarrow \Xi'; \Delta'$ \hfill (13) inversion of (11) \\   
      
      When inverting (11), we have two subcases:
      
      $\Delta_a - \Xi'_1 = \Delta''_a$ \hfill (12) \\
      $\Delta'_b - \Xi'_1 = \Delta''_b$ \hfill (13) \\
      $\Delta''_a, \Delta''_b = \Delta, \Xi_2, ..., \Xi_n$ \hfill (14) \\
      
      \begin{itemize}
         \item End ($n = 1$):
         
         $\contc \Delta, \Xi_2, ..., \Xi_n; \Xi_N, \Xi_1; \Delta_{N1}, \Delta_1; \cdot; \comp A \lolli B; \Omega_N \rightarrow \Xi'; \Delta'$ \hfill (15) \hfill inversion of (11) \\
         $\done \Delta, \Xi_2, ..., \Xi_n; \Xi_N, \Xi_1; \Delta_{N1}, \Delta_1; \Omega_N \rightarrow \Xi'; \Delta'$ \hfill (16) inverting (15), which is what we want \\
         
         \item Next ($n = n' + 1$):
         
         $\Delta'''_b = \Delta''_b, p_2$ \hfill (15) from inversion \\
         $\mc \Delta''_a, \Delta'''_b; \Xi_N, \Xi_1; \Delta_{N1}, \Delta_1; \cdot; \Omega; (\Delta''_a, p_2; \Delta'''_b; \cdot; p; \Omega; \cdot); \comp A \lolli B; \Omega_N; \Delta, \Xi_2, ..., \Xi_n \rightarrow \Xi'; \Delta'$ \hfill (16) inversion of (11) \\
         Apply induction hypotheses to (16) to get results from $n'$.\\ 
      \end{itemize}
      
      \item Backtrack:
      
      $f = (\Delta_a, p_1; \Delta'_b; \cdot; p; \Omega; \cdot)$ \hfill (3) from theorem \\
      turns into $f' = (\Delta_a, p_1, \Delta'''_b, p_2; \Delta''_b; \cdot; p; \Omega; \cdot)$ \hfill (4) from theorem (3) \\
      $\mc \Delta, \Xi_2, ..., \Xi_n; \Xi_N; \Delta_{N1}; p_2, \Xi'_1; \cdot; C', f'; \comp A \lolli B; \Omega_N; \Delta, \Xi_1, ..., \Xi_n \rightarrow \Xi'; \Delta'$ \hfill (5) from theorem (3) \\
      $\mz p_2; p \rightarrow p_2$ \hfill (6) from theorem (3) \\
      $\mz \Xi'_1; \Omega \rightarrow \Xi'_1$ \hfill (7) from theorem (3) \\
      $\mz p_2, \Xi'_1; \Omega \otimes p \rightarrow p_2, \Xi'_1$ \hfill (8) rule application on (6) and (7) \\
      $\feq{p, \Omega}{A}$ \hfill (9) from assumptions \\
      $\mz p_2, \Xi'_1 ; A \rightarrow p_2, \Xi'_1$ \hfill (10) using match equivalence theorem \\
      
      Use the same approach as the last subcase, but using $p_2$ instead of $p_1$ and using the fact that $\Delta_b$ was already reduced because we had to backtrack.
   \end{itemize}
   
   
\end{proof}

\subsubsection{Comprehension Lemma}

If $\mc \Delta, \Xi_1, ..., \Xi_n; \Xi_N; \Delta_{N1}; \cdot; A; \cdot; \comp A \lolli B; \Omega_N; \Delta, \Xi_1, ..., \Xi_n \rightarrow \Xi'; \Delta'$ then we can apply the comprehension $n >= 0$ times: \\
1. \hspace{1cm} $\done \Delta; \Xi_N, \Xi_1, ..., \Xi_n; \Delta_{N1}, \Delta_1, .., \Delta_n; \Omega_N \rightarrow \Xi'; \Delta'$ for $\exists n \geq 0$\\
2. \hspace{1cm} $\mz \Xi_1; A \rightarrow \Xi_1$ ... $\mz \Xi_n; A \rightarrow \Xi_n$.\\
3. \hspace{1cm} $n$ implications from $1 ... i ... n$ such that: $\forall \Omega_x, \Delta_x.$ if $\dz \Delta_x; \Xi_N, \Xi_1, ..., \Xi_i; \Delta_{N1}, \Delta_1, ..., \Delta_i; \Omega_x \rightarrow \Xi'; \Delta'$ then $\dz \Delta_x; \Xi_N, \Xi_1, ..., \Xi_i; \Delta_{N1}, \Delta_1, ..., \Delta_{i-1}; B, \Omega_x \rightarrow \Xi'; \Delta'$.

\begin{proof}
   Applying the comprehension soundness lemma, we get two cases:
   
   \begin{itemize}
      \item Failure:
      
      $\done \Delta, \Xi_1, ..., \Xi_n; \Xi_N; \Delta_{N1}; \Omega_N \rightarrow \Xi'; \Delta'$ \hfill no comprehension application was possible.
      
      \item Success:
      
      $\mc \Delta, \Xi_2, ..., \Xi_n; \Xi_N; \Delta_{N1}; \Xi_1; \cdot; C'; \comp A \lolli B; \Omega_N; \Delta, \Xi_1, ..., \Xi_n \rightarrow \Xi'; \Delta'$ \hfill (1) result from theorem \\
      $\mz \Xi_1; A \rightarrow \Xi_1$ \hfill (2) from theorem \\
      $C$ is flat well-formed in relation to $A$. \hfill (3) from theorem \\
      $C$ is resource invariant in relation to $\Delta, \Xi_1, ..., \Xi_n$ \hfill (4) from theorem \\
      
      $\dc \Xi_N, \Xi_1; \Delta_{N1}; B; (\Delta_a - \Xi_1; \Delta_b - \Xi_1; \cdot; p; \Omega; \cdot); \comp A \lolli B; \Omega_N; (\Delta, \Xi_1, ..., \Xi_n - \Xi_1) \rightarrow \Xi'; \Delta'$ \hfill (5) applying successful comprehension match gives derivation \\
      $\dc \Xi_N, \Xi_1; \Delta_{N1}; B; (\Delta_a - \Xi_1; \Delta_b - \Xi_1; \cdot; p; \Omega; \cdot); \comp A \lolli B; \Omega_N; \Delta, \Xi_2, ..., \Xi_n \rightarrow \Xi'; \Delta'$ \hfill (6) simplifying (5) \\
      $\Delta_a, \Delta_b = \Delta, \Xi_1, ..., \Xi_n$ \hfill (7) from (4) \\
      $(\Delta_a - \Xi_1), (\Delta_b - \Xi_1) = (\Delta_a, \Delta_b) - \Xi_1 = \Delta_, \Xi_2, .., \Xi_n$ and $\Delta'_a = \Delta_a - \Xi_1$ and $\Delta'_b = \Delta_b - \Xi_1$ \hfill (8) from (7) and (6) \\
      $\feq{p, \Omega}{A}$ \hfill (9) from (3) and (6) \\
      $\dc \Xi_N, \Xi_1; \Delta_{N1}, \Delta_1; (\Delta'_a; \Delta'_b; \cdot; p; \Omega; \cdot); \comp A \lolli B; \Omega_N; \Delta, \Xi_2, ..., \Xi_n \rightarrow \Xi'; \Delta'$ \hfill (10) using continuation derivation lemma \\
      if $\forall \Omega_x, \Delta_x. \dz \Delta_x; \Xi_N, \Xi_1; \Delta_{N1}, \Delta_1; \Omega_x \rightarrow \Xi'; \Delta'$ then $\dz \Delta_x; \Xi_N, \Xi_1; \Delta_{N1}; B, \Omega_x \rightarrow \Xi'; \Delta'$ \hfill (11) using the same lemma \\ 
      $\contc \Delta, \Xi_2, ..., \Xi_n; \Xi_N, \Xi_1; \Delta_{N1}, \Delta_1; (\Delta'_a; \Delta'_b; \cdot; p; \Omega; \cdot); \comp A \lolli B; \Omega_N \rightarrow \Xi'; \Delta'$ \hfill (12) inversion of (10) \\
      Apply continuation theorem to (12) to get the remaining $n-1$.
   \end{itemize}
\end{proof}

\subsubsection{Body Derive Soundness}

If $\done \Delta; \Xi; \Delta_1; \Omega \rightarrow \Xi'; \Delta$ then $\dz \Delta; \Xi; \Delta_1; \Omega \rightarrow \Xi'; \Delta'$.

\begin{proof}
   By induction on $\Omega$.
   
   \begin{itemize}
      \item $p, \Omega$
      
      By induction.
      
      \item $1, \Omega$
      
      By induction.
      
      \item $A \otimes B, \Omega$
      
      By induction.
      
      \item $\cdot$
      
      Use axiom.
      
      \item $\comp A \lolli B, \Omega$
      
      By using the comprehension lemma, we get $n$ applications of the comprehension.\\
      $\Delta = \Delta, \Xi_1, ..., \Xi_n$ \hfill (1)\\
      $\done \Delta; \Xi, \Xi_1, ..., \Xi_n; \Delta_1, \Delta_{c_1}, ..., \Delta_{c_n}; \Omega \rightarrow \Xi'; \Delta'$ \hfill (2) from lemma \\
      $\dz \Delta; \Xi, \Xi_1, ..., \Xi_n; \Delta_1, \Delta_{c_1}, ..., \Delta_{c_n}; \Omega \rightarrow \Xi'; \Delta'$ \hfill (3) i.h. on (2) \\
      $\dz \Delta; \Xi, \Xi_1, ..., \Xi_n; \Delta_1, \Delta_{c_1}, ..., \Delta_{c_n}; 1, \Omega \rightarrow \Xi'; \Delta'$ \hfill (4) apply $\dz 1$ on (3) \\
      $\dz \Delta; \Xi, \Xi_1, ..., \Xi_n; \Delta_1, \Delta_{c_1}, ..., \Delta_{c_n}; 1 \with (A \lolli B \otimes \comp A \lolli B), \Omega \rightarrow \Xi'; \Delta'$ \hfill (5) apply $\dz \with L$ on (4) \\
      $\dz \Delta; \Xi, \Xi_1, ..., \Xi_n; \Delta_1, \Delta_{c_1}, ..., \Delta_{c_n}; \comp A \lolli B, \Omega \rightarrow \Xi'; \Delta'$ \hfill (6) apply $\dz comp$ on (5) \\
      Using the $n^{th}$ implication of the comprehension theorem:\\
      $\dz \Delta; \Xi, \Xi_1, ..., \Xi_n; \Delta_1, \Delta_{c_1}, ..., \Delta_{c_{n-1}}; B, \comp A \lolli B, \Omega \rightarrow \Xi'; \Delta'$ \hfill (7) \\
      Using the $\mz \Xi_n; A \rightarrow \Xi_n$ result: \\
      $\dz \Delta, \Xi_n; \Xi_1, ..., \Xi_{n-1}; \Delta_1, \Delta_{c_1}, ..., \Delta_{c_{n-1}}; A \lolli B, \comp A \lolli B, \Omega \rightarrow \Xi'; \Delta'$ \hfill (8) using $\mz$ on (7) \\
      $\dz \Delta, \Xi_n; \Xi_1, ..., \Xi_{n-1}; \Delta_1, \Delta_{c_1}, ..., \Delta_{c_{n-1}}; A \lolli B \otimes \comp A \lolli B, \Omega \rightarrow \Xi'; \Delta'$ \hfill (9) applying rule $\dz \otimes$ on (8) \\
      $\dz \Delta, \Xi_n; \Xi_1, ..., \Xi_{n-1}; \Delta_1, \Delta_{c_1}, ..., \Delta_{c_{n-1}}; 1 \with (A \lolli B \otimes \comp A \lolli B), \Omega \rightarrow \Xi'; \Delta'$ \hfill (10) applying rule $\dz \with R$ on (9) \\
      $\dz \Delta, \Xi_n; \Xi_1, ..., \Xi_{n-1}; \Delta_1, \Delta_{c_1}, ..., \Delta_{c_{n-1}}; \comp A \lolli B, \Omega \rightarrow \Xi'; \Delta'$ \hfill (11) applying rule $\dz comp$ on (10) \\
      By applying the other $n-1$ comprehensions we will get: \\
      $\dz \Delta, \Xi_1, ..., \Xi_n; \Delta_1; \comp A \lolli B, \Omega \rightarrow \Xi'; \Delta'$ \hfill (12)
   \end{itemize}
\end{proof}

\section{Low Level System With Persistent Facts}

\subsection{High Level System}

\subsubsection{Match}

\[
\infer[\mz 1]
{\mz \Gamma; \cdot \rightarrow 1}
{}
\tab
\infer[\mz p]
{\mz \Gamma; p \rightarrow p }
{}
\tab
\infer[\mz \bang p]
{\mz \Gamma, p; \cdot \rightarrow \bang p}
{}
\]

\[
\infer[\mz \otimes]
{\mz \Gamma; \Delta_1, \Delta_2 \rightarrow A \otimes B}
{\mz \Gamma; \Delta_1 \rightarrow A & \mz \Delta_2 \rightarrow B}
\]

\subsubsection{Derive}

\[
\infer[\dz p]
{\dz \Gamma ; \Delta ; \Xi ; \Gamma_1 ; \Delta_1 ; p, \Omega \rightarrow \Xi' ; \Delta' ; \Gamma'}
{\dz \Gamma ; \Delta ; \Xi ; \Gamma_1 ; p, \Delta_1 ; \Omega \rightarrow \Xi' ; \Delta' ; \Gamma'}
\]

\[
\infer[\dz \bang p]
{\dz \Gamma ; \Delta ; \Xi ; \Gamma_1 ; \Delta_1 ; \bang p, \Omega \rightarrow \Xi' ; \Delta' ; \Gamma'}
{\dz \Gamma ; \Delta ; \Xi ; \Gamma_1, p ; \Delta_1 ; \Omega \rightarrow \Xi' ; \Delta' ; \Gamma'}
\]

\[
\infer[\dz \otimes]
{\dz \Gamma ; \Delta ; \Xi ; \Gamma_1 ; \Delta_1 ; A \otimes B, \Omega \rightarrow \Xi' ; \Delta' ; \Gamma'}
{\dz \Gamma ; \Delta ; \Xi ; \Gamma_1 ; \Delta_1 ; A, B, \Omega \rightarrow \Xi' ; \Delta' ; \Gamma'}
\]

\[
\infer[\dz 1]
{\dz \Gamma ; \Delta ; \Xi ; \Gamma_1; \Delta_1 ; 1, \Omega \rightarrow \Xi' ; \Delta' ; \Gamma'}
{\dz \Gamma ; \Delta ; \Xi ; \Gamma_1; \Delta_1 ; \Omega \rightarrow \Xi' ; \Delta' ; \Gamma'}
\]

\[
\infer[\dz end]
{\dz \Gamma ; \Delta ; \Xi' ; \Gamma' ; \Delta' ; \cdot \rightarrow \Xi' ; \Delta' ; \Gamma'}
{}
\]

\[
\infer[\dz comp]
{\dz \Gamma ; \Delta ; \Xi ; \Gamma_1 ; \Delta_1 ; \comp A \lolli B, \Omega \rightarrow \Xi' ; \Delta' ; \Gamma'}
{\dz \Gamma ; \Delta ; \Xi ; \Gamma_1 ; \Delta_1 ; 1 \with (A \lolli B \otimes \comp A \lolli B), \Omega \rightarrow \Xi' ; \Delta' ; \Gamma'}
\]

\[
\infer[\dz \lolli]
{\dz \Gamma ; \Delta_a, \Delta_b ; \Xi ; \Gamma_1 ; \Delta_1 ; A \lolli B, \Omega \rightarrow \Xi' ; \Delta' ; \Gamma'}
{\mz \Gamma ; \Delta_a \rightarrow A & \dz \Gamma ; \Delta_b ; \Xi, \Delta_a ; \Gamma_1 ; \Delta_1 ; B, \Omega \rightarrow \Xi' ; \Delta' ; \Gamma'}
\]

\[
\infer[\dz \with L]
{\dz \Gamma ; \Delta ; \Xi ; \Gamma_1 ; \Delta_1 ; A \with B, \Omega \rightarrow \Xi' ; \Delta'; \Gamma'}
{\dz \Gamma ; \Delta ; \Xi ; \Gamma_1 ; \Delta_1 ; A, \Omega \rightarrow \Xi' ; \Delta'; \Gamma'}
\tab
\infer[\dz \with R]
{\dz \Gamma ; \Delta ; \Xi ; \Gamma_1 ; \Delta_1 ; A \with B, \Omega \rightarrow \Xi' ; \Delta' ; \Gamma'}
{\dz \Gamma ; \Delta ; \Xi ; \Gamma_1 ; \Delta_1 ; B, \Omega \rightarrow \Xi' ; \Delta' ; \Gamma'}
\]

\subsubsection{Apply}

\[
\infer[\az rule]
{\az \Gamma ; \Delta, \Delta'' ; A \lolli B \rightarrow \Xi' ; \Delta' ; \Gamma'}
{\mz \Gamma ; \Delta \rightarrow A & \dz \Gamma ; \Delta''; \Delta; \cdot ; \cdot ; B \rightarrow \Xi' ; \Delta' ; \Gamma'}
\]

\[
\infer[\doz rule]
{\doz \Gamma ; \Delta ; R, \Phi \rightarrow \Xi' ; \Delta' ; \Gamma'}
{\doz \Gamma ; \Delta ; R \rightarrow \Xi' ; \Delta' ; \Gamma'}
\]

\newcommand{\strans}[0]{\m{strans} \;}

\subsection{Low Level System}

We extend the previous system with persistent facts. Most judgments are extended with the $\Gamma$ context for persistent facts. While the matching mechanism has new frames for persistent facts, everything else remains the same. Because persistent facts are never consumed, we only need to make sure that the frames read the facts from the $\Gamma$ context.

All the judgments in this system have been extended with $\Gamma'$, the context that contains the persistent resources created during the application of some rule.

\subsubsection{Continuation Frames}

The system adds a new continuation frame for persistent facts with the form $[\Gamma'; \Delta; \bang p; \Omega; \Xi; \Lambda; \Upsilon]$:

\begin{enumerate}
   \item[$\Gamma'$]: A multi-set of persistent resources that can be used if the current one fails.
   \item[$\Delta$]: The multi-set of linear resources at this point of the matching process.
   \item[$\bang p$]: The persistent atomic term that created this frame.
   \item[$\Omega$]: The remaining terms we need to match past this choice point. This is an ordered list.
   \item[$\Xi$]: A multi-set of linear resources we have consumed to reach this point.
   \item[$\Lambda$]: A multi-set of linear atomic terms that we have matched to reach this choice point. All the linear resources that match these terms are located in $\Xi$.
   \item[$\Upsilon$]: A multi-set of persistent atomic terms that we have matched to reach this point. All the persistent resources used for matching must be located in the $\Gamma$ of the matching judgment.
\end{enumerate}

Please note that the old continuation frame is also extended with $\Upsilon$.

\subsubsection{Match}

The $\mo$ judgments is simply extended with arguments $\Gamma$ and $\Gamma'$.
However, the number of inference rules has duplicated due to the presence of persistent frames and persistent terms.


\[
\infer[\mo p \; ok \; first]
{\mo \Gamma ; \Delta, p_1, \Delta'' ; \Xi; p, \Omega; H; C; R \rightarrow \Xi'; \Delta'; \Gamma'}
{\mo \Gamma ; \Delta, \Delta''; \Xi, p_1; \Omega; H; (\Delta, p_1; \Delta''; p; \Omega; \Xi; \cdot; \cdot); R \rightarrow \Xi'; \Delta'; \Gamma'}
\]

\[
\infer[\mo p \; ok \; other \; q]
{\mo \Gamma ; \Delta, p_1, \Delta'' ; \Xi; p, \Omega; H; C_1, C; R \rightarrow \Xi'; \Delta'; \Gamma'}
{\begin{split}\mo &\Gamma ; \Delta, \Delta''; \Xi, p_1; \Omega; H; (\Delta, p_1; \Delta''; p; \Omega; \Xi; q, \Lambda; \Upsilon), C_1, C; R \rightarrow \Xi'; \Delta'; \Gamma' \\ C_1 &= (\Delta_{old}; \Delta'_{old}; q; \Omega_{old}; \Xi_{old}; \Lambda; \Upsilon)\end{split}}
\]


\[
\infer[\mo p \; ok \; other \; \bang q]
{\mo \Gamma ; \Delta, p_1, \Delta'' ; \Xi; p, \Omega; H; C_1, C; R \rightarrow \Xi'; \Delta'; \Gamma'}
{\begin{split}\mo &\Gamma ; \Delta, \Delta''; \Xi, p_1; \Omega; H; (\Delta, p_1; \Delta''; p; \Omega; \Xi; \Lambda; q, \Upsilon), C_1, C; R \rightarrow \Xi'; \Delta'; \Gamma' \\ C_1 &= [\Gamma_{old}; \Delta_{old}; \bang q; \Omega_{old}; \Xi_{old}; \Lambda; \Upsilon]\end{split}}
\]

\[
\infer[\mo p \; fail]
{\mo \Gamma ; \Delta; \Xi; p, \Omega; H; C; R \rightarrow \Xi'; \Delta'; \Gamma'}
{p \notin \Delta & \cont C ; H; R; \Gamma; \Xi'; \Delta'; \Gamma'}
\]

\[
\infer[\mo \bang p \; ok \; first]
{\mo \Gamma, p, \Gamma' ; \Delta; \Xi; \bang p, \Omega; H; \cdot; R \rightarrow \Xi'; \Delta'; \Gamma'}
{\mo \Gamma, p, \Gamma' ; \Delta; \Xi; \Omega; H; [\Gamma'; \Delta; \bang p ; \Omega; \Xi; \Lambda; \cdot]; R \rightarrow \Xi'; \Delta'; \Gamma'}
\]

\[
\infer[\mo \bang p \; ok \; other \; q]
{\mo \Gamma, p, \Gamma' ; \Delta; \Xi; \bang p, \Omega; H; C_1, C; R \rightarrow \Xi'; \Delta'; \Gamma'}
{\begin{split}\mo &\Gamma, p, \Gamma' ; \Delta; \Xi; \Omega; H; [\Gamma'; \Delta; \bang p ; \Omega; \Xi; q, \Lambda; \Upsilon], C_1, C; R \rightarrow \Xi'; \Delta'; \Gamma' \\ C_1 &= (\Delta_{old}; \Delta'_{old}; q; \Omega_{old}; \Xi_{old}; \Lambda; \Upsilon)\end{split}}
\]


\[
\infer[\mo \bang p \; ok \; other \; \bang q]
{\mo \Gamma, p, \Gamma' ; \Delta; \Xi; \bang p, \Omega; H; C_1, C; R \rightarrow \Xi'; \Delta'; \Gamma'}
{\begin{split}\mo &\Gamma, p, \Gamma' ; \Delta; \Xi; \Omega; H; [\Gamma'; \Delta; \bang p ; \Omega; \Xi; \Lambda; q, \Upsilon], C_1, C; R \rightarrow \Xi'; \Delta'; \Gamma' \\ C_1 &= [\Gamma_{old}; \Delta_{old}; \bang q; \Omega_{old}; \Xi_{old}; \Lambda; \Upsilon]\end{split}}
\]

\[
\infer[\mo \bang p \; fail]
{\mo \Gamma ; \Delta; \Xi; \bang p, \Omega; H; C; R \rightarrow \Xi'; \Delta'; \Gamma'}
{p \notin \Gamma & \cont C; H; R; \Gamma; \Xi'; \Delta'; \Gamma'}
\]

\[
\infer[\mo \otimes]
{\mo \Gamma ; \Delta; \Xi; A \otimes B, \Omega ; H ; C; R \rightarrow \Xi'; \Delta';\Gamma'}
{\mo \Gamma ; \Delta; \Xi; A, B, \Omega; H; C; R \rightarrow \Xi';\Delta';\Gamma'}
\]

\[
\infer[\mo end]
{\mo \Gamma ; \Delta; \Xi; \cdot ; H; C; R \rightarrow \Xi'; \Delta'; \Gamma'}
{\done \Gamma ; \Delta; \Xi; \cdot ; H; \cdot \rightarrow \Xi'; \Delta'; \Gamma'}
\]


\subsubsection{Continuation}

The $\cont$ judgment has been extended with the $\Gamma$ context. We also need to handle the case where the top of the stack contains persistent frames.

\[
\infer[\cont next \; rule]
{\cont \cdot; H; (\Phi, \Delta); \Gamma ; \Xi'; \Delta'; \Gamma'}
{\doo \Gamma; \Delta; \Phi \rightarrow \Xi'; \Delta'; \Gamma'}
\]

\[
\infer[\cont p \; next]
{\cont (\Delta; p_1, \Delta''; p, \Omega; \Xi; \Lambda; \Upsilon), C; H; R; \Gamma; \Xi'; \Delta'; \Gamma'}
{\mo \Gamma ; \Delta, \Delta''; \Xi, p_1; \Omega; H; (\Delta, p_1; \Delta''; p, \Omega; H; \Xi; \Lambda; \Upsilon), C; R \rightarrow \Xi'; \Delta'; \Gamma'}
\]

\[
\infer[\cont p \; no \; more]
{\cont (\Delta; \cdot; p, \Omega; \Xi; \Lambda; \Upsilon), C; H; R; \Gamma; \Xi'; \Delta'; \Gamma'}
{\cont C; H; R; \Gamma; \Xi'; \Delta'; \Gamma'}
\]

\[
\infer[\cont \bang p \; next]
{\cont [p_1, \Gamma'; \Delta; \bang p, \Omega; \Xi; \Lambda; \Upsilon], C; H; R; \Gamma; \Xi'; \Delta'; \Gamma'}
{\mo \Gamma; \Delta; \Xi; \Omega; H; [\Gamma'; \Delta; \bang p, \Omega; \Xi; \Lambda; \Upsilon], C; R \rightarrow \Xi'; \Delta'; \Gamma'}
\]

\[
\infer[\cont \bang p \; no \; more]
{\cont [\cdot; \Delta; \bang p, \Omega; \Xi; \Lambda; \Upsilon], C; H; R; \Gamma; \Xi'; \Delta'; \Gamma'}
{\cont C; H; R; \Gamma; \Xi'; \Delta'; \Gamma'}
\]


\subsubsection{Derivation}

We extended with $\done$ judgment with $\Gamma$ and $\Gamma_1$. $\Gamma_1$ contains the derived persistent resources.


\[
\infer[\done p]
{\done \Gamma ; \Delta; \Xi; \Gamma_1 ; \Delta_1; p, \Omega \rightarrow \Xi'; \Delta'; \Gamma'}
{\done \Gamma ; \Delta; \Xi; \Gamma_1 ; p, \Delta_1; \Omega \rightarrow \Xi'; \Delta'; \Gamma'}
\tab
\infer[\done 1]
{\done \Gamma; \Delta; \Xi; \Gamma_1 ; \Delta_1; 1, \Omega \rightarrow \Xi';\Delta';\Gamma'}
{\done \Gamma; \Delta; \Xi; \Gamma_1 ; \Delta_1; \Omega \rightarrow \Xi'; \Delta';\Gamma'}
\]

\[
\infer[\done \bang p]
{\done \Gamma ; \Delta ; \Xi; \Gamma_1 ; \Delta_1; \bang p, \Omega \rightarrow \Xi'; \Delta'; \Gamma'}
{\done \Gamma ; \Delta ; \Xi; \Gamma_1, p; \Delta_1; \Omega \rightarrow \Xi'; \Delta'; \Gamma'}
\]

\[
\infer[\done \otimes]
{\done \Gamma ; \Delta; \Xi; \Gamma_1; \Delta_1; A \otimes B, \Omega \rightarrow \Xi'; \Delta';\Gamma'}
{\done \Gamma ; \Delta; \Xi; \Gamma_1; \Delta_1; A, B, \Omega \rightarrow \Xi';\Delta';\Gamma'}
\]

\[
\infer[\done end]
{\done \Gamma; \Delta; \Xi; \Gamma_1; \Delta_1; \cdot \rightarrow \Xi; \Delta_1; \Gamma_1}
{}
\]

\[
\infer[\done comp]
{\done \Gamma; \Delta ; \Xi; \Gamma_1; \Delta_1; \comp A \lolli B, \Omega \rightarrow \Xi';\Delta';\Gamma'}
{\mc \Gamma; \Delta; \Xi; \Gamma_1; \Delta_1; \cdot; A ; B ; \cdot; \cdot; \comp A \lolli B; \Omega; \Delta \rightarrow \Xi';\Delta';\Gamma'}
\]


\subsubsection{Match Comprehension}

For the matching process of the comprehensions, we use two stacks, $C$ and $P$. In $P$, we put all the initial persistent frames and in $C$ we put the first linear frame and then everything else. With this we can easily find out the first linear frame and remove everything that was pushed on top of such frame.
The match comprehension judgment $\mc$ has been extended with persistent frames and a few other arguments:

\begin{enumerate}
   \item[$\Gamma$]: The multi-set of persistent resources.
   \item[$\Gamma_{N1}$]: Multi-set of persistent resources derived up to this point in the head of the rule.
   \item[$C$]: The continuation stack that contains both linear and persistent frames. The first frame must be linear.
   \item[$P$]: The second part of the continuation stack with only persistent frames.
   \item[$\Gamma'$]: Multi-set of derived persistent resources.
\end{enumerate}

Like the $\mo$ judgment, we can see a duplication of inference rules due to the presence of persistent frames.


\[
\infer[\mc p \; ok \; first]
{\mc \Gamma; \Delta, p_1, \Delta''; \Xi_N; \Gamma_{N1}; \Delta_{N1}; \cdot; p, \Omega; \cdot; \cdot; AB; \Omega_N; \Delta_N \rightarrow \Xi'; \Delta'; \Gamma'}
{\mc \Gamma; \Delta, \Delta''; \Xi_N; \Gamma_{N1}; \Delta_{N1}; \Xi, p_1; \Omega; (\Delta, p_1; \Delta''; \cdot; p; \Omega; \cdot; \cdot); \cdot; AB; \Omega_N; \Delta_N \rightarrow \Xi'; \Delta'; \Gamma'}
\]

\[
\infer[\mc p \; ok \; other \; q]
{\mc \Gamma; \Delta, p_1, \Delta''; \Xi_N; \Gamma_{N1}; \Delta_{N1}; \Xi; p, \Omega; C_1, C; P; AB; \Omega_N; \Delta_N \rightarrow \Xi'; \Delta'; \Gamma'}
{\begin{split}\mc &\Gamma; \Delta, \Delta''; \Xi_N; \Gamma_{N1}; \Delta_{N1}; \Xi, p_1; \Omega; (\Delta, p_1; \Delta''; \Xi; p; \Omega; q, \Lambda; \Upsilon), C_1, C; P; AB; \Omega_N; \Delta_N \rightarrow \Xi'; \Delta'; \Gamma' \\ C_1 &= (\Delta_{old}; \Delta'_{old}; \Xi_{old}; q; \Omega_{old}; \Lambda; \Upsilon)\end{split}}
\]


\[
\infer[\mc p \; ok \; other \; \bang qC]
{\mc \Gamma; \Delta, p_1, \Delta''; \Xi_N; \Gamma_{N1}; \Delta_{N1}; \Xi; p, \Omega; C_1, C; P; AB; \Omega_N; \Delta_N \rightarrow \Xi'; \Delta'; \Gamma'}
{\begin{split}\mc &\Gamma; \Delta, \Delta''; \Xi_N; \Gamma_{N1}; \Delta_{N1}; \Xi, p_1; \Omega; (\Delta, p_1; \Delta''; \Xi; p; \Omega; \Lambda; q, \Upsilon), C_1, C; P; AB; \Omega_N; \Delta_N \rightarrow \Xi'; \Delta'; \Gamma' \\ C_1 &= [\Gamma_{old}; \Delta_{old}; \Xi_{old}; q; \Omega_{old}; \Lambda; \Upsilon]\end{split}}
\]


\[
\infer[\mc p \; ok \; other \; \bang qP]
{\mc \Gamma; \Delta, p_1, \Delta''; \Xi_N; \Gamma_{N1}; \Delta_{N1}; \cdot; p, \Omega; \cdot; P_1, P; AB; \Omega_N; \Delta_N \rightarrow \Xi'; \Delta'; \Gamma'}
{\begin{split}\mc &\Gamma; \Delta, \Delta''; \Xi_N; \Gamma_{N1}; \Delta_{N1}; p_1; \Omega; (\Delta, p_1; \Delta''; \cdot; p; \Omega; \cdot; q, \Upsilon); P_1, P; AB; \Omega_N; \Delta_N \rightarrow \Xi'; \Delta'; \Gamma' \\ P_1 &= [\Gamma_{old}; \Delta_N; \cdot; q; \Omega_{old}; \cdot; \Upsilon]\\ \Delta_N &= \Delta, p_1, \Delta''\end{split}}
\]


\[
\infer[\mc p \; fail]
{\mc \Gamma; \Delta; \Xi_N; \Gamma_{N1}; \Delta_{N1}; \Xi; p, \Omega; C; P; \comp A \lolli B; \Omega_N; \Delta_N \rightarrow \Xi'; \Delta'; \Gamma'}
{\contc \Gamma; \Delta_N; \Xi_N; \Gamma_{N1}; \Delta_{N1}; C; P; \comp A \lolli B; \Omega_N \rightarrow \Xi'; \Delta'; \Gamma'}
\]

\[
\infer[\mc \bang p \; first]
{\mc \Gamma, \Gamma', p; \Delta_N; \Xi_N; \Gamma_{N1}; \Delta_{N1}; \cdot; \bang p, \Omega; \cdot; \cdot; AB; \Omega_N; \Delta_N \rightarrow \Xi'; \Delta'; \Gamma'}
{\mc \Gamma, \Gamma', p; \Delta_N; \Xi_N; \Gamma_{N1}; \Delta_{N1}; \cdot; \Omega; \cdot; [\Gamma'; \Delta_N; \cdot; \bang p; \cdot; \Omega; \cdot; \cdot]; AB; \Omega_N; \Delta_N \rightarrow \Xi'; \Delta'; \Gamma'}
\]

\[
\infer[\mc \bang p \; other \; \bang qP]
{\mc \Gamma, \Gamma', p; \Delta_N; \Xi_N; \Gamma_{N1}; \Delta_{N1}; \cdot; \bang p, \Omega; \cdot; P_1, P; AB; \Omega_N; \Delta_N \rightarrow \Xi'; \Delta'; \Gamma'}
{\begin{split}\mc &\Gamma, \Gamma', p; \Delta_N; \Xi_N; \Gamma_{N1}; \Delta_{N1}; \cdot; \Omega; [\Gamma'; \Delta_N; \cdot; \bang p; \cdot; \Omega; \cdot; q, \Upsilon], P_1, P; AB; \Omega_N; \Delta_N \rightarrow \Xi'; \Delta'; \Gamma' \\ P_1 &= [\Gamma_{old}; \Delta_N; \cdot; \bang q; \Omega_{old}; \cdot; \Upsilon]\end{split}}
\]

\[
\infer[\mc \bang p \; other \; \bang qC]
{\mc \Gamma, \Gamma', p; \Delta; \Xi_N; \Gamma_{N1}; \Delta_{N1}; \Xi; \bang p, \Omega; C_1, C; P; AB; \Omega_N; \Delta_N \rightarrow \Xi'; \Delta'; \Gamma'}
{\begin{split}\mc &\Gamma, \Gamma', p; \Delta; \Xi_N; \Gamma_{N1}; \Delta_{N1}; \Xi; \Omega; [\Gamma'; \Delta; \Xi; \bang p; \cdot; \Omega; \Lambda; q, \Upsilon], C_1, C; P; AB; \Omega_N; \Delta_N \rightarrow \Xi'; \Delta'; \Gamma' \\ C_1 &= [\Gamma_{old}; \Delta_{old}; \Xi_{old}; \bang q; \Omega_{old}; \Lambda; \Upsilon]\end{split}}
\]


\[
\infer[\mc \bang p \; other \; qC]
{\mc \Gamma, \Gamma', p; \Delta; \Xi_N; \Gamma_{N1}; \Delta_{N1}; \Xi; \bang p, \Omega; C_1, C; P; AB; \Omega_N; \Delta_N \rightarrow \Xi'; \Delta'; \Gamma'}
{\begin{split}\mc &\Gamma, \Gamma', p; \Delta; \Xi_N; \Gamma_{N1}; \Delta_{N1}; \Xi; \Omega; [\Gamma'; \Delta; \Xi; \bang p; \cdot; \Omega; \Lambda, q; \Upsilon], C_1, C; P; AB; \Omega_N; \Delta_N \rightarrow \Xi'; \Delta'; \Gamma' \\ C_1 &= (\Delta_{old}; \Delta'_{old}; \Xi_{old}; q; \Omega_{old}; \Lambda; \Upsilon)\end{split}}
\]

\[
\infer[\mc \otimes]
{\mc \Delta; \Xi_N; \Delta_{N1}; \Xi; X \otimes Y, \Omega; C; P; \comp A \lolli B; \Omega_N; \Delta_N \rightarrow \Xi'; \Delta'}
{\mc \Delta; \Xi_N; \Delta_{N1}; \Xi; X, Y, \Omega; C; P; \comp A \lolli B; \Omega_N; \Delta_N \rightarrow \Xi'; \Delta'}
\]

\[
\infer[\mc end]
{\mc \Gamma; \Delta; \Xi_N; \Gamma_{N1}; \Delta_{N1}; \Xi; \cdot; C; P; \comp A \lolli B; \Omega_N; \Delta_N \rightarrow \Xi'; \Delta'; \Gamma'}
{\dall \Gamma; \Xi_N; \Gamma_{N1}; \Delta_{N1}; \Xi; C; P; \comp A \lolli B; \Omega_N; \Delta_N \rightarrow \Xi'; \Delta'; \Gamma'}
\]


\subsubsection{Match Comprehension Continuation}

When backtracking to a previous frame, we need to be carefully when using the stacks $C$ and $P$.


\[
\infer[\contc end]
{\contc \Gamma; \Delta_N; \Xi_N; \Gamma_{N1}; \Delta_{N1}; \cdot; \cdot; \comp A \lolli B; \Omega \rightarrow \Xi'; \Delta'; \Gamma'}
{\done \Gamma; \Delta_N; \Xi_N; \Gamma_{N1}; \Delta_{N1}; \Omega \rightarrow \Xi'; \Delta'; \Gamma'}
\]

\[
\infer[\contc nextC \; p]
{\contc \Gamma; \Delta_N; \Xi_N; \Gamma_{N1}; \Delta_{N1}; (\Delta; p_1, \Delta''; \Xi; p; \Omega; \Lambda; \Upsilon), C; P; AB; \Omega_N \rightarrow \Xi'; \Delta'; \Gamma'}
{\mc \Gamma; \Delta; \Xi_N; \Gamma_{N1}; \Delta_{N1}; \Xi; \Omega; (\Delta, p_1; \Delta''; \Xi; p; \Omega; \Lambda; \Upsilon), C; P; AB; \Omega_N; \Delta_N \rightarrow \Xi'; \Delta'; \Gamma'}
\]

\[
\infer[\contc nextC \; \bang p]
{\contc \Gamma; \Delta_N; \Xi_N; \Gamma_{N1}; \Delta_{N1}; [p_1, \Gamma'; \Delta; \Xi; \bang p; \Omega; \Lambda; \Upsilon], C; P; AB; \Omega_N \rightarrow \Xi'; \Delta'; \Gamma'}
{\mc \Gamma; \Delta; \Xi_N; \Gamma_{N1}; \Delta_{N1}; \Xi; \Omega; [\Gamma'; \Delta; \Xi; \bang p; \Omega; \Lambda; \Upsilon], C; P; AB; \Omega_N; \Delta_N \rightarrow \Xi'; \Delta'; \Gamma'}
\]

\[
\infer[\contc nextC \; empty \; p]
{\contc \Gamma; \Delta_N; \Xi_N; \Gamma_{N1}; \Delta_{N1}; (\Delta; \cdot; \Xi; p; \Omega; \Lambda; \Upsilon), C; P; AB; \Omega_N \rightarrow \Xi'; \Delta'; \Gamma'}
{\contc \Gamma; \Delta_N; \Xi_N; \Gamma_{N1}; \Delta_{N1}; C; P; AB; \Omega_N \rightarrow \Xi'; \Delta'; \Gamma'}
\]

\[
\infer[\contc nextC \; empty \; \bang p]
{\contc \Gamma; \Delta_N; \Xi_N; \Gamma_{N1}; \Delta_{N1}; [\cdot; \Delta; \Xi; \bang p; \Omega; \Lambda; \Upsilon], C; P; AB; \Omega_N \rightarrow \Xi'; \Delta'; \Gamma'}
{\contc \Gamma; \Delta_N; \Xi_N; \Gamma_{N1}; \Delta_{N1}; C; P; AB; \Omega_N \rightarrow \Xi'; \Delta'; \Gamma'}
\]

\[
\infer[\contc nextP \; \bang p]
{\contc \Gamma; \Delta_N; \Xi_N; \Gamma_{N1}; \Delta_{N1}; \cdot; [p_1, \Gamma'; \Delta_N; \cdot; \bang p; \Omega; \cdot; \Upsilon], P; AB; \Omega_N \rightarrow \Xi'; \Delta'; \Gamma'}
{\mc \Gamma; \Delta_N; \Xi_N; \Gamma_{N1}; \Delta_{N1}; \cdot; \Omega; \cdot; [\Gamma'; \Delta_N; \cdot; \bang p; \Omega; \cdot; \Upsilon], P; AB; \Omega_N; \Delta_N \rightarrow \Xi'; \Delta'; \Gamma'}
\]

\[
\infer[\contc nextP \; empty \; \bang p]
{\contc \Gamma; \Delta_N; \Xi_N; \Gamma_{N1}; \Delta_{N1}; \cdot; [\cdot; \Delta_N; \cdot; \bang p; \Omega; \cdot; \Upsilon], P; AB; \Omega_N \rightarrow \Xi'; \Delta'; \Gamma'}
{\contc \Gamma; \Delta_N; \Xi_N; \Gamma_{N1}; \Delta_{N1}; \cdot; P; AB; \Omega_N \rightarrow \Xi'; \Delta'; \Gamma'}
\]


\subsubsection{Stack Transformation}

Like the previous system, we need to transform continuation stack to be able to used again. First, we remove everything except the first frame in the $C$ stack. Next, we transform the $\Delta$ argument in the first frame of $C$ and in every frame of $P$ to remove recently consumed facts.

The $\strans \Xi; P; P'$ judgments transforms the $P$ stack and has the following meaning:
\begin{enumerate}
   \item[$\Xi$]: Consumed linear resources during the last application of the comprehension.
   \item[$P$]: The $P$ stack.
   \item[$P'$]: The transformed $P$ stack.
\end{enumerate}


\[
\infer[\strans]
{\strans \Xi; [\Gamma'; \Delta_N; \cdot; \bang p; \Omega; \cdot; \Upsilon], P; [\Gamma'; \Delta_N - \Xi; \cdot; \bang p, \Omega; \cdot; \Upsilon], P'}
{\strans \Xi; P; P'}
\]

\[
\infer[\strans end]
{\strans \Xi; \cdot; \cdot}
{\strans \Xi; \cdot; \cdot}
\]



\[
\infer[\dall end \; linear]
{\dall \Gamma; \Xi_N; \Gamma_{N1}; \Delta_{N1}; \Xi; (\Delta_x; \Delta''; \cdot; p; \Omega; \cdot; \Upsilon); P;  \comp A \lolli B; \Omega_N; \Delta_N \rightarrow \Xi'; \Delta'; \Gamma'}
{\begin{split}\strans &\Xi; P; P' \\ \dc \Gamma; \Xi_N, \Xi; \Gamma_{N1}; \Delta_{N1}; (\Delta_x - \Xi; \Delta'' - \Xi; \cdot; p; \Omega; \cdot; \Upsilon) ; P' ; \comp A \lolli B; \Omega_N; (\Delta_N - \Xi) &\rightarrow \Xi'; \Delta'; \Gamma'\end{split}}
\]


\[
\infer[\dall end \; empty]
{\dall \Gamma; \Xi_N; \Gamma_{N1}; \Delta_{N1}; \Xi; \cdot; P;  \comp A \lolli B; \Omega_N; \Delta_N \rightarrow \Xi'; \Delta'; \Gamma'}
{\begin{split}\strans &\Xi; P; P' \\ \dc \Gamma; \Xi_N, \Xi; \Gamma_{N1}; \Delta_{N1}; \cdot ; P' ; \comp A \lolli B; \Omega_N; (\Delta_N - \Xi) &\rightarrow \Xi'; \Delta'; \Gamma'\end{split}}
\]

\[
\infer[\dall more]
{\dall \Gamma; \Xi_N; \Gamma_{N1}; \Delta_{N1}; \Xi; \_, X, C; P; \comp A \lolli B; \Omega_N; \Delta_N \rightarrow \Xi'; \Delta'; \Gamma'}
{\dall \Gamma; \Xi_N; \Gamma_{N1}; \Delta_{N1}; \Xi; X, C; P; \comp A \lolli B; \Omega_N; \Delta_N \rightarrow \Xi'; \Delta'; \Gamma'}
\]



\subsubsection{Comprehension Derivation}

The $\dc$ is identical to the previous system, however it has been extended with $\Gamma$, $\Gamma_1$, $C$, $P$ and $\Gamma'$.


\[
\infer[\dc p]
{\dc \Gamma; \Xi; \Gamma_1; \Delta_1; p, \Omega; C; P; \comp A \lolli B; \Omega_N; \Delta_N \rightarrow \Xi'; \Delta'; \Gamma'}
{\dc \Gamma; \Xi; \Gamma_1; \Delta_1, p; \Omega; C; P; \comp A \lolli B; \Omega_N; \Delta_N \rightarrow \Xi'; \Delta'; \Gamma'}
\]

\[
\infer[\dc \bang p]
{\dc \Gamma; \Xi; \Gamma_1; \Delta_1; \bang p, \Omega; C; P; \comp A \lolli B; \Omega_N; \Delta_N \rightarrow \Xi'; \Delta'; \Gamma'}
{\dc \Gamma; \Xi; \Gamma_1, p; \Delta_1; \Omega; C; P; \comp A \lolli B; \Omega_N; \Delta_N \rightarrow \Xi'; \Delta'; \Gamma'}
\]

\[
\infer[\dc \otimes]
{\dc \Gamma; \Xi; \Gamma_1; \Delta_1; A \otimes B, \Omega; C; P; AB; \Omega_N; \Delta_N \rightarrow \Xi'; \Delta';\Gamma'}
{\dc \Gamma; \Xi; \Gamma_1; \Delta_1; A, B, \Omega; C; P; AB; \Omega_N; \Delta_N \rightarrow \Xi'; \Delta';\Gamma'}
\]

\[
\infer[\dc end]
{\dc \Gamma; \Xi; \Gamma_1; \Delta_1; \cdot; C; P; \comp A \lolli B; \Omega_N; \Delta_N \rightarrow \Xi'; \Delta'; \Gamma'}
{\contc \Gamma; \Delta_N; \Xi; \Gamma_1; \Delta_1; C; P; \comp A \lolli B; \Omega_N \rightarrow \Xi'; \Delta'; \Gamma'}
\]


\subsubsection{Application}

Finally, we add $\Gamma$ and $\Gamma'$ to the main inference rules to complete the system.

\[
\infer[\ao start \; matching]
{\ao \Gamma; \Delta; A \lolli B; R \rightarrow \Xi'; \Delta'; \Gamma'}
{\mo \Gamma; \Delta; \cdot; \cdot; A; B; \cdot; R \rightarrow \Xi'; \Delta'; \Gamma'}
\]

\[
\infer[\doo rule]
{\doo \Gamma; \Delta; R, \Phi \rightarrow \Xi'; \Delta';\Gamma'}
{\ao \Gamma; \Delta; R; (\Phi, \Delta) \rightarrow \Xi';\Delta';\Gamma'}
\]

\subsection{Definitions}

\input{ll-system/persistent-definitions}

\subsection{Theorems}

\subsubsection{Body Match Soundness Theorem}\label{body_match_theorem_persistent}

\begin{theorem}

If a $\mo \Gamma; \Delta_1, \Delta_2; \Xi; \Omega; H; C; R \rightarrow \Xi'; \Delta'; \Gamma'$ is related to term $A$ and a context $\Delta_1, \Delta_2, \Xi$ and a context $\Gamma$ then either:

\begin{enumerate}
   \item $\cont \cdot; H; R; \Gamma; \Xi'; \Delta'; \Gamma'$
   \item $\mz \Gamma; \Delta_x \rightarrow A$ (where $\Delta_x$ is a subset of $\Delta_1, \Delta_2, \Xi$) and one of the two sub-cases are true:
      \begin{enumerate}
         \item $\mo \Gamma; \Delta_1; \Xi, \Delta_2; \cdot; H; C'', C; R \rightarrow \Xi'; \Delta'; \Gamma'$ (related) and $\Delta_x = \Xi, \Delta_2$
         \item $\exists f = (\Delta_a; \Delta_{b_1}, p_2, \Delta_{b_2}; p; \Omega_1, ..., \Omega_k; \Xi_1, ..., \Xi_m; \Lambda_1, ..., \Lambda_m; \Upsilon_1, ..., \Upsilon_n) \in C$ where $C = C', f, C''$ and $f$ turns into $f' = (\Delta_a, \Delta_{b_1}, p_2; \Delta_{b_2}; p; \Omega_1, ..., \Omega_k; \Xi_1, ..., \Xi_m; \Lambda_1, ..., \Lambda_m; \Upsilon_1, ..., \Upsilon_n)$ such that:
         \begin{enumerate}
            \item $\mo \Gamma; \Delta_c; \Xi_1, ..., \Xi_m, p_2, \Xi_c; \cdot; H; C''', f', C''; R \rightarrow \Xi'; \Delta'; \Gamma'$ (related) where $\Delta_c = (\Delta_1, \Delta_2, \Xi) - (\Xi_1, ..., \Xi_m, p_2, \Xi_c)$
         \end{enumerate}
         \item $\exists f = [\Gamma_1, p_2, \Gamma_2; \Delta_{c_1}, \Delta_{c_2}; \Xi_c; \bang p; \Omega_1, ..., \Omega_k; \Lambda_1, ..., \Lambda_m; \Upsilon_1, ..., \Upsilon_n] \in C$ where $C = C', f, C''$ and $f$ turns into $f' = [\Gamma_2; \Delta_{c_1}, \Delta_{c_2}; \Xi_1, ..., \Xi_m; \bang p; \Omega_1, ..., \Omega_k; \Lambda_1, ..., \Lambda_m; \Upsilon_1, ..., \Upsilon_n]$ such that:
         \begin{enumerate}
            \item $\mo \Gamma; \Delta_{c_1}; \Xi_1, ..., \Xi_m, \Delta_{c_2}; \cdot; H; C'', f', C''; R \rightarrow \Xi'; \Delta'; \Gamma'$ (related) where $\Delta_{c_1}, \Delta_{c_2} = \Delta_1, \Delta_2, \Xi - (\Xi_1, ..., \Xi_m)$
         \end{enumerate}
      \end{enumerate}
\end{enumerate}

If $\cont C; H; R; \Gamma; \Xi'; \Delta'; \Gamma'$ and $C$ is well formed in relation to $A$ and $\Delta_1, \Delta_2, \Xi$ then either:

\begin{enumerate}
   \item $\cont \cdot; H; R; \Gamma; \Xi'; \Delta'; \Gamma'$
   \item $\mz \Delta_x \rightarrow A$ (where $\Delta_x \subseteq \Delta_1, \Delta_2, \Xi$) where one sub-case is true:
   \begin{enumerate}
      \item $\exists f = (\Delta_a; \Delta_{b_1}, p_2, \Delta_{b_2}; \Xi_1, ..., \Xi_m; p; \Omega_1, ..., \Omega_k; \Lambda_1, ..., \Lambda_m; \Upsilon_1, ..., \Upsilon_n) \in C$ where $C = C', f, C''$ and $f$ turns into $f' = (\Delta_a, \Delta_{b_1}, p_2; \Delta_{b_2}; p; \Omega_1, ..., \Omega_k; \Xi_1, ..., \Xi_m; \Lambda_1, ..., \Lambda_m; \Upsilon_1, ..., \Upsilon_n)$ such that:
      \begin{enumerate}
         \item $\mo \Gamma; \Delta_c; \Xi_1, ..., \Xi_m, p_2, \Xi_c; \cdot; H; C''', f', C''; R \rightarrow \Xi'; \Delta'; \Gamma'$ (related) where $\Delta_c = (\Delta_1, \Delta_2, \Xi) - (\Xi_1, ..., \Xi_m, p_2, \Xi_c)$
      \end{enumerate}
      \item $\exists f = [\Gamma_1, p_2, \Gamma_2; \Delta_{c_1}, \Delta_{c_2}; \Xi_c; \bang p; \Omega_1, ..., \Omega_k; \Lambda_1, ..., \Lambda_m; \Upsilon_1, ..., \Upsilon_n] \in C$ where $C = C', f, C''$ and $f$ turns into $f' = [\Gamma_2; \Delta_{c_1}, \Delta_{c_2}; \Xi_1, ..., \Xi_m; \bang p; \Omega_1, ..., \Omega_k; \Lambda_1, ..., \Lambda_m; \Upsilon_1, ..., \Upsilon_n]$ such that:
      \begin{enumerate}
         \item $\mo \Gamma; \Delta_{c_1}; \Xi_1, ..., \Xi_m, \Delta_{c_2}; \cdot; H; C'', f', C''; R \rightarrow \Xi'; \Delta'; \Gamma'$ (related) where $\Delta_{c_1}, \Delta_{c_2} = \Delta_1, \Delta_2, \Xi - (\Xi_1, ..., \Xi_m)$
      \end{enumerate}
   \end{enumerate}
\end{enumerate}
\end{theorem}

\begin{proof}
   Proof by mutual induction. In $\mo$ on the size of $\Omega$ and on $\cont$, first on the size of $\Delta''$ and $\Gamma''$ and then on the size of $C$. Use the stack constraints and the Match Equivalence Theorem to prove $\mz \Delta_x \rightarrow A$.

   \begin{enumerate}
      \item $\mo p \; ok \; first$
      
      By induction on $\Omega$.
      
      \item $\mo p \; ok \; other \; q$
      
      By induction on $\Omega$.
      
      \item $\mo p \; ok \; other \; \bang q$
      
      By induction on $\Omega$.
      
      \item $\mo p \; fail$
      
      Induction on the inverted $\cont$ judgment.
      
      \item $\mo \bang p \; ok \; first$
      
      Induction on $\Omega$.
      
      \item $\mo \bang p \; ok \; other \; \bang q$
      
      Induction on $\Omega$.
      
      \item $\mo \bang p \; fail$
      
      Induction on the inverted $\cont$ judgment.
      
      \item $\mo \otimes$
      
      Induction on $\Omega$.
      
      \item $\mo end$
      
      Use assumption.
      
      \item $\cont next \; rule$
      
      Use assumption.
      
      \item $\cont p \; next$
      
      Induction on $\Delta''$.
      
      \item $\cont p \; no \; more$
      
      Induction on $C$.
      
      \item $\cont \bang p \; next$
      
      Induction on $\Gamma'$.
      
      \item $\cont \bang p \; no \; more$
      
      Induction on $C$.
   \end{enumerate}
\end{proof}

\subsubsection{Body Match Soundness Lemma}\label{body_match_lemma_persistent}

\begin{lemma}
   Given a match $\mo \Gamma; \Delta_1, \Delta_2; \cdot; A; B; \cdot; R \rightarrow \Xi'; \Delta'; \Gamma'$ that is related to $A$, $\Delta_1, \Delta_2$ and $\Gamma$, we get either:
   
   \begin{enumerate}
      \item $\cont \cdot; B; R; \Gamma; \Xi'; \Delta'; \Gamma'$;
      \item $\mz \Delta_2 \rightarrow A$:
      \begin{enumerate}
         \item $\mo \Gamma; \Delta_1; \Delta_2; \cdot; B; C'; R \rightarrow \Xi'; \Delta'; \Gamma'$ (related)
      \end{enumerate}
   \end{enumerate}
   
\end{lemma}

\begin{proof}
   Use the Body Match Soundness Theorem.
\end{proof}

\subsubsection{Comprehension Match Soundness Theorem}

\begin{theorem}
   \begin{itemize}
      \item If a match $\mc \Gamma; \Delta_1, \Delta_2; \Xi_N; \Gamma_{N1}; \Delta_{N1}; \Xi; \Omega; C; P; \comp A \lolli B; \Omega_N; \Delta_N \rightarrow \Xi'; \Delta'; \Gamma'$ is related to term $A$, context $\Delta_N = \Delta_1, \Delta_2, \Xi$ and context $\Gamma$ then either:
      \begin{enumerate}
         \item $\done \Gamma; \Delta_N; \Xi_N; \Gamma_{N1}; \Delta_{N1}; \Omega_N \rightarrow \Xi'; \Delta'; \Gamma'$;
         \item $\mz \Delta_x \rightarrow A$ (where $\Delta_x \subseteq \Delta_N$) and one of the following sub-cases is true:
         \begin{enumerate}
            \item $\mc \Gamma; \Delta_1; \Xi_N; \Gamma_{N1}; \Delta_{N1}; \Xi, \Delta_2; \cdot; C', C; P; \comp A \lolli B; \Omega_N; \Delta_N \rightarrow \Xi'; \Delta'; \Gamma'$ (related) and $\Delta_x = \Delta_2$, if $C \neq \cdot$
            \item $\mc \Gamma; \Delta_1; \Xi_N; \Gamma_{N1}; \Delta_{N1}; \Xi, \Delta_2; \cdot; C'; P', P; \comp A \lolli B; \Omega_N; \Delta_N \rightarrow \Xi'; \Delta'; \Gamma'$ (related) and $\Delta_x = \Delta_2$, if $C = \cdot$.
            \item $\exists f = (\Delta_a; \Delta_{b_1}, p_2, \Delta_{b_2}; p; \Xi_1, ..., \Xi_n; \Omega_1, ..., \Omega_k; \Lambda_1, ..., \Lambda_n; \Upsilon_1, ..., \Upsilon_m) \in C$ where $C = C', f, C''$ that turns into $f' = (\Delta_a, \Delta_{b_1}, p_2; \Delta_{b_2}; p; \Xi_1, ..., \Xi_n; \Omega_1, ..., \Omega_k; \Lambda_1, ..., \Lambda_n; \Upsilon_1, ..., \Upsilon_m)$ such that:
               \begin{enumerate}
                  \item $\mc \Gamma; \Delta_c; \Xi_N; \Gamma_{N1}; \Delta_{N1}; \Xi_1, ..., \Xi_n, \Xi_c; \cdot; C''', f', C''; P; \comp A \lolli B; \Omega_N; \Delta_N \rightarrow \Xi'; \Delta'; \Gamma'$ (related)
               \end{enumerate}
               
            \item $\exists f = [\Gamma_1, p_2, \Gamma_2; \Delta_{c_1}, \Delta_{c_2}; \Xi_c; \bang p; \Omega_1, ..., \Omega_k; \Lambda_1, ..., \Lambda_n; \Upsilon_1, ..., \Upsilon_m] \in C$ where $C = C', f, C''$ that turns into $f' = [\Gamma_2; \Delta_{c_1}, \Delta_{c_2}; \Xi_c; \bang p; \Omega_1, ..., \Omega_k; \Lambda_1, ..., \Lambda_n; \Upsilon_1, ..., \Upsilon_m]$ such that:
               \begin{enumerate}
                  \item $\mc \Gamma; \Delta_{c_1}; \Xi_N; \Gamma_{N1}; \Delta_{N1}; \Delta_{c_2}, \Xi_c; \cdot; C''', f', C''; P; \comp A \lolli B; \Omega_N; \Delta_N \rightarrow \Xi'; \Delta'; \Gamma'$ (related)
               \end{enumerate}
               
            \item $\exists f = [\Gamma_1, p_2, \Gamma_2; \Delta_{c_1}, \Delta_{c_2}; \Xi_c; \bang p; \Omega_1, ..., \Omega_k; \Lambda_1, ..., \Lambda_n; \Upsilon_1, ..., \Upsilon_m] \in P$ where $P = P', f, P''$ that turns into $f' = [\Gamma_2; \Delta_{c_1}, \Delta_{c_2}; \Xi_c; \bang p; \Omega_1, ..., \Omega_k; \Lambda_1, ..., \Lambda_n; \Upsilon_1, ..., \Upsilon_m]$ such that:
                  \begin{enumerate}
                     \item $\mc \Gamma; \Delta_{c_1}; \Xi_N; \Gamma_{N1}; \Delta_{N1}; \Delta_{c_2}, \Xi_c; \cdot; C'; P''', f', P''; \comp A \lolli B; \Omega_N; \Delta_N \rightarrow \Xi'; \Delta'; \Gamma'$ (related)
                  \end{enumerate}
         \end{enumerate}
      \end{enumerate}
      
   \item If $\contc \Gamma; \Delta_{N}; \Xi_{N}; \Gamma_{N1}; \Delta_{N1}; C; P; \comp A \lolli B; \Omega_N \rightarrow \Xi'; \Delta'; \Gamma'$ and $C$ and $P$ is well formed in relation to $A; \Gamma; \Delta_N$ then either:
   
   \begin{enumerate}
      \item $\exists f = (\Delta_a; \Delta_{b_1}, p_2, \Delta_{b_2}; p; \Xi_1, ..., \Xi_n; \Omega_1, ..., \Omega_k; \Lambda_1, ..., \Lambda_n; \Upsilon_1, ..., \Upsilon_m) \in C$ where $C = C', f, C''$ that turns into $f' = (\Delta_a, \Delta_{b_1}, p_2; \Delta_{b_2}; p; \Xi_1, ..., \Xi_n; \Omega_1, ..., \Omega_k; \Lambda_1, ..., \Lambda_n; \Upsilon_1, ..., \Upsilon_m)$ such that:
         \begin{enumerate}
            \item $\mc \Gamma; \Delta_c; \Xi_N; \Gamma_{N1}; \Delta_{N1}; \Xi_1, ..., \Xi_n, \Xi_c; \cdot; C''', f', C''; P; \comp A \lolli B; \Omega_N; \Delta_N \rightarrow \Xi'; \Delta'; \Gamma'$ (related)
         \end{enumerate}
         
      \item $\exists f = [\Gamma_1, p_2, \Gamma_2; \Delta_{c_1}, \Delta_{c_2}; \Xi_c; \bang p; \Omega_1, ..., \Omega_k; \Lambda_1, ..., \Lambda_n; \Upsilon_1, ..., \Upsilon_m] \in C$ where $C = C', f, C''$ that turns into $f' = [\Gamma_2; \Delta_{c_1}, \Delta_{c_2}; \Xi_c; \bang p; \Omega_1, ..., \Omega_k; \Lambda_1, ..., \Lambda_n; \Upsilon_1, ..., \Upsilon_m]$ such that:
         \begin{enumerate}
            \item $\mc \Gamma; \Delta_{c_1}; \Xi_N; \Gamma_{N1}; \Delta_{N1}; \Delta_{c_2}, \Xi_c; \cdot; C''', f', C''; P; \comp A \lolli B; \Omega_N; \Delta_N \rightarrow \Xi'; \Delta'; \Gamma'$ (related)
         \end{enumerate}
         
      \item $\exists f = [\Gamma_1, p_2, \Gamma_2; \Delta_{c_1}, \Delta_{c_2}; \Xi_c; \bang p; \Omega_1, ..., \Omega_k; \Lambda_1, ..., \Lambda_n; \Upsilon_1, ..., \Upsilon_m] \in P$ where $P = P', f, P''$ that turns into $f' = [\Gamma_2; \Delta_{c_1}, \Delta_{c_2}; \Xi_c; \bang p; \Omega_1, ..., \Omega_k; \Lambda_1, ..., \Lambda_n; \Upsilon_1, ..., \Upsilon_m]$ such that:
            \begin{enumerate}
               \item $\mc \Gamma; \Delta_{c_1}; \Xi_N; \Gamma_{N1}; \Delta_{N1}; \Delta_{c_2}, \Xi_c; \cdot; C'; P''', f', P''; \comp A \lolli B; \Omega_N; \Delta_N \rightarrow \Xi'; \Delta'; \Gamma'$ (related)
            \end{enumerate}
   \end{enumerate}
   \end{itemize}
\end{theorem}

\begin{proof}
   Mutual induction on $\mo$ and $\contc$. $\mo$ on the size of $\Omega$ and $\contc$ first on the size of $\Delta'$ or $\Gamma'$ and then on the size of $C$ and $P$ merged. We use the same approach as in Section~\ref{sec:comprehension_match_soundness}.
\end{proof}

\subsubsection{Comprehension Match Soundness Lemma}

\begin{lemma}
If $\mc \Gamma; \Delta, \Xi; \Xi_N; \Gamma_{N1}; \Delta_{N1}; \cdot; A; \cdot; \cdot; \comp A \lolli B; \Omega_N; \Delta_N \rightarrow \Xi'; \Delta'; \Gamma'$ and the match is related to both $A$, $\Delta_N$ and $\Gamma$ then either:
\begin{enumerate}
   \item $\done \Gamma; \Delta_N; \Xi_N; \Gamma_{N1}; \Delta_{N1}; \Omega_N \rightarrow \Xi'; \Delta'; \Gamma'$;
   \item $\mc \Gamma; \Delta; \Xi_N; \Gamma_{N1}; \Delta_{N1}; \Xi; \cdot; C'; P'; \comp A \lolli B; \Omega_N; \Delta_N \rightarrow \Xi'; \Delta'; \Gamma'$ and
   \begin{enumerate}
      \item $\mz \Gamma; \Xi \rightarrow A$
      \item $C'$ and $P'$ are well formed in relation to $A; \Gamma; \Delta, \Xi$.
   \end{enumerate}
\end{enumerate}
\end{lemma}

\begin{proof}
   Direct application of the previous theorem.
\end{proof}

\subsubsection{Strans Theorem}

\begin{theorem}
   If $\strans \Xi; P; P'$ then $P'$ will be the transformation of stack $P$ where $\forall f = [\Gamma'; \Delta_N; \cdot; \bang p; \Omega; \cdot; \Upsilon] \in P$ will turn into $f' = [\Gamma'; \Delta_N - \Xi; \cdot; \bang p; \Omega; \cdot; \Upsilon]$.
\end{theorem}

\begin{proof}
   Induction on the size of $P$.
\end{proof}

\subsubsection{Comprehension Transformation Theorem}

\begin{theorem}
   If $\dall \Gamma; \Xi_N; \Gamma_{N1}; \Delta_{N1}; \Xi; C; P; \comp A \lolli B; \Omega_N; \Delta_N \rightarrow \Xi'; \Delta'; \Gamma'$ then \\
      $\dc \Gamma; \Xi_N, \Xi; \Gamma_{N1}; \Delta_{N1}; B; C' ; P'; \comp A \lolli B; \Omega_N; (\Delta_N - \Xi) \rightarrow \Xi'; \Delta'; \Gamma'$, where:
   
   \begin{enumerate}
      \item If $C = \cdot$ then $C' = \cdot$;
      \item If $C = C_1, (\Delta_a; \Delta_b; \cdot; p; \Omega; \cdot; \Upsilon)$ then $C' = (\Delta_a - \Xi; \Delta_b - \Xi; \cdot; p; \Omega; \cdot; \Upsilon)$;
      \item $P'$ is the transformation of stack $P$, where $\forall f = [\Gamma'; \Delta_N; \cdot; \bang p; \Omega; \cdot; \Upsilon] \in P$ will turn into $f' = [\Gamma'; \Delta_N - \Xi; \cdot; \bang p; \Omega; \cdot; \Upsilon]$.
   \end{enumerate}
\end{theorem}

\begin{proof}
   Use induction on the size of the stack $C$ to get the result of $C'$ then apply the strans theorem to get $P'$.
\end{proof}

\subsubsection{Successful Comprehension Match Gives Derivation}

We can apply the previous theorem to know that after a successful matching we will start the derivation process:

\begin{lemma}
   If $\mc \Gamma; \Delta; \Xi_N; \Gamma_{N1}; \Delta_{N1}; \Xi; \cdot; B; C; P; \comp A \lolli B;\Omega_N; \Delta_N \rightarrow \Xi'; \Delta'; \Gamma'$ then \\
      $\dc \Gamma; \Xi_N, \Xi; \Gamma_{N1}; \Delta_{N1}; B; C'; P'; \comp A \lolli B; \Omega_N; (\Delta_N - \Xi) \rightarrow \Xi'; \Delta'; \Gamma'$ where:
   
   \begin{enumerate}
      \item If $C = \cdot$ then $C' = \cdot$;
      \item If $C = C_1, (\Delta_a; \Delta_b; \cdot; p; \Omega; \cdot; \Upsilon)$ then $C' = (\Delta_a - \Xi; \Delta_b - \Xi; \cdot; p; \Omega; \cdot; \Upsilon)$ then $C' = (\Delta_a - \Xi; \Delta_b - \Xi; \cdot; p; \Omega; \cdot; \Upsilon)$;
      \item $P'$ is the transformation of stack $P$, where $\forall f = [\Gamma'; \Delta_N; \cdot; \bang p; \Omega; \cdot; \Upsilon] \in P$ will turn into $f' = [\Gamma'; \Delta_N - \Xi; \cdot; \bang p; \Omega; \cdot; \Upsilon]$.
   \end{enumerate}
\end{lemma}

\begin{proof}
   Invert the assumption and then apply the comprehension transformation theorem.
\end{proof}

\subsubsection{Comprehension Derivation Theorem}

We prove that if we try to derive the head of the comprehension, we will finish this process and that giving a derivation at the high level with the same results we can restore the terms we have derived. We will use this result to prove the soundness of the comprehension mechanism.

\begin{theorem}
   If $\dc \Gamma; \Xi_N; \Gamma_{N1}; \Delta_{N1}; \Omega_1, ..., \Omega_n; C; P; \comp A \lolli B; \Omega_N; \Delta_N \rightarrow \Xi'; \Delta'; \Gamma'$ then:
   
   \begin{enumerate}
      \item $\dc \Gamma; \Xi_N; \Gamma_{N1}, \Gamma_1, ..., \Gamma_n; \Delta_{N1}, \Delta_1, ..., \Delta_n; \cdot; C; P; \comp A \lolli B; \Omega_N; \Delta_N \rightarrow \Xi'; \Delta'; \Gamma'$;
      \item If $\dz \Gamma; \Delta; \Xi_N; \Gamma_{N1}, \Gamma_1, ..., \Gamma_n; \Delta_{N1}, \Delta_1, ..., \Delta_n; \Omega \rightarrow \Xi'; \Delta'; \Gamma'$ then $\dz \Gamma; \Delta; \Xi_N; \Gamma_{N1}; \Delta_{N1}; \Omega_1, ..., \Omega_n, \Omega \rightarrow \Xi'; \Delta'; \Gamma'$
   \end{enumerate}
\end{theorem}

\begin{proof}
   Induction on $\Omega_1, ..., \Omega_n$.
\end{proof}

\subsubsection{Comprehension Derivation Lemma}

Now using the previous theorem we prove that we can do the same for the head of the comprehension.

\begin{theorem}
   If $\dc \Gamma; \Xi_N; \Gamma_{N1}; \Delta_{N1}; B; C; P; \comp A \lolli B; \Omega_N; \Delta_N \rightarrow \Xi'; \Delta'; \Gamma'$ then:
   
   \begin{enumerate}
      \item $\dc \Gamma; \Xi_N; \Gamma_{N1}, \Gamma_B; \Delta_{N1}, \Delta_B; \cdot; C; P; \comp A \lolli B; \Omega_N; \Delta_N \rightarrow \Xi'; \Delta'; \Gamma'$;
      \item If $\dz \Gamma; \Delta; \Xi_N; \Gamma_{N1}, \Gamma_B; \Delta_{N1}, \Delta_B; \Omega \rightarrow \Xi'; \Delta'; \Gamma'$ then $\dz \Gamma; \Delta; \Xi_N; \Gamma_{N1}; \Delta_{N1}; B, \Omega \rightarrow \Xi'; \Delta'; \Gamma'$.
   \end{enumerate}
\end{theorem}

\begin{proof}
   Use the theorem above.
\end{proof}

\subsubsection{Comprehension Theorem}

If we have a matching process with at most a continuation frame in $C$ and a continuation stack $P$, where $C, P$ are not empty, we can derive $n \neq 0$ comprehensions and have $n$ valid matching processes and $n$ derivations at the high level. Since $C, P$ will be reduced in size, either by frame use or the arguments in the frames get reduced each time a comprehension is applied, we have a valid induction hypothesis.

\begin{theorem}
   If $\mc \Gamma; \Delta_a, \Delta'_b; \Xi_N; \Gamma_{N1}; \Delta_{N1}; p_1; \Omega; (\Delta_a, p_1; \Delta'_b; \cdot; p; \Omega; \cdot; \Upsilon); P; \comp A \lolli B; \Omega_N; \Delta, \Xi_1, ..., \Xi_n \rightarrow \Xi'; \Delta'; \Gamma'$ (related to $A$, $\Delta_a, \Delta_b, p_1 = \Delta_N$ and $\Gamma$) and $\Delta_a, \Delta_b, p_1 = \Delta, \Xi_1, ..., \Xi_n$  then $\exists n \geq 0$ such that:
   
   \begin{enumerate}
      \item $\done \Gamma; \Delta_N; \Xi_N, \Xi_1, ..., \Xi_n; \Gamma_{N1}, \Gamma_1, ..., \Gamma_n; \Delta_{N1}, \Delta_1, ..., \Delta_n; \Omega_N \rightarrow \Xi'; \Delta'; \Gamma'$
      \item $\mz \Gamma; \Xi_1 \rightarrow A$ ... $\mz \Gamma; \Xi_n \rightarrow A$
      \item $n$ implications from $1...i...n$ such that: $\forall \Omega_x, \Delta_x.$ if $\done \Gamma; \Delta_x; \Xi_N, \Xi_1, ..., \Xi_i; \Gamma_{N1}, \Gamma_1, ..., \Gamma_i; \Delta_{N1}, \Delta_1, ..., \Delta_i; \Omega_x \rightarrow \Xi'; \Delta'; \Gamma'$ then $\dz \Gamma; \Delta_x; \Xi_N, \Xi_1, ..., \Xi_i; \Delta_{N1}, \Delta_1, ..., \Delta_{i-1}; B, \Omega_X \rightarrow \Xi'; \Delta'$
   \end{enumerate}
   
   If $\mc \Gamma; \Delta_N; \Xi_N; \Gamma_{N1}; \Delta_{N1}; \cdot; \Omega; \cdot; P; \comp A \lolli B; \Omega_N; \Delta, \Xi_1, ..., \Xi_n \rightarrow \Xi'; \Delta'; \Gamma'$ (related to $A$, $\Delta_N$ and $\Gamma$) and $\Delta_N = \Delta, \Xi_1, ..., \Xi_n$ then $\exists n \geq 0$ such that:
   
   \begin{enumerate}
      \item $\done \Gamma; \Delta_N; \Xi_N, \Xi_1, ..., \Xi_n; \Gamma_{N1}, \Gamma_1, ..., \Gamma_n; \Delta_{N1}, \Delta_1, ..., \Delta_n; \Omega_N \rightarrow \Xi'; \Delta'; \Gamma'$
      \item $\mz \Gamma; \Xi_1 \rightarrow A$ ... $\mz \Gamma; \Xi_n \rightarrow A$
      \item $n$ implications from $1...i...n$ such that: $\forall \Omega_x, \Delta_x.$ if $\done \Gamma; \Delta_x; \Xi_N, \Xi_1, ..., \Xi_i; \Gamma_{N1}, \Gamma_1, ..., \Gamma_i; \Delta_{N1}, \Delta_1, ..., \Delta_i; \Omega_x \rightarrow \Xi'; \Delta'; \Gamma'$ then $\dz \Gamma; \Delta_x; \Xi_N, \Xi_1, ..., \Xi_i; \Delta_{N1}, \Delta_1, ..., \Delta_{i-1}; B, \Omega_X \rightarrow \Xi'; \Delta'$
   \end{enumerate}
\end{theorem}

\begin{proof}
   By mutual induction, first on either the size of $\Delta'_b$ (first argument) or $\Gamma'$ (second argument) and then on the size of $C, P$.
   
   \begin{enumerate}
      \item first implication:
      
      $\Delta_b = p_1, \Delta'_b$ \hfill (1) from assumption \\
      $\Delta_a, p_1, \Delta'_b = \Delta, p_1, \Xi'_1, ..., \Xi_n = \Delta_N$ ($\Xi_1 = p_1, \Xi_1$) \hfill (2) from assumption \\
      By applying the comprehension soundness theorem to the assumption, we get:
      
      \begin{itemize}
         \item Failure:
         
         $\done \Gamma; \Delta_N; \Xi_N; \Gamma_{N1}; \Delta_{N1}; \Omega_N \rightarrow \Xi'; \Delta'; \Gamma'$ \hfill (3) from theorem (so $n = 0$)\\
         
         \item Success:
         
         $\mz \Gamma; \Xi_1 \rightarrow A$ \hfill (3) from theorem \\
         
         \begin{enumerate}
            \item Without backtracking:
            
            $\mc \Gamma; \Delta, \Xi_2, ..., \Xi_n; \Xi_N; \Gamma_{N1}; \Delta_{N1}; p_1, \Xi'_1; \cdot; C', (\Delta_a, p_1; \Delta'_b; \cdot; p; \Omega; \cdot; \Upsilon); P; \comp A \lolli B; \Omega_N; \Delta_N \rightarrow \Xi'; \Delta'; \Gamma'$ \hfill (4) from theorem \\
            $\dc \Gamma; \Xi_N, \Xi_1; \Gamma_{N1}; \Delta_{N1}; B; (\Delta_a; \Delta'_b - (\Xi_1); \cdot; p; \Omega; \cdot; \Upsilon); P; \comp A \lolli B; \Omega_N; \Delta_, \Xi_2, ..., \Xi_n \rightarrow \Xi'; \Delta'; \Gamma'$ \hfill (5) using successful comprehension matches gives derivation lemma to (4) \\
              $\dc \Gamma; \Xi_N, \Xi_1; \Gamma_{N1}, \Gamma_1; \Delta_{N1}, \Delta_1; \cdot; (\Delta_a; \Delta'_b - (\Xi_1); \cdot; p; \Omega; \cdot; \Upsilon); P; \comp A \lolli B; \Omega_N; \Delta, \Xi_2, ..., \Xi_n \rightarrow \Xi'; \Delta'$ \hfill (6) applying comprehension derivation lemma on (5) \\
              if $\forall \Omega_x, \Delta_x. \dz \Gamma; \Delta_x; \Xi_N, \Xi_1; \Gamma_{N1}, \Gamma_1; \Delta_{N1}, \Delta_1; \Omega_x \rightarrow \Xi'; \Delta'; \Gamma'$ then $\dz \Gamma; \Delta_x; \Xi_N, \Xi_1; \Gamma_{N1}; \Delta_{N1}; B, \Omega_x \rightarrow \Xi'; \Delta'; \Gamma'$ \hfill (7) from the same lemma \\
              $\contc \Gamma; \Delta, \Xi_2, ..., \Xi_n; \Xi_N, \Xi_1; \Gamma_{N1}, \Gamma_1; \Delta_{N1}, \Delta_1; (\Delta_a; \Delta'_b - (\Xi_1); \cdot; p; \Omega; \cdot; \Upsilon); P; \comp A \lolli B; \Omega_N \rightarrow \Xi'; \Delta'; \Gamma'$ \hfill (8) inversion of (7) \\
              
              By inverting (8) we either fail ($n = 1$) or we get a new match. For the latter case, we apply mutual induction to get the remaining $n - 1$ comprehensions.
            
            \item With backtracking:
            
            \begin{enumerate}
               \item Linear frame
               
               $f = (\Delta_a, p_1; \Delta'_b; \cdot; p; \Omega; \cdot; \Upsilon)$ \hfill (4) from theorem \\
               turns into $f' = (\Delta_a, p_1, \Delta'''_b, p_2; \Delta''_b; \cdot; p; \Omega; \cdot; \Upsilon)$ \hfill (5) from theorem \\
               $\mc \Gamma; \Delta, \Xi_2, ..., \Xi_n; \Xi_N; \Gamma_{N1}; \Delta_{N1}; p_2, \Xi'_1; \cdot; C', f'; P; \comp A \lolli B; \Omega_N; \Delta_N \rightarrow \Xi'; \Delta'; \Gamma'$ \hfill (6) from theorem \\
               
               Use the same approach as the case without backtracking.
               
               
               \item Persistent frame
               
               $f = [\Gamma''_1, p_2, \Gamma''_2; \Delta_N; \cdot; \bang p; \Omega; \cdot; \Upsilon]$ \hfill (4) from theorem \\
               turns into $f' = [\Gamma''_2; \Delta_N; \cdot; \bang p; \Omega; \cdot; \Upsilon]$ \hfill (5) from theorem \\
               $\mc \Gamma; \Delta, \Xi_2, ..., \Xi_n; \Xi_N; \Gamma_{N1}; \Delta_{N1}; \Xi_1; \cdot; C'; P', f', P''; \comp A \lolli B; \Omega_N; \Delta_N \rightarrow \Xi'; \Delta'; \Gamma'$ \hfill (6) from theorem \\
               
               Use the same approach as the case without backtracking.
            \end{enumerate}
         \end{enumerate}
      \end{itemize}
      
      \item second implication:
      
      Use the same approach as the one used in the first implication.
   \end{enumerate}
\end{proof}

\subsubsection{Comprehension Lemma}

We prove that by starting with empty continuation stacks, we get $n$ comprehensions.

\begin{lemma}
   If $\mc \Gamma; \Delta, \Xi_1, ..., \Xi_n; \Xi_N; \Gamma_{N1}; \Delta_{N1}; \cdot; A; \cdot; \cdot; \comp A \lolli B; \Omega_N; \Delta, \Xi_1, ..., \Xi_n \rightarrow \Xi'; \Delta'; \Gamma'$ (related to $A$, $\Delta, \Xi_1, ..., \Xi_n = \Delta_N$ and $\Gamma$) then $\exists n \geq 0$ such that:
   
   \begin{enumerate}
      \item $\done \Gamma; \Delta_N; \Xi_N, \Xi_1, ..., \Xi_n; \Gamma_{N1}, \Gamma_1, ..., \Gamma_n; \Delta_{N1}, \Delta_1, ..., \Delta_n; \Omega_N \rightarrow \Xi'; \Delta'; \Gamma'$
      \item $\mz \Gamma; \Xi_1 \rightarrow A$ ... $\mz \Gamma; \Xi_n \rightarrow A$
      \item $n$ implications from $1...i...n$ such that: $\forall \Omega_x, \Delta_x.$ if $\done \Gamma; \Delta_x; \Xi_N, \Xi_1, ..., \Xi_i; \Gamma_{N1}, \Gamma_1, ..., \Gamma_i; \Delta_{N1}, \Delta_1, ..., \Delta_i; \Omega_x \rightarrow \Xi'; \Delta'; \Gamma'$ then $\dz \Gamma; \Delta_x; \Xi_N, \Xi_1, ..., \Xi_i; \Delta_{N1}, \Delta_1, ..., \Delta_{i-1}; B, \Omega_X \rightarrow \Xi'; \Delta'$
   \end{enumerate}
\end{lemma}

\begin{proof}
   If we apply the comprehension soundness lemma, we get two cases:
   
   \begin{enumerate}
      \item Failure:
      
      $\done \Gamma; \Delta_N; \Xi_N; \Gamma_{N1}; \Delta_{N1}; \Omega_N \rightarrow \Xi'; \Delta'; \Gamma'$ \hfill (1) no comprehension application was possible ($n = 0$)\\
      
      \item Success:
      
      $\mc \Gamma; \Xi_2, ..., \Xi_n; \Xi_N; \Gamma_{N1}; \Delta_{N1}; \Xi_1; \cdot; C; P; \comp A \lolli B; \Omega_N; \Delta_N \rightarrow \Xi'; \Delta'; \Gamma'$ (related) \hfill (1) result from theorem \\
      $\mz \Gamma; \Xi_1 \rightarrow A$ \hfill (2) from theorem \\
      $\dc \Gamma; \Xi_N, \Xi_1; \Gamma_{N1}; \Delta_{N1}; B; C'; P'; \comp A \lolli B; \Omega_N; \Delta, \Xi_2, ..., \Xi_n \rightarrow \Xi'; \Delta'; \Gamma'$ \hfill (3) applying successful comprehension match gives derivation \\
      $\dc \Gamma; \Xi_N, \Xi_1; \Gamma_{N1}, \Gamma_1; \Delta_{N1}, \Delta_1; \cdot; C'; P'; \comp A \lolli B; \Omega_N; \Delta, \Xi_2, ..., \Xi_n \rightarrow \Xi'; \Delta'; \Gamma'$ \hfill (4) using comprehension derivation lemma \\
      if $\forall \Omega_x, \Delta_x. \dz \Gamma; \Delta_x; \Xi_N, \Xi_1; \Gamma_{N1}, \Gamma_1; \Delta_{N1}, \Delta_1; \Omega_x \rightarrow \Xi'; \Delta'; \Gamma'$ then $\dz \Gamma; \Delta_x; \Xi_N, \Xi_1; \Gamma_{N1}; \Delta_{N1}; B, \Omega_x \rightarrow \Xi'; \Delta'; \Gamma'$ \hfill (5) from the same lemma \\
      $\contc \Gamma; \Delta, \Xi_2, ..., \Xi_n; \Xi_N, \Xi_1; \Gamma_{N1}, \Gamma_1; \Delta_{N1}, \Delta_1; C'; P'; \comp A \lolli B; \Omega_N \rightarrow \Xi'; \Delta'; \Gamma'$ \hfill (6) inversion of (5) \\
      
      Invert (6) to get either $0$ applications of the comprehension or apply the comprehension theorem to the inversion to get the remaining $n-1$. 
   \end{enumerate}
\end{proof}

\subsubsection{Derivation Soundness Theorem}

\begin{theorem}
   If $\done \Gamma; \Delta; \Xi; \Gamma_1; \Delta_1; \Omega \rightarrow \Xi'; \Delta'; \Gamma'$ then $\dz \Gamma; \Delta; \Gamma_1; \Delta_1; \Omega \rightarrow \Xi'; \Delta'; \Gamma'$.
\end{theorem}

\begin{proof}
   Use the same approach used in \ref{sec:derivation_theorem}, but using the theorems presented in this section.
\end{proof}



\section{Low Level System With Aggregates}


\end{document}

