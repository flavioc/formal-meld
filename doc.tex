\documentclass[9pt]{article}
% \documentclass[11pt,twoside]{article}

\usepackage{graphicx}
\usepackage{textcomp}
\usepackage{comment}
\usepackage{proof-dashed}
\usepackage{url}
\usepackage{amsmath}
\usepackage{turnstile}
\usepackage[in]{fullpage}

\usepackage{latexsym}
\usepackage{amssymb}            % for \multimap (-o)
\usepackage{stmaryrd}           % for \binampersand (&), \bindnasrepma (\paar)

\newcommand{\m}[1]{\mathsf{#1}}
\newcommand{\f}[1]{\framebox{#1}}

\newcommand{\eph}{\mathit{eph}}
\newcommand{\pers}{\mathit{pers}}
\newcommand{\um}[1]{\underline{\m{#1}}}

\newcommand{\seq}{\vdash}
\newcommand{\semi}{\mathrel{;}}
\newcommand{\lequiv}{\mathrel{\dashv\vdash}}

% symbols of linear logic
\newcommand{\lolli}{\multimap}
\newcommand{\tensor}{\otimes}
\newcommand{\with}{\mathbin{\binampersand}}
\newcommand{\paar}{\mathbin{\bindnasrepma}}
\newcommand{\one}{\mathbf{1}}
\newcommand{\zero}{\mathbf{0}}
\newcommand{\bang}{{!}}
\newcommand{\whynot}{{?}}
\newcommand{\bilolli}{\mathrel{\raisebox{1pt}{\ensuremath{\scriptstyle\circ}}{\lolli}}}
% \oplus, \top, \bot



\title{Meld 2.0 Semantics}
\author{Flavio Cruz}

\begin{document}

\newcommand{\defeq}{\buildrel\triangle\over =}
\newcommand{\trnstile}{\sststile{}{}}
\newcommand{\typ}[1]{\m{#1} \; \m{typ}}
\newcommand{\typelist}[1]{\m{#1} \; \m{type \; list}}
\newcommand{\eexpr}[2]{\m{#1}:\m{#2}}
\newcommand{\aexp}[4]{#1;#2 \sststile{}{} \eexpr{#3}{#4}}
\newcommand{\expr}[3]{\aexp{\Psi}{#1}{#2}{#3}}
\newcommand{\tab}[0]{\;\;\;\;}
\newcommand{\elet}[3]{\m{let} \; #1 \; = \; #2 \; \m{in} \; #3 \; \m{end}}
\newcommand{\const}[2]{\m{const}(\mathit{#1}, #2)}
\newcommand{\getconst}[1]{\m{getconst}(\mathit{#1})}
\newcommand{\external}[2]{\m{external}(\mathit{#1}, #2)}
\newcommand{\callexternal}[2]{\m{callexternal}(\mathit{#1}, #2)}
\newcommand{\fun}[3]{\m{fun}(\mathit{#1}, #2, #3)}
\newcommand{\callfun}[2]{\m{callfun}(\mathit{#1}, #2)}
\newcommand{\decl}[2]{\m{decl} \; #1 \; [#2]}
\newcommand{\val}[2]{\m{val} \; #1 : \m{#2}}
\newcommand{\declconst}[3]{\const{#1}{#2} \; \m{of} \; #3}
\newcommand{\declfun}[4]{\fun{#1}{(#2)}{#3} \; \m{of} \; #4}
\newcommand{\eval}[2]{\Psi \; ; \; #1 \rightarrow #2}
\newcommand{\constraint}[1]{\m{constraint} \; #1}
\newcommand{\fact}[3]{#1[@#2](#3)}
\newcommand{\mif}[3]{\m{if} \; #1 \; \m{then} \; #2 \; \m{else} \; #3 \; \m{end}}
\newcommand{\mrule}[4]{\Psi ; #1 ; #2 \vdash_{#4} #3 \; \m{rule}}
\newcommand{\mrulebody}[4]{\Psi ; #1 ; #2 ; #3 \vdash #4 \; \m{body}}
\newcommand{\mrulehead}[4]{\Psi ; #1 ; #2 \vdash_{#4} #3 \; \m{head}}
\newcommand{\mrulestart}[1]{\m{rule} \; \Psi \; / \; #1}
\newcommand{\comp}[0]{\m{comp} \; }
\newcommand{\aggregate}[4]{[\m{#1} ; #2 ; #3 \Rightarrow #4]}
\newcommand{\aggregatetype}[3]{[\m{#1}] \; / \; #2 \rightsquigarrow #3}
\newcommand{\aggregatestart}[2]{[\m{#1}] \hookrightarrow #2}
\newcommand{\aggregateop}[4]{[\m{#1}] \; #2 / #3 \Rightarrow #4}
\newcommand{\changes}[7]{#1 ; #2 ; #3 ; #4 \Rightarrow #5 ; #6 ; #7}
\newcommand{\changesb}[7]{#1 ; #2 ; #3 ; #4 \Rightarrow #5 ; #6 ; #7}
\newcommand{\apply}[6]{\m{apply} \; #1 ; #2 ; #3 \rightarrow #4 ; #5 ; #6}
\newcommand{\applyb}[6]{\m{apply} \; #1 ; #2 ; #3 \rightarrow #4 ; #5 ; #6}
\newcommand{\derive}[8]{\m{derive} \; \Psi ; #1 ; #2 ; #3 ; #4 ; #5 \rightarrow #6 ; #7 ; #8}
\newcommand{\deriveb}[9]{\m{derive} \; #1 ; #2 ; #3 ; #4 ; #5 ; #6 \rightarrow #7 ; #8 ; #9}
\newcommand{\match}[5]{\m{match} \; #1 ; #2 ; #3 \rightarrow #4 ; #5}
\newcommand{\equal}[2]{#1 = #2}
\newcommand{\at}[2]{#1 \; @ \; #2}
\newcommand{\compr}[1]{\m{def} \; #1}
\newcommand{\comprehension}[1]{\comp #1}
\newcommand{\comprrec}[1]{\m{comp2} \; #1}

\maketitle

\section{Static Semantics}

\subsection{Types}

\[
\infer[\m{addr}]
{\typ{addr}}{}
\tab
\infer[\m{int}]
{\typ{int}}
{}
\tab
\infer[\m{float}]
{\typ{float}}
{}
\tab
\infer[\m{bool}]
{\typ{bool}}{}
\tab
\infer[\m{string}]
{\typ{string}}{}
\]

\[
\infer[\m{list}]
{\typ{list \; \tau}}{\typ{\tau}}
\tab
\infer[\m{struct}]
{\typ{struct \; T}}{\typelist{T}}
\]

\[
\infer[\m{type \; list \; end}]
{\typelist{\cdot}}
\tab
\infer[\m{type \; list}]
{\typelist{\tau ; T}}{\typelist{T}}
\]

\subsection{Expressions}

\[
\infer[\m{addr \; literal}]
{\expr{\Gamma}{\m{addr}(N)}{\m{addr}}}
{}
\]

\[
\infer[\m{int \; literal}]
{\expr{\Gamma}{\m{int}(N)}{int}}{\m{N \; is \; a \; literal}}
\tab
\infer[\m{float \; literal}]
{\expr{\Gamma}{\m{float}(F)}{float}}{\m{F \; is \; a \; float \; literal}}
\]

\[
\infer[\m{string \; literal}]
{\expr{\Gamma}{\m{string}(S)}{string}}{\m{S \; is \; a \; string \; literal}}
\tab
\infer[\m{var}]
{\expr{\Gamma, \eexpr{X}{\tau}}{X}{\tau}}{}
\]

\[
\infer[\m{nil}]
{\expr{\Gamma}{[]}{\m{list \; \tau}}}{\m{\tau}}
\tab
\infer[\m{cons}]
{\expr{\Gamma}{[e_1 \; | \; e_2]}{\m{list \; \tau}}}
{\expr{\Gamma}{e_1}{\m{\tau}} &
   \expr{\Gamma}{e_2}{\m{list \; \tau}}}
\]

\[
\infer[\m{make \; struct}]
{\expr{\Gamma}{\m{struct} \; e_1; ...; e_n}{\m{struct \; \tau_1; ...; \tau_n}}}
{\expr{\Gamma}{e_1}{\tau_1} & ... & \expr{\Gamma}{e_n}{\tau_n}}
\tab
\infer[\m{get \; struct}]
{\expr{\Gamma}{\#i(e)}{\tau_i}}
{\expr{\Gamma}{e}{\m{struct \; \tau_1; ...; \tau_n}}}
\]

\[
\infer[\m{math \; int}]
{\expr{\Gamma}{e_1 \; \m{op} \; e_2}{\m{int}}}
{\expr{\Gamma}{e_1}{\m{int}} & \expr{\Gamma}{e_2}{\m{int}}}
\tab
\infer[\m{math \; float}]
{\expr{\Gamma}{e_1 \; \m{op} \; e_2}{\m{float}}}
{\expr{\Gamma}{e_1}{\m{float}} & \expr{\Gamma}{e_2}{\m{float}}}
\]

\[
\infer[\m{math \; cast1}]
{\expr{\Gamma}{e_1 \; \m{op} \; e_2}{\m{float}}}
{\expr{\Gamma}{e_1}{\m{int}} & \expr{\Gamma}{e_2}{\m{float}}}
\tab
\infer[\m{math \; cast2}]
{\expr{\Gamma}{e_1 \; \m{op} \; e_2}{\m{float}}}
{\expr{\Gamma}{e_1}{\m{float}} & \expr{\Gamma}{e_2}{\m{int}}}
\]

\[
\infer[\m{if}]
{\expr{\Gamma}{\mif{c}{e_1}{e_2}}{\tau}}
{\expr{\Gamma}{c}{\m{bool}} &
   \expr{\Gamma}{e_1}{\m{\tau}} &
      \expr{\Gamma}{e_2}{\m{\tau}}}
\]

\[
\infer[\m{cmp} \; \m{int}]
{\expr{\Gamma}{e_1 \; \m{cmp} \; e_2}{\m{bool}}}
{\expr{\Gamma}{e_1}{\m{int}} &
   \expr{\Gamma}{e_2}{\m{int}}}
\tab
\infer[\m{cmp} \; \m{float}]
{\expr{\Gamma}{e_1 \; \m{cmp} \; e_2}{\m{bool}}}
{\expr{\Gamma}{e_1}{\m{float}} &
   \expr{\Gamma}{e_2}{\m{float}}}
\]

\[
\infer[\m{cmp} \; \m{bool}]
{\expr{\Gamma}{e_1 \; \m{cmp} \; e_2}{\m{bool}}}
{\expr{\Gamma}{e_1}{\m{bool}} &
   \expr{\Gamma}{e_2}{\m{bool}}}
\tab
\infer[\m{cmp} \; \m{string}]
{\expr{\Gamma}{e_1 \; \m{cmp} \; e_2}{\m{bool}}}
{\expr{\Gamma}{e_1}{\m{string}} &
   \expr{\Gamma}{e_2}{\m{string}}}
\]

\[
\infer[\m{cmp} \; \m{addr}]
{\expr{\Gamma}{e_1 \; \m{cmp} \; e_2}{\m{bool}}}
{\expr{\Gamma}{e_1}{\m{addr}} &
   \expr{\Gamma}{e_2}{\m{addr}}}
\]

\[
\infer[\m{or}]
{\expr{\Gamma}{e_1 \; \m{or} \; e_2}{\m{bool}}}
{\expr{\Gamma}{e_1}{\m{bool}} & \expr{\Gamma}{e_2}{\m{bool}}}
\]

\[
\infer[\m{let}]
{\expr{\Gamma}{\elet{X}{e_1}{e_2}}{\tau}}
{\expr{\Gamma}{e_1}{\tau_1} &
   \expr{\Gamma, \eexpr{X}{\tau_1}}{e_2}{\tau}}
\]

\[
\infer[\m{const}]
{\aexp{\Psi, \eexpr{\const{name}{v}}{\tau}}{\Gamma}{\getconst{name}}{\tau}}
{}
\]

\[
\infer[\m{external}]
{\expr{\Gamma}{\callexternal{name}{[e_1, ..., e_n]}}{\tau}}
{\expr{\Gamma}{e_1}{\tau_1} & ... & \expr{\Gamma}{e_n}{\tau_n} &
   \eexpr{\external{name}{[arg_1, ..., arg_n]}}{(\tau_1, ..., \tau_n)\overrightarrow{\tau}} \in \Psi
}
\]

\[
\infer[\m{fun}]
{\expr{\Gamma}{\callfun{name}{[e_1, ..., e_n]}}{\tau}}
{\expr{\Gamma}{e_1}{\tau_1} & ... & \expr{\Gamma}{e_n}{\tau_n} &
   \eexpr{\fun{name}{[arg_1, ..., arg_n]}{e}}{(\tau_1, ..., \tau_n)\overrightarrow{\tau}} \in \Psi
}
\]

\[
\infer[\m{world}]
{\expr{\Gamma}{\m{world}}{\m{int}}}
{}
\tab
\infer[\m{arg}]
{\expr{\Gamma}{\m{arg}(N)}{\m{string}}}
{}
\]

\subsection{Declarations}

\[
\infer[\m{decl}]
{\decl{name}{\m{addr}, \tau_1, ..., \tau_n}}
{\typ{addr} & \typ{\tau_1} & ... & \typ{\tau_n}}
\tab
\infer[\m{\bang decl}]
{\bang\decl{name}{\m{addr}, \tau_1, ..., \tau_n}}
{\typ{addr} & \typ{\tau_1} & ... & \typ{\tau_n}}
\]

\[
\infer[\m{const}]
{\declconst{name}{v}{\tau}}
{\expr{\Gamma}{e}{\tau} & \eval{e}{v} & \val{v}{\tau}}
\]

\[
\infer[\m{fun}]
{\declfun{name}{arg_1 : \tau_1, ..., arg_n : \tau_n}{e}{\tau}}
{\expr{arg_1 : \tau_1, ..., arg_n : \tau_n}{e}{\tau}}
\]

\subsection{Rules}

\[
\infer[\m{rule \; start}]
{\mrulestart{\forall H : \m{addr}. A}}
{\mrule{\cdot}{H}{A}{1}}
\]

\[
\infer[\m{rule \; add \; var}]
{\mrule{\Gamma}{H}{\forall X : \tau. A}{N}}
{\mrule{\Gamma, X : \tau}{H}{A}{N}}
\]

\[
\infer[\m{rule \; body \; head}]
{\mrule{\Gamma}{H}{A \lolli B}{N}}
{\mrulebody{\Gamma, H : \m{addr}}{H}{\Gamma}{A} & \mrulehead{\Gamma, H:\m{addr}}{H}{B}{N}}
\]

\[
\infer[\m{rule \; body \; tensor}]
{\mrulebody{\Gamma}{H}{\Gamma', \Gamma''}{A \otimes B}}
{\mrulebody{\Gamma}{H}{\Gamma''}{B} &
   \mrulebody{\Gamma}{H}{\Gamma'}{A} & }
\]

\[
\infer[\m{rule \; body \; 1}]
{\mrulebody{\Gamma}{H}{\cdot}{1}}
{}
\]

\[
\infer[\m{rule \; body \; exists}]
{\mrulebody{\Gamma}{H}{\Gamma'}{\exists X : \tau. A}}
{\mrulebody{\Gamma, X : \tau}{H}{\Gamma', X : \tau}{A}}
\]

\[
\infer[\m{rule \; body \; fact}]
{\mrulebody{\Gamma, H_{fact} : \m{addr}}{H}{X_1 : \tau_1, ..., X_n : \tau_n}{\fact{name}{H_{fact}}{X_1, ..., X_n}}}
{\decl{name}{\m{addr}, \tau_1, ..., \tau_n} \in \Psi}
\]


\[
\infer[\m{rule \; body \; \bang fact}]
{\mrulebody{\Gamma, H_{fact} : \m{addr}}{H}{X_1 : \tau_1, ..., X_n : \tau_n}{\bang\fact{name}{H_{fact}}{X_1, ..., X_n}}}
{\bang\decl{name}{\m{addr}, \tau_1, ..., \tau_n} \in \Psi}
\]

\[
\infer[\m{rule}]
{\mrulebody{\Gamma}{H}{\cdot}{\bang (\constraint{e})}}
{\expr{\Gamma}{e}{\m{bool}}}
\]

\[
\infer[\m{rule \; head \; tensor}]
{\mrulehead{\Gamma}{H}{A \otimes B}{N}}
{\mrulehead{\Gamma}{H}{A}{N} & \mrulehead{\Gamma}{H}{B}{N}}
\]

\[
\infer[\m{rule \; head \; 1}]
{\mrulehead{\Gamma}{H}{\m{1}}{N}}
{}
\]

\[
\infer[\m{rule \; head \; fact}]
{\mrulehead{\Gamma}{H}{\fact{name}{e}{e_1, ..., e_n}}{N}}
{\expr{\Gamma}{e}{\m{addr}} & \expr{\Gamma}{e_1}{\tau_1} & ... & \expr{\Gamma}{e_n}{\tau_n} &
   \decl{name}{\m{addr}, \tau_1, ..., \tau_n} \in \Psi}
\]

\[
\infer[\m{rule \; head \; \bang fact}]
{\mrulehead{\Gamma}{H}{\bang\fact{name}{e}{e_1, ..., e_n}}{N}}
{\expr{\Gamma}{e}{\m{addr}} & \expr{\Gamma}{e_1}{\tau_1} & ... & \expr{\Gamma}{e_n}{\tau_n} &
   \bang\decl{name}{\m{addr}, \tau_1, ..., \tau_n} \in \Psi}
\]

\[
\infer[\m{rule \; head \; exists}]
{\mrulehead{\Gamma}{H}{\exists X : \m{addr}. A}{N}}
{\mrulehead{\Gamma, X : \m{addr}}{H}{A}{N}}
\]

\[
\infer[\m{rule \; head \; comprehension}]
{\mrulehead{\Gamma}{H}{\comprehension{A}}{1}}
{\mrule{\Gamma}{H}{A}{2}}
\]

\[
\infer[\m{rule \; head \; aggregate}]
{\mrulehead{\Gamma}{H}{\aggregate{Op}{X}{A}{B}}{1}}
{\aggregatetype{Op}{\tau_1}{\tau_2} & \mrule{\Gamma, X : \tau_1}{H}{A \lolli 1}{2} & \mrulehead{\Gamma, X : \tau_2}{H}{B}{2}}
\]

\subsection{Aggregate Types}

\[
\infer[\m{agg \; count}]
{\aggregatetype{count}{\m{int}}{\m{int}}}
{}
\tab
\infer[\m{agg \; collect \; int}]
{\aggregatetype{collect \; int}{\m{int}}{\m{list \; int}}}
{}
\]

\[
\infer[\m{agg \; int \; sum}]
{\aggregatetype{sum \; int}{\m{int}}{\m{int}}}
{}
\]

\[
\infer[\m{agg \; int \; max}]
{\aggregatetype{max \; int}{\m{int}}{\m{int}}}
{}
\tab
\infer[\m{agg \; int \; min}]
{\aggregatetype{min \; int}{\m{int}}{\m{int}}}
{}
\]

\section{Dynamic Semantics}

\subsection{Expression Values}

\[
\infer[\m{int}]
{\val{\m{int}(N)}{int}}
{}
\tab
\infer[\m{bool}]
{\val{\m{bool}(B)}{bool}}
{}
\tab
\infer[\m{float}]
{\val{\m{float}(F)}{float}}
{}
\]

\[
\infer[\m{string}]
{\val{\m{string}(S)}{string}}
{}
\tab
\infer[\m{addr}]
{\val{\m{addr}(A)}{addr}}
{}
\]

\[
\infer[\m{nil}]
{\val{[]}{list \; \tau}}
{}
\tab
\infer[\m{cons}]
{\val{x :: ls}{list \; \tau}}
{\val{x}{\tau} & \val{ls}{list \; \tau}}
\]

\[
\infer[\m{struct}]
{\val{:(v_1, ..., v_n)}{\m{struct \; \tau_1; ...; \tau_n}}}
{\val{v_1}{\tau_1} & ... & \val{v_n}{\tau_n}}
\]

\subsection{Expression Evaluation}

\[
\infer[\m{int}]
{\eval{\m{int}(N)}{\m{int}(N)}}
{}
\tab
\infer[\m{float}]
{\eval{\m{float}(F)}{\m{float}(F)}}
{}
\tab
\infer[\m{addr}]
{\eval{\m{addr}(A)}{\m{addr}(A)}}
{}
\]

\[
\infer[\m{bool}]
{\eval{\m{bool}(B)}{\m{bool}(B)}}
{}
\]

\[
\infer[\m{string}]
{\eval{\m{string}(S)}{\m{string}(S)}}
{}
\tab
\infer[\m{list\; nil}]
{\eval{[]}{[]}}
{}
\tab
\infer[\m{list}]
{\eval{L}{L}}
{}
\]

\[
\infer[\m{cons}]
{\eval{[e_1 | e_2]}{v_1 :: v_2}}
{\eval{e_1}{v_1} & \eval{e_2}{v_2}}
\]

\[
\infer[\m{make \; struct}]
{\eval{\m{struct \; e_1; ...; e_n}}{:(v_1; ...; v_n)}}
{\eval{e_1}{v_1} & ... & \eval{e_n}{v_n}}
\]

\[
\infer[\m{get \; struct}]
{\eval{\m{\#i(e)}}{v_i}}
{\eval{e}{:(v_1; ...; v_i ; ... ; v_n)}}
\]

\[
\infer[\m{math \; int}]
{\eval{e_1 \; \m{op} \; e_2}{\m{int}(V)}}
{\expr{\cdot}{e_1}{\m{int}} & \expr{\cdot}{e_2}{\m{int}} & \eval{e_1}{\m{int}(A)} & \eval{e_2}{\m{int}(B)} &
   V = A \; \m{op} \; B}
\]

\[
\infer[\m{math \; float}]
{\eval{e_1 \; \m{op} \; e_2}{\m{float}(V)}}
{\expr{\cdot}{e_1}{\m{float}} & \expr{\cdot}{e_2}{\m{float}} &
   \eval{e_1}{\m{float}(A)} & \eval{e_2}{\m{float}(B)} &
   V = A \; \m{op} \; B}
\]

\[
\infer[\m{math \; cast1}]
{\eval{e_1 \; \m{op} \; e_2}{\m{float}(V)}}
{\expr{\cdot}{e_1}{\m{int}} & \expr{\cdot}{e_2}{\m{float}} &
   \eval{e_1}{\m{int}(A)} & \eval{e_2}{\m{float}(B)} &
   V = A \; \m{op} \; B}
\]

\[
\infer[\m{math \; cast2}]
{\eval{e_1 \; \m{op} \; e_2}{\m{float}(V)}}
{\expr{\cdot}{e_1}{\m{float}} & \expr{\cdot}{e_2}{\m{int}} &
   \eval{e_1}{\m{float}(A)} & \eval{e_2}{\m{int}(B)} &
   V = A \; \m{op} \; B}
\]

\[
\infer[\m{if \; true}]
{\eval{\mif{c}{e_1}{e_2}}{v_1}}
{\eval{c}{\m{bool}(true)} &
   \eval{e_1}{v_1}}
\tab
\infer[\m{if \; false}]
{\eval{\mif{c}{e_1}{e_2}}{v_2}}
{\eval{c}{\m{bool}(false)} &
   \eval{e_2}{v_2}}
\]

\[
\infer[\m{cmp \; int}]
{\eval{e_1 \; \m{cmp} \; e_2}{\m{bool}(V)}}
{\eval{e_1}{\m{int}(A)} &
   \eval{e_2}{\m{int}(B)} &
   V = A \; \m{cmp} \; B
}
\]

\[
\infer[\m{cmp \; float}]
{\eval{e_1 \; \m{cmp} \; e_2}{\m{bool}(V)}}
{\eval{e_1}{\m{float}(A)} &
   \eval{e_2}{\m{float}(B)} &
   V = A \; \m{cmp} \; B
}
\]

\[
\infer[\m{cmp \; bool}]
{\eval{e_1 \; \m{cmp} \; e_2}{\m{bool}(V)}}
{\eval{e_1}{\m{bool}(A)} &
   \eval{e_2}{\m{bool}(B)} &
   V = A \; \m{cmp} \; B
}
\]

\[
\infer[\m{cmp \; string}]
{\eval{e_1 \; \m{cmp} \; e_2}{\m{bool}(V)}}
{\eval{e_1}{\m{string}(A)} &
   \eval{e_2}{\m{string}(B)} &
   V = A \; \m{cmp} \; B
}
\]

\[
\infer[\m{cmp \; addr}]
{\eval{e_1 \; \m{cmp} \; e_2}{\m{bool}(V)}}
{\eval{e_1}{\m{addr}(A)} &
   \eval{e_2}{\m{addr}(B)} &
   V = A \; \m{cmp} \; B
}
\]

\[
\infer[\m{or}]
{\eval{e_1 \; \m{or} \; e_2}{\m{bool}(V)}}
{\eval{e_1}{\m{bool}(A)} &
   \eval{e_2}{\m{bool}(B)} &
   V = A \; \m{or} \; B
}
\]

\[
\infer[\m{let}]
{\eval{\elet{X}{e_1}{e_2}}{v}}
{\eval{e_1}{v_1} &
   \eval{[v_1/x]e_2}{v}}
\]

\[
\infer[\m{const}]
{\eval{\getconst{name}}{v}}
{\const{name}{v} : \tau \in \Psi}
\]

\[
\infer[\m{external}]
{\eval{\callexternal{name}{[e_1, ..., e_n]}}{v}}
{\begin{array}{ccc}
   \eval{e_1}{v_1} & ... & \eval{e_n}{v_n} \\
   \multicolumn{3}{c}{\eexpr{\external{name}{[arg_1, ..., arg_n]}}{(\tau_1, ..., \tau_n)\overrightarrow{\tau}} \in \Psi} \\
   \multicolumn{3}{c}{v = \callexternal{name}{[v_1, ..., v_n]}} \\
 \end{array}
}
\]

\[
\infer[\m{fun}]
{\eval{\callfun{name}{[e_1, ..., e_n]}}{v}}
{\begin{array}{ccc}
   \eval{e_1}{v_1} & ... & \eval{e_n}{v_n} \\
   \multicolumn{3}{c}{\eexpr{\fun{name}{[arg_1, ..., arg_n]}{e}}{(\tau_1, ..., \tau_n)\overrightarrow{\tau}} \in \Psi} \\
   \multicolumn{3}{c}{\eval{[v_1/\m{arg}_1]...[v_n/\m{arg}_n]e}{v}} \\
   \multicolumn{3}{c}{v = \callfun{name}{v_1, ..., v_n}} \\
 \end{array}
}
\]

\[
\infer[\m{world}]
{\eval{\m{world}}{\m{int}(N)}}
{}
\tab
\infer[\m{arg}]
{\eval{\m{arg}(N)}{\m{string}(S)}}
{}
\]

\subsection{Aggregates}

\[
\infer[\m{init \; count}]
{\aggregatestart{count}{\m{int}(0)}}
{}
\]

\[
\infer[\m{init \; collect \; int}]
{\aggregatestart{collect \; int}{[]}}
{}
\]

\[
\infer[\m{init \; sum}]
{\aggregatestart{sum}{\m{int}(0)}}
{}
\tab
\infer[\m{init \; max}]
{\aggregatestart{max}{\m{int}(-\infty)}}
{}
\tab
\infer[\m{init \; min}]
{\aggregatestart{min}{\m{int}(+\infty)}}
{}
\]

\[
\infer[\m{op \; sum}]
{\aggregateop{sum}{A}{B}{A + B}}
{}
\]

\[
\infer[\m{op \; count}]
{\aggregateop{count}{A}{B}{A + 1}}
{}
\]

\[
\infer[\m{op \; collect \; int}]
{\aggregateop{collect \; int}{A}{B}{[B | A]}}
{}
\]

\[
\infer[\m{op \; min}]
{\aggregateop{min}{A}{B}{\mif{A \leq B}{A}{B}}}
{}
\]

\[
\infer[\m{op \; max}]
{\aggregateop{max}{A}{B}{\mif{A \leq B}{B}{A}}}
{}
\]


\subsection{Global Semantics}

Meaning of variables:

\begin{description}
\item[$\Psi$]: Program state: constants, functions, external functions and declarations.
\item[$\Theta$]: Rules with priority.
\item[$\Phi$]: Rules without priority.
\item[$\Gamma$]: Persistent fact context.
\item[$\Delta$]: Linear fact context.
\end{description}

\[
\infer[\m{rule \; app}]
{\changes{\Psi}{\Theta, \Phi, R}{\Gamma}{\Delta}{\Gamma'}{\Delta'}{\Xi}}
{\apply{\Psi}{\Gamma}{\Delta, [R]}{\Gamma'}{\Delta'}{\Xi}}
\]

\[
\infer[\m{\lolli L}]
{\apply{\Psi}{\Gamma}{\Delta_1, \Delta_2, [A \lolli B]}{\Gamma'}{\Delta'}{\Delta_1, \Xi}}
{\match{\Psi}{\Gamma}{\Delta_1}{[A]}{\Delta_1} &
   \derive{\Gamma}{\Delta_2}{[B]}{\cdot}{\cdot}{\Gamma'}{\Delta'}{\Xi}
}
\]

\[
\infer[\m{\forall L}]
{\apply{\Psi}{\Gamma}{\Delta, [\forall X : \tau. A]}{\Gamma'}{\Delta'}{\Xi}}
{\val{M}{\tau} & \apply{\Psi}{\Gamma}{\Delta, [A\{M/X\}]}{\Gamma'}{\Delta'}{\Xi}}
\]

\subsubsection{Match}

\[
\infer[\m{match \; exists}]
{\match{\Psi}{\Gamma}{\Delta}{[\exists X. A]}{\Xi}}
{\match{\Psi}{\Gamma}{\Delta}{[[M/X]A]}{\Xi}}
\]

\[
\infer[\m{match \; one}]
{\match{\Psi}{\Gamma}{\cdot}{[1]}{\cdot}}
{}
\]

\[
\infer[\m{match \; split}]
{\match{\Psi}{\Gamma}{\Delta, \Delta'}{[A \otimes B]}{\Xi_1, \Xi_2}}
{\match{\Psi}{\Gamma}{\Delta}{[A]}{\Xi_1} &
   \match{\Psi}{\Gamma}{\Delta'}{[B]}{\Xi_2}
}
\]

\[
\infer[\m{match \; end \; constraint}]
{\match{\Psi}{\Gamma}{\cdot}{[\bang\constraint{e}]}{\cdot}}
{\eval{e}{\m{bool}(\m{true})}}
\]

\[
\infer[\m{match \; end \; linear}]
{\match{\Psi}{\Gamma}{\fact{name}{v_1}{v_2, ..., v_n}}{[\fact{name}{v_1}{v_2, ..., v_n}]}{\fact{name}{v_1}{v_2, ..., v_n}}}
{\equal{v_1}{v_1'} & ... & \equal{v_n}{v_n'}}
\]
%% do I need a val here?

\[
\infer[\m{match \; end \; persistent}]
{\match{\Psi}{\Gamma, \bang\fact{name}{v_1}{v_2, ..., v_n}}{\cdot}{[\bang\fact{name}{v_1'}{v_2', ..., v_n'}]}{\cdot}}
{\equal{v_1}{v_1'} & ... & \equal{v_n}{v_n'}}
\]

\[
\infer[\m{equal \; int}]
{\equal{\m{int}(N)}{\m{int}(N)}}
{}
\tab
\infer[\m{equal \; float}]
{\equal{\m{float}(F)}{\m{float}(F)}}
{}
\tab
\infer[\m{equal \; addr}]
{\equal{\m{addr}(A)}{\m{addr}(A)}}
{}
\]

\[
\infer[\m{equal \; string}]
{\equal{\m{string}(S)}{\m{string}(S)}}
{}
\tab
\infer[\m{equal \; bool}]
{\equal{\m{bool}(B)}{\m{bool}(B)}}
{}
\]

\[
\infer[\m{equal \; nil}]
{\equal{[]}{[]}}
{}
\tab
\infer[\m{equal \; cons}]
{\equal{x :: ls}{x' :: ls'}}
{\equal{x}{x'} & \equal{ls}{ls'}}
\]

\subsection{Derive}

\[
\infer[\m{derive \; blur}]
{\derive{\Gamma}{\Delta}{[A]}{\Delta_1}{\Gamma_1}{\Gamma'}{\Delta'}{\Xi}}
{\derive{\Gamma}{\Delta}{A}{\Delta_1}{\Gamma_1}{\Gamma'}{\Delta'}{\Xi}}
\]

\[
\infer[\m{derive \; \otimes}]
{\derive{\Gamma}{\Delta}{A \otimes B, \Omega}{\Delta_1}{\Gamma_1}{\Gamma'}{\Delta'}{\Xi}}
{\derive{\Gamma}{\Delta}{A, B, \Omega}{\Delta_1}{\Gamma_1}{\Gamma'}{\Delta'}{\Xi}}
\]

\[
\infer[\m{derive \; exists}]
{\derive{\Gamma}{\Delta}{\exists X : \m{addr}. A, \Omega}{\Delta_1}{\Gamma_1}{\Gamma'}{\Delta'}{\Xi}}
{\derive{\Gamma}{\Delta}{[x/X]A, \Omega}{\Delta_1}{\Gamma_1}{\Gamma'}{\Delta'}{\Xi} &
   x = \m{new} \; \m{addr}(A)}
\]

\[
\infer[\m{derive \; fact}]
{\derive{\Gamma}{\Delta}{\fact{name}{e_1}{e_2, ..., e_n}, \Omega}{\Delta_1}{\Gamma_1}{\Gamma'}{\Delta'}{\Xi}}
{\begin{array}{ccc}
   \eval{e_1}{v_1} & ... & \eval{e_n}{v_n} \\
   \multicolumn{3}{c}{\derive{\Gamma}{\Delta}{\Omega}{\Delta_1, \fact{name}{v_1}{v_2, ..., v_n}}{\Gamma_1}{\Gamma'}{\Delta'}{\Xi}} \\
   \end{array}
}
\]

\[
\infer[\m{derive \; \bang fact}]
{\derive{\Gamma}{\Delta}{\bang\fact{name}{e_1}{e_2, ..., e_n}, \Omega}{\Delta_1}{\Gamma_1}{\Gamma'}{\Delta'}{\Xi}}
{\begin{array}{ccc}
   \eval{e_1}{v_1} & ... & \eval{e_n}{v_n} \\
   \multicolumn{3}{c}{\derive{\Gamma}{\Delta}{\Omega}{\Delta_1}{\Gamma_1, \bang\fact{name}{v_1}{v_2, ..., v_n}}{\Gamma'}{\Delta'}{\Xi}} \\
   \end{array}
}
\]

\[
\infer[\m{derive \; 1}]
{\derive{\Gamma}{\Delta}{1, \Omega}{\Delta_1}{\Gamma_1}{\Gamma'}{\Delta'}{\Xi}}
{\derive{\Gamma}{\Delta}{\Omega}{\Delta_1}{\Gamma_1}{\Gamma'}{\Delta'}{\Xi}}
\]

\[
\infer[\m{derive \; comprehension}]
{\derive{\Gamma}{\Delta}{\comprehension{A}, \Omega}{\Delta_1}{\Gamma_1}{\Gamma'}{\Delta'}{\Xi}}
{\derive{\Gamma}{\Delta}{1 \with (A \otimes \comprehension{A}), \Omega}{\Delta_1}{\Gamma_1}{\Gamma'}{\Delta'}{\Xi}}
\]


\[
\infer[\m{derive \; \with \; left}]
{\derive{\Gamma}{\Delta}{A \with B, \Omega}{\Delta_1}{\Gamma_1}{\Gamma'}{\Delta'}{\Xi}}
{\derive{\Gamma}{\Delta}{A, \Omega}{\Delta_1}{\Gamma_1}{\Gamma'}{\Delta'}{\Xi}}
\]

\[
\infer[\m{derive \; \with \; right}]
{\derive{\Gamma}{\Delta}{A \with B, \Omega}{\Delta_1}{\Gamma_1}{\Gamma'}{\Delta'}{\Xi}}
{\derive{\Gamma}{\Delta}{B, \Omega}{\Delta_1}{\Gamma_1}{\Gamma'}{\Delta'}{\Xi}}
\]

\newcommand{\aggdef}[4]{\m{agg} \; \m{#1} \; #2 \; #3 \; #4}

\[
\infer[\m{derive \; aggregate}]
{\derive{\Gamma}{\Delta}{\aggregate{Op}{X}{A}{B}, \Omega}{\Delta_1}{\Gamma_1}{\Gamma'}{\Delta'}{\Xi}}
{\aggregatestart{Op}{V} & \derive{\Gamma}{\Delta}{\aggdef{Op}{V}{(x. A(x))}{(y. B(y))}, \Omega}{\Delta_1}{\Gamma_1}{\Gamma'}{\Delta'}{\Xi}}
\]

\[
\infer[\m{derive \; aggregate \; unfold}]
{\derive{\Gamma}{\Delta}{\aggdef{Op}{V'}{(x. A(x))}{(y. B(y))}, \Omega}{\Delta_1}{\Gamma_1}{\Gamma'}{\Delta'}{\Xi}}
{\derive{\Gamma}{\Delta}{B(V) \with (\forall X'. A(X') \lolli \aggdef{Op}{E}{(x. A(x))}{(y. B(y))}), \Omega}{\Delta_1}{\Gamma_1}{\Gamma'}{\Delta'}{\Xi} & \aggregateop{Op}{V'}{X'}{E}
}
\]

\[
\infer[\m{derive \; forall}]
{\derive{\Gamma}{\Delta}{\forall X : \tau. A, \Omega}{\Delta_1}{\Gamma_1}{\Gamma'}{\Delta'}{\Xi}}
{\derive{\Gamma}{\Delta}{[M/X]A, \Omega}{\Delta_1}{\Gamma_1}{\Gamma'}{\Delta'}{\Xi} & \val{M}{\tau}}
\]

\[
\infer[\m{derive \; lolli}]
{\derive{\Gamma}{\Delta}{A \lolli B, \Omega}{\Delta_1}{\Gamma_1}{\Gamma'}{\Delta'}{\Xi, \Xi'}}
{\match{\Psi}{\Gamma}{\Delta}{A}{\Xi'} &
   \derive{\Gamma}{\Delta - \Xi'}{B, \Omega}{\Delta_1}{\Gamma_1}{\Gamma'}{\Delta'}{\Xi}
}
\]

\[
\infer[\m{derive \; end}]
{\derive{\Gamma}{\Delta}{\cdot}{\Delta_1}{\Gamma_1}{\Gamma_1}{\Delta_1}{\cdot}}
{}
\]

\subsection{Local Semantics}

\[
\infer[\m{rule \; app}]
{\at{\changes{\Psi}{\Theta, R}{\Gamma}{\Delta}{\Gamma, \Gamma'}{\Delta', N}}{\pi}}
{\at{\apply{\Psi}{\Gamma}{\Delta, [R]}{\Gamma'}{\Delta'}{\Xi}}{\pi} & N = \Delta - \Xi}
\]

\[
\infer[\m{\lolli L}]
{\at{\apply{\Psi}{\Gamma}{\Delta_1, \Delta_2, [A \lolli B]}{\Gamma'}{\Delta'}{\Xi}}{\pi}}
{\at{\match{\Psi}{\Gamma}{\Delta_1}{[A]}{\Xi}}{\pi} &
   \derive{\Gamma}{\Delta_2}{[B]}{\cdot}{\cdot}{\Gamma'}{\Delta'}{\Xi}}
\]

\[
\infer[\m{\forall L}]
{\at{\apply{\Psi}{\Gamma}{\Delta, [\forall X : \tau. A]}{\Gamma'}{\Delta'}{\Xi}}{\pi}}
{\val{M}{\tau} & \at{\apply{\Psi}{\Gamma}{\Delta, [A\{M/X\}]}{\Gamma'}{\Delta'}{\Xi}}{\pi}}
\]

\subsubsection{Match}

\[
\infer[\m{match \; exists}]
{\at{\match{\Psi}{\Gamma}{\Delta}{[\exists X. A]}{\Xi}}{\pi}}
{\at{\match{\Psi}{\Gamma}{\Delta}{[[M/X]A]}{\Xi}}{\pi}}
\]

\[
\infer[\m{match \; one}]
{\at{\match{\Psi}{\Gamma}{\cdot}{[1]}{\cdot}}{\pi}}
{}
\]

\[
\infer[\m{match \; split}]
{\at{\match{\Psi}{\Gamma}{\Delta, \Delta'}{[A \otimes B]}{\Xi_1, \Xi_2}}{\pi}}
{\at{\match{\Psi}{\Gamma}{\Delta}{[A]}{\Xi_1}}{\pi} &
   \at{\match{\Psi}{\Gamma}{\Delta'}{[B]}{\Xi_2}}{\pi}}
\]

\[
\infer[\m{match \; end \; constraint}]
{\at{\match{\Psi}{\Gamma}{\cdot}{[\bang\constraint{e}]}{\cdot}}{\pi}}
{\eval{e}{\m{bool}(\m{true})}}
\]

\[
\infer[\m{match \; end \; linear}]
{\at{\match{\Psi}{\Gamma}{\fact{name}{v_1}{v_2, ..., v_n}}{[\fact{name}{v_1}{v_2, ..., v_n}]}{\fact{name}{v_1}{v_2, ..., v_n}}}{\pi}}
{\equal{v_1}{v_1'} & ... & \equal{v_n}{v_n'} & v_1 = \m{addr}(\pi)}
\]
%% do I need a val here?

\[
\infer[\m{match \; end \; persistent}]
{\at{\match{\Psi}{\Gamma, \bang\fact{name}{v_1}{v_2, ..., v_n}}{\cdot}{[\bang\fact{name}{v_1'}{v_2', ..., v_n'}]}{\cdot}}{\pi}}
{\equal{v_1}{v_1'} & ... & \equal{v_n}{v_n'} & v_1 = \m{addr}(\pi)}
\]

\begin{comment}
\section{Comprehensions}

Comprehensions have the following syntax:

\begin{verbatim}
body -o {Var-List | CompBody | CompHead}.
\end{verbatim}

We can distinguish between two types of comprehensions:

\subsection{Persistent Only Comprehensions}

These comprehensions only use persistent facts in the body. The head may have linear facts.
Since we only use persistent facts, we are unable to check if we are done with the comprehension just by being unable to do further matchings.
Thus, the only way to check for a stop condition is to verify repeated variables in \texttt{CompBody}.

\begin{verbatim}
body(A) -o {X1, X2, X3 | !a(A, X1), !b(A, X2), !c(A, X3) | CompHead}

// is transformed into
body(A) -o do_comp(A, CommVar1, ..., CommVarN, []).

do_comp(A, CommVar1, ..., CommVarN, L),
!a(A, X1),
!b(A, X2),
!c(A, X3),
(X1, X2, X3) not in L
   -o do_comp(A, CommVar1, ..., CommVarN, [(X1, X2, X3) | L]),
      CompHead.
      
do_comp(A, CommVar1, ..., CommVarN, L) -o 1.
\end{verbatim}

This suffers from a few flaws though. In one hand, we may have several \texttt{!a(A, X1)}, where \texttt{X1} has the same value. With this scheme,
only one \texttt{CompHead} will be derived. On another hand, if \texttt{CompHead} also derives anything that is used in the body of the comprehension, the comprehension may never terminate, so we must constraint \texttt{CompHead} to not include predicates used in the body.

\subsection{Comprehensions with linear facts}

When the comprehension body also contains linear facts we may use another strategy, where we consume all the linear facts to derive all the possible comprehension heads.

\begin{verbatim}
body(A) -o {X1, X2, X3 | !a(A, X1), !b(A, X2), c(A, X3) | CompHead}.

// is transformed into
body(A) -o do_comp(A, CommVar1, ..., CommVarN).

do_comp(A, CommVar1, ..., CommVarN),
!a(A, X1),
!b(A, X2),
c(A, X3)
   -o do_comp(A, CommVar1, ..., CommVarN),
      CompHead.
      
do_comp(A, CommVar1, ..., CommVarN) -o 1.
\end{verbatim}

Of course, we can also use the other approach.

As I said before, problems will arise if \texttt{CompHead} uses predicates from \texttt{CompBody}, because the comprehension may not terminate.
\end{comment}

\subsection{Extending Linear Logic with Comprehensions}

\[
\m{comp} \; A \; B \defeq 1 \; \& \; ((\forall X. A \lolli B) \otimes \m{comp} \; A \; B)
\]

\[
\m{agg} \; V \; A \; C \defeq C \; \& \; (\forall X. A \lolli \m{agg} \; (X + V) \; A \; C)
\]

An example from Meld:

\begin{verbatim}
a(H) -o [sum => S | B | !edge(H, B), !weight(H, B, S) | total(H, S)].

a(H) -o agg1(H, 0).

agg1(H, V) := total(H, V) &
             (forall B, S. !edge(H, B), !weight(H, B, S) -o agg1(H, S + V)).
\end{verbatim}


These would be the left and right rules for definitions:

\[
\infer[\m{def} \; L]
{\Delta, \compr{A'} \trnstile C}
{
   \Delta, B\theta \trnstile C & A \defeq B & A' \doteq A\theta
}
\]

\[
\infer[\m{def} \; R]
{\Delta \trnstile \compr{A'}}
{\Delta \trnstile B \theta & A \defeq B & A' \doteq A\theta}
\]

Identity expansion:

\[
\infer[\m{def} \; R]
{\compr{A'} \trnstile \compr{A'}}
{
   \infer[\m{def} \; L]
   {
      \compr{A'} \trnstile B\theta
   }
   {
      \infer[\m{id}]
      {B \theta \trnstile B \theta}
      {
      }
      & A \defeq B & A' \doteq A\theta
   }
   & A \defeq B & A' \doteq A \theta
}
\]

Cut reduction:

\[
\infer[\m{cut}]
{\Delta \trnstile C}
{
   \infer[\m{def} \; R]
   {
      \Delta \trnstile \compr{A'}
   }
   {
      \Delta \trnstile B\theta & A \defeq B & A' \doteq A'\theta
   }
   &
   \infer[\m{def} \; L]
   {
      \Delta, \compr{A'} \trnstile C
   }
   {
      \Delta, B\theta \trnstile C & A \defeq B & A'\doteq A\theta
   }
}
\]

Reduces to:

\[
\infer[\m{cut}]
{\Delta \trnstile C}
{\Delta, B\theta \trnstile C
   &
   \Delta \trnstile B\theta
}
\]

\begin{comment}
\section{Aggregates}

Aggregates have the following syntax form:

\begin{verbatim}
body(A) -o [op => F | Var-List | CompBody(A) | CompHead(A, F)}.
\end{verbatim}

Like comprehensions, we may distinguish between two types of aggregates.

\subsection{Persistent Only Aggregates}

For these types of aggregates we only use persistent facts in the body.
The transformation verifies that the variable combination has not been tried before and then applies the operator function.

\begin{verbatim}
body(A) -o [sum => W | B | !edge(A, B, W) | total(A, W)].

// is transformed into

body(A) -o do_aggregate(A, CommVar1, ..., CommVarN, 0, []).

do_aggregate(A, CommVar1, ..., CommVarN, Sum, L),
!edge(A, B, W),
(B, W) not in L
   -o do_aggregate(A, CommVar1, ..., CommVarN, Sum + W, [(B, W) | L]).
   
do_aggregate(A, CommVar1, ..., CommVarN, Sum, L) -o total(A, Sum).
\end{verbatim}

\subsection{Aggregates with linear facts}

In this case, the aggregate uses linear facts. We don't need to restrict the predicates used in the body/head since there's only a body.

\begin{verbatim}
body(A) -o [sum => W | B | !edge(A, B), weight(A, B, W) | total(A, W)].

// is transformed into

body(A) -o do_aggregate(A, CommVar1, ..., CommVarN, 0).

do_aggregate(A, CommVar1, ..., CommVarN, Sum),
!edge(A, B),
weight(A, B, W)
   -o do_aggregate(A, CommVar1, ..., CommVarN, Sum + W).
   
do_aggregate(A, CommVar1, ..., CommVarN, Sum) -o total(A, Sum).
\end{verbatim}
\end{comment}

\section{Linear Logic}

\newcommand{\sequent}[3]{#1 ; #2 \vdash #3}
\newcommand{\seqnocut}[3]{#1 ; #2 \Rightarrow #3}

\[
\infer[\one R]
{\sequent{\Gamma}{\cdot}{\one}}
{}
\tab
\infer[\one L]
{\sequent{\Gamma}{\Delta, \one}{C}}
{\sequent{\Gamma}{\Delta}{C}}
\]

\[
\infer[\with R]
{\sequent{\Gamma}{\Delta}{A \with B}}
{\sequent{\Gamma}{\Delta}{A} & \sequent{\Gamma}{\Delta}{B}}
\tab
\infer[\with L_1]
{\sequent{\Gamma}{\Delta, A \with B}{C}}
{\sequent{\Gamma}{\Delta, A}{C}}
\tab
\infer[\with L_2]
{\sequent{\Gamma}{\Delta, B \with B}{C}}
{\sequent{\Gamma}{\Delta, B}{C}}
\]

\[
\infer[\otimes R]
{\sequent{\Gamma}{\Delta, \Delta'}{A \otimes B}}
{\sequent{\Gamma}{\Delta}{A} & \sequent{\Gamma}{\Delta}{B}}
\tab
\infer[\otimes L]
{\sequent{\Gamma}{\Delta, A \otimes B}{C}}
{\sequent{\Gamma}{\Delta, A, B}{C}}
\]

\[
\infer[\lolli R]
{\sequent{\Gamma}{\Delta}{A \lolli B}}
{\sequent{\Gamma}{\Delta, A}{B}}
\tab
\infer[\lolli L]
{\sequent{\Gamma}{\Delta, \Delta', A \lolli B}{C}}
{\sequent{\Gamma}{\Delta}{A} &
   \sequent{\Gamma}{\Delta', B}{C}}
\]

\[
\infer[\forall R]
{\Psi ; \sequent{\Gamma}{\Delta}{\forall n:\tau. A}}
{\Psi, m:\tau ; \sequent{\Gamma}{\Delta}{A\{m/n\}}}
\tab
\infer[\forall L]
{\Psi ; \sequent{\Gamma}{\Delta, \forall n:\tau. A}{C}}
{\Psi \vdash M : \tau & \Psi ; \sequent{\Gamma}{\Delta, A\{M/n\}}{C}}
\]

\[
\infer[\exists R]
{\Psi \; \sequent{\Gamma}{\Delta}{\exists n: \tau. A}}
{\Psi \vdash M : \tau &
   \Psi \; \sequent{\Gamma}{\Delta}{A \{M/n\}}}
\tab
\infer[\exists L]
{\Psi ; \sequent{\Gamma}{\Delta, \exists n:\tau. A}{C}}
{\Psi, m:\tau ; \sequent{\Gamma}{\Delta, A\{m/n\}}{C}}
\]

\[
\infer[\bang R]
{\sequent{\Gamma}{\cdot}{\bang A}}
{\sequent{\Gamma}{\cdot}{A}}
\tab
\infer[\bang L]
{\sequent{\Gamma}{\Delta, \bang A}{C}}
{\sequent{\Gamma, A}{\Delta}{C}}
\tab
\infer[\m{copy}]
{\sequent{\Gamma, A}{\Delta}{C}}
{\sequent{\Gamma, A}{\Delta, A}{C}}
\]

\[
\infer[\m{def} \; R]
{\sequent{\Gamma}{\Delta}{\compr{A'}}}
{\sequent{\Gamma}{\Delta}{B\theta} &
 A \defeq B & A' \doteq A\theta}
\tab
\infer[\m{def} \; L]
{\sequent{\Gamma}{\Delta, \compr{A'}}{C}}
{
   \sequent{\Gamma}{\Delta, B\theta}{C} & A \defeq B & A' \doteq A\theta
}
\]

\[
\infer[\m{cut}]
{\sequent{\Gamma}{\Delta, \Delta'}{C}}
{\sequent{\Gamma}{\Delta}{A} &
   \sequent{\Gamma}{\Delta', A}{C}}
\tab
\infer[\m{cut}\bang]
{\sequent{\Gamma}{\Delta}{C}}
{\sequent{\Gamma}{\cdot}{A} &
   \sequent{\Gamma, A}{\Delta}{C}}
\]

\subsection{Cut Free System}

\[
\infer[\one R]
{\seqnocut{\Gamma}{\cdot}{\one}}
{}
\tab
\infer[\one L]
{\seqnocut{\Gamma}{\Delta, \one}{C}}
{\seqnocut{\Gamma}{\Delta}{C}}
\]

\[
\infer[\with R]
{\seqnocut{\Gamma}{\Delta}{A \with B}}
{\seqnocut{\Gamma}{\Delta}{A} & \seqnocut{\Gamma}{\Delta}{B}}
\tab
\infer[\with L_1]
{\seqnocut{\Gamma}{\Delta, A \with B}{C}}
{\seqnocut{\Gamma}{\Delta, A}{C}}
\tab
\infer[\with L_2]
{\seqnocut{\Gamma}{\Delta, B \with B}{C}}
{\seqnocut{\Gamma}{\Delta, B}{C}}
\]

\[
\infer[\otimes R]
{\seqnocut{\Gamma}{\Delta, \Delta'}{A \otimes B}}
{\seqnocut{\Gamma}{\Delta}{A} & \seqnocut{\Gamma}{\Delta}{B}}
\tab
\infer[\otimes L]
{\seqnocut{\Gamma}{\Delta, A \otimes B}{C}}
{\seqnocut{\Gamma}{\Delta, A, B}{C}}
\]

\[
\infer[\lolli R]
{\seqnocut{\Gamma}{\Delta}{A \lolli B}}
{\seqnocut{\Gamma}{\Delta, A}{B}}
\tab
\infer[\lolli L]
{\seqnocut{\Gamma}{\Delta, \Delta', A \lolli B}{C}}
{\seqnocut{\Gamma}{\Delta}{A} &
   \seqnocut{\Gamma}{\Delta', B}{C}}
\]

\[
\infer[\forall R]
{\Psi ; \seqnocut{\Gamma}{\Delta}{\forall n:\tau. A}}
{\Psi, m:\tau ; \seqnocut{\Gamma}{\Delta}{A\{m/n\}}}
\tab
\infer[\forall L]
{\Psi ; \seqnocut{\Gamma}{\Delta, \forall n:\tau. A}{C}}
{\Psi \vdash M : \tau & \Psi ; \seqnocut{\Gamma}{\Delta, A\{M/n\}}{C}}
\]

\[
\infer[\exists R]
{\Psi \; \seqnocut{\Gamma}{\Delta}{\exists n: \tau. A}}
{\Psi \vdash M : \tau &
   \Psi \; \seqnocut{\Gamma}{\Delta}{A \{M/n\}}}
\tab
\infer[\exists L]
{\Psi ; \seqnocut{\Gamma}{\Delta, \exists n:\tau. A}{C}}
{\Psi, m:\tau ; \seqnocut{\Gamma}{\Delta, A\{m/n\}}{C}}
\]

\[
\infer[\bang R]
{\seqnocut{\Gamma}{\cdot}{\bang A}}
{\seqnocut{\Gamma}{\cdot}{A}}
\tab
\infer[\bang L]
{\seqnocut{\Gamma}{\Delta, \bang A}{C}}
{\seqnocut{\Gamma, A}{\Delta}{C}}
\tab
\infer[\m{copy}]
{\seqnocut{\Gamma, A}{\Delta}{C}}
{\seqnocut{\Gamma, A}{\Delta, A}{C}}
\]

\[
\infer[\m{def} \; R]
{\seqnocut{\Gamma}{\Delta}{\compr{A'}}}
{\seqnocut{\Gamma}{\Delta}{B\theta} &
 A \defeq B & A' \doteq A\theta}
\tab
\infer[\m{def} \; L]
{\seqnocut{\Gamma}{\Delta, \compr{A'}}{C}}
{
   \seqnocut{\Gamma}{\Delta, B\theta}{C} & A \defeq B & A' \doteq A\theta
}
\]

\subsection{Cut Elimination Theorem}

If $\seqnocut{\Gamma}{\Delta}{A}$ and $\seqnocut{\Gamma}{\Delta', A}{C}$ then $\seqnocut{\Gamma}{\Delta, \Delta'}{C}$

\section{Optimization Ideas}

\subsection{Data vs Control}

\begin{itemize}
   \item Discover facts that work like node fields and actually never go away (only their arguments change).
   \item Discover facts that drive the computation (usually are consumed and then go away).
\end{itemize}

\subsection{JIT Compilation}

Compile most used rules to assembly.

\subsection{Improve indexing}

Do just in time indexing by gathering statistics about indexing.

\subsection{Improve rule engine}

Indexing of the current set of facts needs to be vastly improved.

\subsection{Find consuming chain of linear facts}

Sometimes a linear fact $a$ derives a $b$ that derives a $c$, etc. Once $a$ is derived we know that a set of rules will be run in sequence. We need to prove that this will happen no matter what.

\subsection{Improved fact loading}

Allow compilation of facts to a separate file. Also, load facts faster.

\section{Low Level Dynamic Semantics}

Low level dynamic semantics handle:

\begin{itemize}
\item Rule priorities.
\item No guessing of values for variables.
\item Maximality for definitions.
\end{itemize}

For the low level semantics, we consider that $\Theta$ (rules with priority)
is an ordered context of rules.

\newcommand{\applyl}[6]{\m{apply1} \; #1 ; #2 ; #3 \rightarrow #4 ; #5 ; #6}

\[
\infer[\m{rule \; app \; priority}]
{\Psi ; R, \Theta; \Phi; \Gamma; \Delta \hookrightarrow \Gamma'; \Delta' ; \Xi'}
{\applyl{\Psi}{\Gamma}{\Delta ; [R] ; (\m{rule}; \Theta; \Phi ; \Delta)}{\Gamma'}{\Delta'}{\Xi'}}
\]

\[
\infer[\m{rule \; app \; no \; priority}]
{\Psi ; \cdot; R, \Phi ; \Gamma ; \Delta \hookrightarrow \Gamma'; \Delta' ; \Xi'}
{\applyl{\Psi}{\Gamma}{\Delta ; [R] ; (\m{rule} ; \Theta; \Phi ; \Delta)}{\Gamma'}{\Delta'}{\Xi'}}
\]

Note that in the following rule, we do not guess the terms for the variables. Instead, we will try to match the variables against the available facts.

\[
\infer[\m{\forall L}]
{\applyl{\Psi}{\Gamma}{\Delta ; [\forall X : \tau. A] ; C}{\Gamma'}{\Delta'}{\Xi}}
{\applyl{\Psi}{\Gamma}{\Delta ; [A\{\m{var}(X)/X\}]; C}{\Gamma'}{\Delta'}{\Xi}}
\]

Once we get to the implication, we pick both body and head with a rule continuations. The continuation context will have the different facts that may be used to apply the rule. The body $A$ is an ordered context.

\[
\infer[\m{\lolli L}]
{\applyl{\Psi}{\Gamma}{\Delta ; [A \lolli B] ; C}{\Gamma'}{\Delta'}{\Xi'}}
{\m{matchbody} \; \Psi;\Gamma; \Delta ; \cdot ; A ; \cdot ; B ; (\cdot, C) \rightarrow \Gamma' ; \Delta' ; \Xi'}
\]

\subsection{Match Body}

This judgment goes through the ordered body context and matches the facts gainst the linear or persistent context. Constraints are put into the context at the right. Once we match everything correctly, we go through the constraint context (note: not an ordered context) and validate each constraint. If a constraint fails or a match fails, we pick the next continuation (body + facts).

For $\exists$, we do the same thing as we did above for $\forall$.

\[
\infer[\m{matchbody \; exists}]
{\m{matchbody} \; \Psi ; \Gamma ; \Delta ; \Xi ; \exists X. A', A ; B ; H ; C \rightarrow \Gamma' ; \Delta' ; \Xi'}
{\m{matchbody} \; \Psi ; \Gamma ; \Delta ; \Xi ; [\m{var}(X)/X]A', [\m{var}(X)/X]A ; [\m{var}(X)/X]B; [\m{var}(X)/X]H ; C \rightarrow \Gamma' ; \Delta' ; \Xi'}
\]


If we get $1$, we just skip it.

\[
\infer[\m{matchbody \; one}]
{\m{matchbody} \; \Psi;\Gamma;\Delta ; \Xi ; 1, A ; B ; H ; C \rightarrow \Gamma' ; \Delta' ; \Xi'}
{\m{matchbody} \; \Psi;\Gamma;\Delta ; \Xi ; A ; B ; H ; C \rightarrow \Gamma' ; \Delta' ; \Xi'}
\]

For $\otimes$ we simply deconstruct the connective and keep both elements in the ordered context.

\[
\infer[\m{matchbody \; split}]
{\m{matchbody} \; \Psi;\Gamma;\Delta; \Xi ; A_1 \otimes A_2, A ; B ; H ; C \rightarrow \Gamma' ; \Delta' ; \Xi'}
{\m{matchbody} \; \Psi;\Gamma;\Delta; \Xi ; A_1, A_2, A ; B ; H ; C \rightarrow \Gamma' ; \Delta' ; \Xi'}
\]

This is the constraint case. We simply move the constraint to the constraint context.

\[
\infer[\m{matchbody \; constraint}]
{\m{matchbody} \; \Psi;\Gamma;\Delta; \Xi ; \bang\constraint{e}, A; B ; H ; C \rightarrow \Gamma' ; \Delta' ; \Xi'}
{\m{matchbody} \; \Psi ; \Gamma; \Delta; \Xi ; A ; \bang\constraint{e}, B ; H ; C \rightarrow \Gamma' ; \Delta' ; \Xi'}
\]

Finally, the linear case! Here we have two cases, either we have facts in the linear context of this type or we don't.

\[
\infer[\m{matchbody \; linear}]
{\m{matchbody} \; \Psi;\Gamma;\Delta, \Delta_f; \Xi ; \fact{name}{e_1}{e_2, ..., e_n}, A; B; H; (C_1, C_2) \rightarrow \Gamma' ; \Delta'; \Xi'}
{\begin{array}{c}
   \Delta_f = \m{list} \; [ \fact{name}{v_1}{v_2, ..., v_n} | Xs] \\
    NC = (\m{body}; \fact{name}{e_1}{e_2, ..., e_n} ; Xs ; \Delta ; \Xi ; A ; B ; H; C_1) \\
    \Delta_1 = \Delta, \Delta_f - \{\fact{name}{v_1}{v_2, ..., v_n}\} \\
   \m{matchfact} \; \Psi;\Gamma; \Delta_1; \fact{name}{v_1}{v_2, ..., v_n}, \Xi ; [v_1, ..., v_n] ; [e_1, ..., e_n]; A ; B ; H; (NC, C_2) \rightarrow \Gamma' ; \Delta'; \Xi' \\
 \end{array}
}
\]

\[
\infer[\m{matchbody \; linear \; empty}]
{\m{matchbody} \; \Psi;\Gamma;\Delta,\Delta_f; \Xi ; \fact{name}{e_1}{e_2, ..., e_n}, A ; B; H; C \rightarrow \Gamma';\Delta';\Xi'}
{\begin{array}{c}
   \Delta_f = [] \\
   \m{cont} \; \Psi; \Gamma ; C \rightarrow \Gamma';\Delta';\Xi'
  \end{array}
}
\]

Persistent facts are very similar, except we don't mess with the linear context.

\[
\infer[\m{matchbody \; persistent}]
{\m{matchbody} \; \Psi;\Gamma, \Gamma_f;\Delta; \Xi ; \bang\fact{name}{e_1}{e_2, ..., e_n}, A; B; H; (C_1, C_2) \rightarrow \Gamma' ; \Delta'; \Xi'}
{\begin{array}{c}
   \Gamma_f = \m{list} \; [\bang\fact{name}{v_1}{v_2, ..., v_n} | Xs] \\
    NC = (\m{body}; \bang\fact{name}{e_1}{e_2, ..., e_n} ; Xs ; \Delta ; \Xi ; A ; B ; H; C_1) \\
   \m{matchfact} \; \Psi;\Gamma, \Gamma_f; \Delta; \Xi ; [v_1, ..., v_n] ; [e_1, ..., e_n]; A ; B ; H; (NC, C_2) \rightarrow \Gamma' ; \Delta'; \Xi' \\
 \end{array}
}
\]

\[
\infer[\m{matchbody \; persistent \; empty}]
{\m{matchbody} \; \Psi;\Gamma, \Gamma_f;\Delta; \Xi ; \bang\fact{name}{e_1}{e_2, ..., e_n}, A ; B; H; C \rightarrow \Gamma';\Delta';\Xi'}
{\begin{array}{c}
   \Gamma_f = [] \\
   \m{cont} \; \Psi; \Gamma ; C \rightarrow \Gamma';\Delta';\Xi'
  \end{array}
}
\]

Now we get to the case where we have no more facts to process. We use $\m{matchconstr}$ to match the required constraints. Note that all constraints will be instantiated at this point, so they can be evaluated.

\[
\infer[\m{matchbody \; end}]
{\m{matchbody} \; \Psi; \Gamma ; \Delta; \Xi; \cdot ; B ; H; C \rightarrow \Gamma'; \Delta'; \Xi'}
{\m{matchconstr} \; \Psi; \Gamma; \Delta; \Xi; B; H; C \rightarrow \Gamma'; \Delta'; \Xi'}
\]

\subsubsection{Match Facts}

The following judgments match the fact from the context with the fact from the rule.

\[
\infer[\m{matchfact \; var}]
{\m{matchfact} \; \Psi;\Gamma;\Delta;\Xi; [v | v_s] ; [\m{var}(X) \| e_s]; A; B; H; C \rightarrow \Gamma'; \Delta'; \Xi'}
{\m{matchfact} \; \Psi;\Gamma;\Delta;\Xi; v_s; [v/\m{var}(X)]e_s; [v/\m{var}(X)]A; [v/\m{var}(X)]B; [v/\m{var}(X)]H; C \rightarrow \Gamma'; \Delta'; \Xi'}
\]

\[
\infer[\m{matchfact \; equal}]
{\m{matchfact} \; \Psi;\Gamma;\Delta;\Xi; [v_1 | v_s] ; [v_2 | e_s]; A; B; H; C \rightarrow \Gamma'; \Delta'; \Xi'}
{\m{matchfact} \; \Psi;\Gamma;\Delta;\Xi; v_s; e_s; A; B; H; C \rightarrow \Gamma'; \Delta'; \Xi' & \equal{v_1}{v_2}}
\]

If they are not equal, we fail and grab the next continuation:

\[
\infer[\m{matchfact \; not \; equal}]
{\m{matchfact} \; \Psi;\Gamma;\Delta;\Xi; [v_1 | v_s] ; [v_2 | e_s]; A; B; H; C \rightarrow \Gamma'; \Delta'; \Xi'}
{\m{cont} \; \Psi ; \Gamma ; C \rightarrow \Gamma'; \Delta'; \Xi' & v_1 \neq v_2}
\]

\[
\infer[\m{matchfact \; done}]
{\m{matchfact} \; \Psi;\Gamma; \Delta; \Xi ; [] ; []; A ; B ; H; C \rightarrow \Gamma' ; \Delta'; \Xi'}
{\m{matchbody} \; \Psi;\Gamma;\Delta;\Xi; A; B; H; C \rightarrow \Gamma'; \Delta'; \Xi'}
\]

\subsubsection{Match Constraints}

If a constraint succeeds, we keep going on.

\[
\infer[\m{matchconstr \; true}]
{\m{matchconstr} \; \Psi;\Gamma;\Delta;\Xi; \bang\constraint{e}, B ; H; C \rightarrow \Gamma'; \Delta'; \Xi'}
{\eval{e}{\m{bool}(\m{true})} & \m{matchconstr} \; \Psi;\Gamma;\Delta;\Xi; B; H; C \rightarrow \Gamma'; \Delta'; \Xi'}
\]

If not, we get a continuation to try another fact.

\[
\infer[\m{matchconstr \; false}]
{\m{matchconstr} \; \Psi;\Gamma;\Delta;\Xi; \bang\constraint{e}, B ; H; C \rightarrow \Gamma'; \Delta'; \Xi'}
{\eval{e}{\m{bool}(\m{false})} & \m{cont} \; \Psi;\Gamma; C \rightarrow \Gamma'; \Delta'; \Xi'}
\]

Once all constraints are validated, we have succeeded in matching the body rule, so we can start deriving new facts.
Note that we get rid of all continuations.

\[
\infer[\m{matchconstr \; end}]
{\m{matchconstr} \; \Psi;\Gamma;\Delta;\Xi; \cdot ; H ; (C_1, (\m{rule}; ...)) \rightarrow \Gamma';\Delta';\Xi'}
{\m{derive1} \; \Psi;\Gamma;\Delta;\Xi; \cdot; \cdot; H ; \cdot \rightarrow \Gamma';\Delta';\Xi'}
\]

The derive continuation is kept however. This way we can return back to the original derivation.

\[
\infer[\m{matchconstr \; end}]
{\m{matchconstr} \; \Psi;\Gamma;\Delta;\Xi; \cdot ; H ; (C_1, (\m{derive}; \Delta''; \Xi''; \Gamma_1 ; \Delta_1 ; K ; \Omega)) \rightarrow \Gamma';\Delta';\Xi'}
{\m{derive1} \; \Psi;\Gamma;\Delta;\Xi; \cdot; \cdot; H ; (\m{derive}; \Delta'' ; \Xi''; \Gamma_1; \Delta_1 ; K; \Omega) \rightarrow \Gamma';\Delta';\Xi'}
\]

\subsubsection{Continuation}

If we have no more fact continuations, we need to get the rule continuation to try another rule.

\[
\infer[\m{cont \; rule}]
{\m{cont} \; \Psi ; \Gamma ; (\cdot , (\m{rule} ; \Theta ; \Phi ; \Delta)) \rightarrow \Gamma'; \Delta'; \Xi'}
{\Psi ; \Theta; \Phi; \Gamma ; \Delta \hookrightarrow \Gamma'; \Delta' ; \Xi'}
\]

... If there is a derive continuation, it means that an aggregate or continuation has failed.

\[
\infer[\m{cont \; comp}]
{\m{cont} \; \Psi ; \Gamma ; (\cdot , (\m{derive}; \Delta ; \Xi; \Gamma_1; \Delta_1; \comprehension{A}; \Omega)) \rightarrow \Gamma'; \Delta'; \Xi'}
{\m{derive1} \; \Psi ; \Gamma ; \Delta ; \Xi ; \Gamma_1 ; \Delta_1 ; \Omega ; \cdot \rightarrow \Gamma' ; \Delta' ; \Xi'}
\]

\[
\infer[\m{cont \; aggregate}]
{\m{cont} \; \Psi ; \Gamma ; (\cdot , (\m{derive}; \Delta ; \Xi; \Gamma_1; \Delta_1; \aggdef{Op}{V}{(x. A(x))}{(y. B(y))}; \Omega)) \rightarrow \Gamma'; \Delta'; \Xi'}
{\m{derive1} \; \Psi ; \Gamma ; \Delta ; \Xi ; \Gamma_1 ; \Delta_1 ; B(V), \Omega; \cdot \rightarrow \Gamma' ; \Delta' ; \Xi'}
\]

If we have a fact continuation but no more facts for that continuation, we fail and continue:

\[
\infer[\m{cont \; body \; fail}]
{\m{cont} \; \Psi ; \Gamma ; ((\m{body} ; \fact{name}{e_1}{e_2, ..., e_n} ; []; \Delta ; \Xi ; A ; B; H; C), C_2) \rightarrow \Gamma'; \Delta'; \Xi'}
{
   \m{cont} \; \Psi ; \Gamma ; (C, C_2) \rightarrow \Gamma'; \Delta'; \Xi'
}
\]

If we have a fact continuation and also more facts, restore the continuation and continue:

\[
\infer[\m{cont \; body \; ok}]
{\m{cont} \; \Psi ; \Gamma ; ((\m{body} ; \fact{name}{e_1}{e_2, ..., e_n} ; [\fact{name}{v_1}{v_2, ..., v_n} | Xs]; \Delta ; \Xi ; A ; B ; H; C_1), C_2) \rightarrow \Gamma'; \Delta'; \Xi'}
{
   \begin{array}{c}
   \Delta_1 = \Delta - {\fact{name}{v_1}{v_2, ..., v_n}}\\
   NC = (\m{body}; \fact{name}{e_1}{e_2, ..., e_n}; \Delta; A; B; H; C_1) \\
   \m{matchfact} \; \Psi ; \Gamma; \Delta_1 ; \fact{name}{v_1}{v_2, ..., v_n}, \Xi ; [v_1, ..., v_n]; [e_1, ..., e_n]; A; B; H; (NC, C_2) \rightarrow \Gamma' ; \Delta'; \Xi'\\
   \end{array}
}
\]

\subsubsection{Derive}

\[
\infer[\m{derive \; \otimes}]
{\m{derive1} \; \Psi; \Gamma ; \Delta ; \Xi ; \Gamma_1; \Delta_1 ; A \otimes B, \Omega ; C \rightarrow \Gamma'; \Delta'; \Xi'}
{\m{derive1} \; \Psi; \Gamma ; \Delta ; \Xi ; \Gamma_1; \Delta_1 ; A, B, \Omega; C \rightarrow \Gamma'; \Delta'; \Xi'}
\]

\[
\infer[\m{derive \; exists}]
{\m{derive1} \; \Psi ; \Gamma ; \Delta ; \Xi; \Gamma_1 ; \Delta_1 ; \exists X : \m{addr}. A, \Omega ; C \rightarrow \Gamma'; \Delta' ; \Xi'}
{\m{derive1} \; \Psi ; \Gamma ; \Delta ; \Xi; \Gamma_1 ; \Delta_1 ; [x/X]A, \Omega ; C \rightarrow \Gamma'; \Delta'; \Xi'
   & x = \m{new} \; \m{addr}(A)}
\]

\[
\infer[\m{derive \; 1}]
{\m{derive1} \; \Psi ; \Gamma ; \Delta ; \Xi; \Gamma_1 ; \Delta_1 ; 1, \Omega ; C \rightarrow \Gamma'; \Delta' ; \Xi'}
{\m{derive1} \; \Psi ; \Gamma ; \Delta ; \Xi; \Gamma_1 ; \Delta_1 ; \Omega ; C \rightarrow \Gamma'; \Delta' ; \Xi'}
\]


\[
\infer[\m{derive \; fact}]
{\m{derive1} \; \Psi ; \Gamma ; \Delta ; \Xi ; \Gamma_1; \Delta_1 ; \fact{name}{e_1}{e_2, ..., e_n}, \Omega ; C \rightarrow \Gamma'; \Delta'; \Xi'}
{\begin{array}{ccc}
   \eval{e_1}{v_1} & ... & \eval{e_n}{v_n} \\
   \multicolumn{3}{c}{\m{derive1} \; \Psi ; \Gamma ; \Delta ; \Xi ; \Gamma_1 ; \Delta_1, \fact{name}{v_1}{v_2, ..., v_n} ; \Omega ; C \rightarrow \Gamma'; \Delta' ; \Xi'} \\
   \end{array}
}
\]

\[
\infer[\m{derive \; \bang fact}]
{\m{derive1} \; \Psi ; \Gamma ; \Delta ; \Xi ; \Gamma_1 ; \Delta_1 ; \bang \fact{name}{e_1}{e_2, ..., e_n}, \Omega ; C \rightarrow \Gamma'; \Delta' \Xi'}
{\begin{array}{ccc}
   \eval{e_1}{v_1} & ... & \eval{e_n}{v_n} \\
   \multicolumn{3}{c}{\m{derive1} \; \Psi ; \Gamma ; \Delta ; \Xi; \Gamma_1, \bang\fact{name}{v_1}{v_2, ..., v_n} ; \Delta_1 ; \Omega ; C \rightarrow \Gamma'; \Delta'; \Xi'} \\
   \end{array}
}
\]

For the comprehension, we define a new continuation for the current state of derivation and call $\m{apply}$ in order to attempt applying the comprehension. Comprehension fails in one of the $\m{cont}$ cases. It succeeds when derive reaches the end and a continuation is in place.


\[
\infer[\m{derive \; comprehension}]
{\m{derive1} \; \Psi ; \Gamma; \Delta;\Xi;\Gamma_1;\Delta_1; \comprehension{A}, \Omega; \cdot \rightarrow \Gamma' ;\Delta'; \Xi'}
{\applyl{\Psi}{\Gamma}{\Delta ; [A] ; (\m{derive1}; \Delta ; \Xi; \Gamma_1; \Delta_1; \comprehension{A} ; \Omega)}{\Gamma'}{\Delta'}{\Xi'}}
\]

We first change the aggregate definition. Note that this only happens at this derivation level (no continuation possible).

\[
\infer[\m{derive \; aggregate}]
{\m{derive1} \; \Psi; \Gamma; \Delta; \Xi; \Gamma_1; \Delta_1; \aggregate{Op}{X}{A}{B}, \Omega; \cdot \rightarrow \Gamma' ; \Delta'; \Xi'}
{\aggregatestart{Op}{V} & \m{derive1} \; \Psi;\Gamma;\Delta;\Xi;\Gamma_1;\Delta_1; \aggdef{Op}{V}{(x. A(x))}{(y. B(y))}, \Omega; \cdot \rightarrow \Gamma'; \Delta'; \Xi'}
\]

When unfolding the aggregate and if there is an aggregate continuation (aggregate has already been applied multiple times), we need to change the definition of the aggregate inside the continuation. Note that the derivation context ($\Omega$) must only contain the aggregate.

\fontsize{8}{9.5}\selectfont
\[
\infer[\m{derive \; aggregate \; unfold}]
{\m{derive1} \; \Psi; \Gamma; \Delta; \Xi; \Gamma_1; \Delta_1; \aggdef{Op}{V'}{(x. A(x))}{(y. B(y))} ; (\m{derive} ; \Delta'' ; \Xi''; \Gamma'_1; \Delta'_1; \aggdef{Op}{V}{(x. A(x))}{(y. B(y))} ; \Omega) \rightarrow \Gamma'; \Delta'; \Xi'}
{\begin{array}{c}
   \applyl{\Psi}{\Gamma}{\Delta ; [\forall X'. A(X') \lolli \aggdef{Op}{E}{(x. A(x))}{(y. B(y))}] ; (\m{derive}; \Delta ; \Xi, \Xi''; \Gamma_1, \Gamma'_1; \Delta_1, \Delta'_1; \aggdef{Op}{V'}{(x. A(x))}{(y. B(y))} ; \Omega)}{\Gamma'}{\Delta'}{\Xi'}\\
   \aggregateop{Op}{V'}{X'}{E} \\
      \end{array}
}
\]

\fontsize{10}{9.5}\selectfont
Otherwise, if we get an aggregate without a continuation:

\fontsize{8}{9.5}\selectfont
\[
\infer[\m{derive \; aggregate \; unfold}]
{\m{derive1} \; \Psi; \Gamma; \Delta; \Xi; \Gamma_1; \Delta_1; \aggdef{Op}{V}{(x. A(x))}{(y. B(y))}, \Omega ; \cdot \rightarrow \Gamma'; \Delta'; \Xi'}
{\begin{array}{c}
   \applyl{\Psi}{\Gamma}{\Delta ; [\forall X'. A(X') \lolli \aggdef{Op}{E}{(x. A(x))}{(y. B(y))}] ; (\m{derive}; \Delta ; \Xi; \Gamma_1; \Delta_1; \aggdef{Op}{V}{(x. A(x))}{(y. B(y))} ; \Omega)}{\Gamma'}{\Delta'}{\Xi'}\\
   \aggregateop{Op}{V}{X'}{E} \\
      \end{array}
}
\]

\fontsize{10}{9.5}\selectfont

If $\m{derive}$ ends and there is a continuation, it means that either the aggregate or comprehension can be reused again.

\[
\infer[\m{derive \; comprehension \; end}]
{\m{derive1} \; \Psi ; \Gamma ; \Delta; \Xi; \Gamma_1; \Delta_1 ; \cdot ; (\m{derive}; \Delta'' ; \Xi''; \Gamma'_1; \Delta'_1; \comprehension{A} ; \Omega) \rightarrow \Gamma' ; \Delta' ; \Xi'}
{
   \m{derive1} \; \Psi ; \Gamma ; \Delta ; \Xi, \Xi''; \Gamma_1, \Gamma'_1; \Delta_1, \Delta'_1; \comprehension{A}, \Omega; \cdot \rightarrow \Gamma'; \Delta'; \Xi'}
\]

This is the axiom that wraps everything up. If no rule is applicable in the system, then there is no valid proof derivation.

\[
\infer[\m{derive \; end}]
{\m{derive1} \; \Psi ; \Gamma ; \Delta; \Xi; \Gamma_1; \Delta_1; \cdot ; \cdot \rightarrow \Gamma_1 ; \Delta_1 ; \Xi}
{}
\]

\section{Simplified Systems}

\newcommand{\mz}{\m{m}_0 \;}
\newcommand{\mo}{\m{m}_1 \;}
\newcommand{\dz}{\m{d}_0 \;}
\newcommand{\done}{\m{d}_1 \;}
\newcommand{\az}{\m{a}_0 \;}
\newcommand{\ao}{\m{a}_1 \;}
\newcommand{\doz}{\m{do}_0 \;}
\newcommand{\doo}{\m{do}_1 \;}
\newcommand{\cont}{\m{cont} \;}
\newcommand{\contc}{\m{contc} \;}
\newcommand{\dc}{\m{dc} \;}

\subsection{High Level System}

\[
\infer[]
{\mz \cdot ; 1 \rightarrow 1}
{}
\]

\[
\infer[]
{\mz p ; p \rightarrow p}
{}
\]

\[
\infer[]
{\mz \Delta_1, \Delta_2 ; A \otimes B \rightarrow \Xi_1, \Xi_2}
{\mz \Delta_1 ; A \rightarrow \Xi_1 & \mz \Delta_2 ; B \rightarrow \Xi_2}
\]

\[
\infer[]
{\dz \Delta ; \Xi ; \Delta_1 ; p, \Omega \rightarrow \Xi' ; \Delta'}
{\dz \Delta ; \Xi ; p, \Delta_1 ; \Omega \rightarrow \Xi' ; \Delta'}
\tab
\infer[]
{\dz \Delta; \Xi; \Delta_1; 1, \Omega \rightarrow \Xi';\Delta'}
{\dz \Delta; \Xi; \Delta_1; \Omega \rightarrow \Xi';\Delta'}
\]


\[
\infer[]
{\dz \Delta;\Xi;\Delta_1;A \otimes B, \Omega \rightarrow \Xi'; \Delta'}
{\dz \Delta;\Xi;\Delta_1;A, B, \Omega \rightarrow \Xi';\Delta'}
\tab
\]

\[
\infer[]
{\dz \Delta; \Xi; \Delta_1; A \with B, \Omega \rightarrow \Xi';\Delta'}
{\dz \Delta; \Xi; \Delta_1; A, \Omega \rightarrow \Xi';\Delta'}
\tab
\infer[]
{\dz \Delta; \Xi; \Delta_1; A \with B, \Omega \rightarrow \Xi';\Delta'}
{\dz \Delta; \Xi; \Delta_1; B, \Omega \rightarrow \Xi';\Delta'}
\]

\[
\infer[]
{\dz \Delta ; \Xi; \Delta_1; \m{comp} A \lolli B, \Omega \rightarrow \Xi';\Delta'}
{\dz \Delta; \Xi; \Delta_1; 1 \with (A \lolli B \otimes \m{comp} A \lolli B), \Omega \rightarrow \Xi';\Delta'}
\]

\[
\infer[]
{\dz \Delta_a, \Delta_b; \Xi; \Delta_1; A \lolli B, \Omega \rightarrow \Xi';\Delta'}
{\mz \Delta_a; A \rightarrow \Delta_a & \dz \Delta_b ; \Xi; \Delta_a; \Delta_1; B, \Omega \rightarrow \Xi'; \Delta'}
\]

\[
\infer[]
{\dz \Delta; \Xi'; \Delta'; \cdot \rightarrow \Xi';\Delta'}
{}
\]

\[
\infer[]
{\az \Delta, \Delta''; A \lolli B \rightarrow \Xi'; \Delta'}
{\mz \Delta; A \rightarrow \Delta & \dz \Delta''; \Delta; \cdot ; B \rightarrow \Xi'; \Delta'}
\]

\[
\infer[]
{\doz \Delta; R, \Phi \rightarrow \Xi';\Delta'}
{\doz \Delta; R \rightarrow \Xi';\Delta'}
\]

\subsection{Low Level System}

\[
\infer[]
{\done \Delta; \Xi; \Delta_1; p, \Omega; C \rightarrow \Xi'; \Delta'}
{\done \Delta; \Xi; p, \Delta_1; \Omega; C \rightarrow \Xi'; \Delta'}
\tab
\infer[]
{\done \Delta; \Xi; \Delta_1; 1, \Omega; C \rightarrow \Xi';\Delta'}
{\done \Delta; \Xi; \Delta_1; \Omega; C \rightarrow \Xi'; \Delta'}
\]

\[
\infer[]
{\done \Delta; \Xi; \Delta_1; A \otimes B, \Omega; C \rightarrow \Xi'; \Delta'}
{\done \Delta; \Xi; \Delta_1; A, B, \Omega; C \rightarrow \Xi';\Delta'}
\]

\[
\infer[]
{\done \Delta; \Xi; \Delta_1; \m{comp} A \lolli B, \Omega; \cdot \rightarrow \Xi; \Delta'}
{\ao \Delta; A \lolli B; (\done \Delta; \Xi; \Delta_1; \m{comp} A \lolli B, \Omega; \cdot) \rightarrow \Xi'; \Delta'}
\]

\[
\infer[ok]
{\mo \Delta, p ; \Xi; p, \Omega; H; C \rightarrow \Xi'; \Delta'}
{\mo \Delta; \Xi, p; \Omega; H; C \rightarrow \Xi'; \Delta'}
\tab
\infer[fail]
{\mo \Delta; \Xi; p, \Omega; H; C \rightarrow \Xi'; \Delta'}
{p \notin \Delta & \cont C ; H; \Xi'; \Delta'}
\]

\[
\infer[]
{\mo \Delta; \Xi; A \otimes B, \Omega ; H ; C \rightarrow \Xi'; \Delta'}
{\mo \Delta; \Xi; A, B, \Omega; H; C \rightarrow \Xi';\Delta'}
\]

\[
\infer[\m{rule} \; \m{cont}]
{\mo \Delta; \Xi; \cdot ; H; (\Phi; \Delta) \rightarrow \Xi'; \Delta'}
{\done \Delta; \Xi; \cdot ; H; \cdot \rightarrow \Xi'; \Delta'}
\]

\[
\infer[\m{compr} \; \m{cont}]
{\mo \Delta; \Xi; \cdot; H; (\done \Delta''; \Xi''; \Delta_1; \m{comp} A \lolli B, \Omega; \cdot) \rightarrow \Xi'; \Delta'}
{\done \Delta; \Xi; \cdot ; H; (\done \Delta''; \Xi''; \Delta_1; \m{comp} A \lolli B, \Omega; \cdot) \rightarrow \Xi'; \Delta'}
\]

\[
\infer[]
{\cont (\Phi; \Delta); \Xi'; \Delta'}
{\doo \Delta; \Phi \rightarrow \Xi'; \Delta'}
\]

\[
\infer[]
{\cont (\done \Delta''; \Xi''; \Delta_1; \m{comp} A \lolli B, \Omega); \Xi'; \Delta'}
{\done \Delta; \Xi''; \Delta_1; \Omega; \cdot \rightarrow \Xi'; \Delta'}
\]

\[
\infer[]
{\done \Delta; \Xi; \Delta_1; \cdot; \cdot \rightarrow \Xi; \Delta_1}
{}
\]

\[
\infer[]
{\done \Delta; \Xi; \Delta_1; \cdot; (\done \Delta''; \Xi''; \Delta''_1; \m{comp} A \lolli B, \Omega) \rightarrow \Xi'; \Delta'}
{\done \Delta; \Xi, \Xi''; \Delta_1, \Delta''_1; \m{comp} A \lolli B, \Omega; \cdot \rightarrow \Xi'; \Delta'}
\]

\[
\infer[]
{\ao \Delta; A \lolli B; C \rightarrow \Xi'; \Delta'}
{\mo \Delta; \cdot; A; B; C \rightarrow \Xi'; \Delta'}
\]

\[
\infer[]
{\doo \Delta; R, \Phi \rightarrow \Xi'; \Delta'}
{\ao \Delta; R; (\Phi; \Delta) \rightarrow \Xi';\Delta'}
\]

\subsection{Low level comprehension match succeeds or fails}

If $\mo \Delta'', \Delta_1, ..., \Delta_n; \Xi; \Omega; H; (\done \Delta'''; \Xi''; \Delta_1; \m{comp} A \lolli B, \Omega') \rightarrow \Xi'; \Delta'$ then either:

\begin{itemize}
\item $\cont (\done \Delta'''; \Xi''; \Delta_1; \m{comp} A \lolli B, \Omega'); \Xi'; \Delta'$ or
\item $\mo \Delta''; \Xi, \Delta_1, ..., \Delta_n' \cdot ; H ; (\done ...) \rightarrow \Xi'; \Delta'$ and $\Omega = \Omega_1, ..., \Omega_n$ where $\mz \Delta_1; \Omega_1 \rightarrow \Delta_1$, ..., $\mz \Delta_n ; \Omega_n \rightarrow \Delta_n$.
\end{itemize}

It's trivial by induction on the assumption, except the case $p, \Omega$ and $A \otimes B, \Omega$.

\subsection{Low level comprehension gives one match}

If $\mo \Delta'', \Xi''; \cdot; A ; H; (\done \Delta'''; \Xi; \Delta_1; \m{comp} A \lolli B, \Omega') \rightarrow \Xi'; \Delta'$ then either

\begin{itemize}
\item $\cont (\done \Delta'''; \Xi; \Delta_1; \m{comp} A \lolli B, \Omega'); \Xi'; \Delta'$ or
\item $\mo \Delta'' ; \Xi''; \cdot; H; (\done \Delta'''; \Xi; \Delta_1; \m{comp} A \lolli B, \Omega') \rightarrow \Xi'; \Delta'$ and $\mz \Xi''; A \rightarrow \Xi''$
\end{itemize}

This follows trivially from the previous theorem.

\subsection{Comprehension head is another derivation theorem}

If $\done \Delta; \Xi; \Delta_1; \Omega'; (\done \Delta''; \Xi''; \Delta'_1; \m{comp} A \lolli B, \Omega) \rightarrow \Xi'; \Delta'$ \\ then \\ $\done \Delta; \Xi, \Xi''; \Delta_1, \Delta'_1; \Omega', \m{comp} A \lolli B, \Omega; \cdot \rightarrow \Xi'; \Delta'$.

\begin{itemize}
\item $p$

$\done \Delta; \Xi; \Delta_1; p, \Omega''; (\done ...) \rightarrow \Xi'; \Delta'$ \hfill (1) assumption \\
$\Omega' = p, \Omega''$ \hfill (2) from (1) \\
$\done \Delta; \Xi; p, \Delta_1; \Omega''; (\done ...) \rightarrow \Xi'; \Delta'$ \hfill (3) inversion of (1) \\
$\done \Delta; \Xi, \Xi''; p, \Delta_1, \Delta'_1; \Omega'', \m{comp} A \lolli B, \Omega; \cdot \rightarrow \Xi'; \Delta'$ \hfill (4) i.h. on (3) \\
$\done \Delta; \Xi, \Xi''; \Delta_1, \Delta'_1; p, \Omega'', \m{comp} A \lolli B, \Omega; \cdot \rightarrow \Xi'; \Delta'$ \hfill (5) apply rule on (4) \\

\item $A \otimes B$

$\done \Delta; \Xi, \Xi''; \Delta_1, \Delta'_1; A, B, \Omega''; \m{comp} A \lolli B, \Omega; \cdot \rightarrow \Xi'; \Delta'$ \hfill (1) by i.h. \\
$\done \Delta; \Xi, \Xi''; \Delta_1, \Delta'_1; A \otimes B, \Omega''; \m{comp} A \lolli B, \Omega; \cdot \rightarrow \Xi'; \Delta'$ \hfill (2) rule application on (1) \\

\item $\cdot$

$\done \Delta; \Xi; \Delta_1; \cdot; (\done \Delta''; \Xi''; \Delta'_1; \m{comp} A \lolli B, \Omega) \rightarrow \Xi'; \Delta'$ \hfill (1) assumption \\
$\done \Delta; \Xi, \Xi''; \Delta_1, \Delta'_1; \m{comp} A \lolli B, \Omega; \cdot \rightarrow \Xi'; \Delta'$ \hfill (2) inversion of (1) \\

\end{itemize}

\subsection{Low level matching gives high level matching theorem}

If \\
$\mo \Delta, \Delta_1, ..., \Delta_n; \Xi; A_1, ..., A_n; H; \cdot \rightarrow \Xi'; \Delta'$ \\
then \\
$\mz \Delta_1; A_1 \rightarrow \Delta_1$ through $\mz \Delta_n; A_n \rightarrow \Delta_n$ and \\
$\mo \Delta; \Xi, \Delta_1, ..., \Delta_n; \cdot; H; \cdot \rightarrow \Xi'; \Delta'$

Induction on the assumption judgment.

\begin{itemize}
\item $1, \Omega$

Trivial.

\item $p, \Omega$ and $p \notin \Delta$

Not applicable.

\item $p, \Omega$

$\mo \Delta, \Delta_1, ..., \Delta_n, p; \Xi; p, A_1, ..., A_n; H; \cdot \rightarrow \Xi'; \Delta'$ \hfill (1) assumption \\
$\mo \Delta, \Delta_1, ..., \Delta_n; \Xi, p; A_1, ..., A_n; H; \cdot \rightarrow \Xi'; \Delta'$ \hfill (2) inversion of (1) \\
$\mz \Delta_1 ; A_1 \rightarrow \Delta_1$ ... $\mz \Delta_n ; A_n \rightarrow \Delta_n$ \hfill (3) induction on (2) \\
$\mo \Delta; \Xi, p, \Delta_1, ..., \Delta_n; \cdot ; H ; \cdot \rightarrow \Xi'; \Delta'$ \hfill (4) induction on (2) \\
$\mz p; p \rightarrow p$ \hfill (5) axiom \\

\item $A \otimes B, \Omega$

$\mo \Delta, \Delta_1, ..., \Delta_n; \Xi; A \otimes B, A_1, ..., A_n; H; \cdot \rightarrow \Xi'; \Delta'$ \hfill (1) assumption \\
$\mo \Delta, \Delta_1, ..., \Delta_n; \Xi; A, B, A_1, ..., A_{n-2}; H; \cdot \rightarrow \Xi'; \Delta'$ \hfill (2) inversion of (1) \\
$\mo \Delta; \Xi, \Delta_1, ..., \Delta_n; \cdot ; H; \cdot \rightarrow \Xi'; \Delta'$ \hfill (3) induction on (2) \\
$\mz \Delta_1 ; A \rightarrow \Delta_1$, $\mz \Delta_2; B \rightarrow \Delta_2$, $\mz \Delta_n; A_{n-2} \rightarrow \Delta_n$ \hfill (4) induction on (2) \\

\item $\cdot$

$\mo \Delta ; \Xi; \cdot ; H; (\cdot ; \Delta'') \rightarrow \Xi'; \Delta'$ \hfill (1) assumption \\
$n = 0$ \hfill since $\Omega = \cdot$ \\

\end{itemize}

\subsection{Derive soundness}

If $\done \Delta; \Xi; \Delta_1; \Omega; \cdot \rightarrow \Xi'; \Delta'$ then \\
$\dz \Delta; \Xi; \Delta_1; \Omega \rightarrow \Xi'; \Delta'$


By induction on the assumption.

\begin{itemize}
\item $p, \Omega$

$\done \Delta; \Xi; \Delta_1; p, \Omega; \cdot \rightarrow \Xi'; \Delta'$ \hfill (1) assumption \\
$\done \Delta; \Xi; \Delta_1, p; \Omega; \cdot \rightarrow \Xi'; \Delta'$ \hfill (2) inversion of (1) \\
$\dz \Delta; \Xi; \Delta_1, p; \Omega \rightarrow \Xi'; \Delta'$ \hfill (3) by induction on (2) \\
$\dz \Delta; \Xi; \Delta_1; p, \Omega \rightarrow \Xi'; \Delta'$ \hfill (4) rule application on (3) \\

\item $1, \Omega$

Same as before.

\item $A \otimes B, \Omega$

Same as before.

\item $\comp A \lolli B, \Omega$

$\done \Delta; \Xi; \Delta_1; \comp A \lolli B, \Omega; \cdot \rightarrow \Xi'; \Delta'$ \hfill (1) assumption \\
$\ao \Delta; A \lolli B; (\done \Delta; \Xi; \Delta_1; \comp A \lolli B, \Omega) \rightarrow \Xi'; \Delta'$ \hfill (2) inversion of (1) \\
$\mo \Delta; \cdot; A; B ; (\done \Delta; \Xi; \Delta_1; \comp A \lolli B, \Omega) \rightarrow \Xi'; \Delta'$ \hfill (3) inversion of (2) \\
Using (3) on theorem "Low level comprehension gives one match" we get two subcases:

\begin{itemize}
\item Comprehension fails:

$\cont (\done \Delta; \Xi; \Delta_1; \comp A \lolli B, \Omega); \Xi' \Delta'$ \hfill (4) from theorem \\
$\done \Delta; \Xi; \Delta_1; \Omega; \cdot \rightarrow \Xi'; \Delta'$ \hfill (5) inversion of (4) \\
$\dz \Delta; \Xi; \Delta_1; \Omega \rightarrow \Xi'; \Delta'$ \hfill (6) induction on (5) \\
$\dz \Delta; \Xi; \Delta_1; 1, \Omega \rightarrow \Xi';\Delta'$ \hfill (7) rule on (6) \\
$\dz \Delta; \Xi; \Delta_1; 1 \with (A \lolli B \otimes \comp A \lolli B), \Omega \rightarrow \Xi'; \Delta'$ \hfill (8) rule application on (7) \\
$\dz \Delta; \Xi; \Delta_1; \comp A \lolli B, \Omega \rightarrow \Xi'; \Delta'$ (9) rule application on (8) \\
\item Comprehension succeeds:
$\Delta = \Delta'', \Xi''$ \hfill (4) from theorem \\
$\mo \Delta''; \Xi''; \cdot ; B; (\done \Delta; \Xi; \Delta_1; \comp A \lolli B, \Omega) \rightarrow \Xi'; \Delta'$ \hfill (5) from theorem \\
$\mz \Xi''; A \rightarrow \Xi''$ \hfill (6) from theorem \\
$\done \Delta''; \Xi''; \cdot; B; (\done \Delta; \Xi; \Delta_1; \comp A \lolli B, \Omega) \rightarrow \Xi'; \Delta'$ \hfill (7) inversion of (5) \\
$\done \Delta''; \Xi'', \Xi; \Delta_1; B, \comp A \lolli B, \Omega; \cdot \rightarrow \Xi'; \Delta'$ \hfill (8) using theorem "Comprehension head is another derivation theorem" on (7) \\
$\dz \Delta''; \Xi''; \Xi; \Delta_1; B, \comp A \lolli B, \Omega \rightarrow \Xi'; \Delta'$ \hfill (9) by i.h. on (8) \\
$\dz \Xi'', \Delta''; \Xi'; \Delta_1; A \lolli B, \comp A \lolli B, \Omega \rightarrow \Xi'; \Delta'$ \hfill (10) using rule on (8) and (6) \\
$\dz \Xi'', \Delta''; \Xi'; \Delta_1; (A \lolli B) \otimes (\comp A \lolli B), \Omega \rightarrow \Xi'; \Delta'$ \hfill (11) using rule on (10) \\
$\dz \Xi'', \Delta''; \Xi'; \Delta_1; 1 \with ((A \lolli B) \otimes (\comp A \lolli B)), \Omega \rightarrow \Xi'; \Delta'$ \hfill (12) rule on (11) \\
$\dz \Xi'', \Delta''; \Xi'; \Delta_1; \comp A \lolli B, \Omega \rightarrow \Xi'; \Delta'$ \hfill (13) rule on (12) \\

\end{itemize}
\end{itemize}

\subsection{Soundness Theorem}

If $\ao \Delta; R; (\cdot; \Delta) \rightarrow \Xi'; \Delta'$ then \\
      $\az \Delta; R \rightarrow \Xi'; \Delta'$

Case by case analysis:

\begin{itemize}
\item $R = A \lolli B$

$\ao \Delta; A \lolli B ; (\cdot ; \Delta) \rightarrow \Xi'; \Delta'$ \hfill (1) assumption \\
$\mo \Delta_1, \Delta_2; \cdot; A ; B ; (\cdot ; \Delta) \rightarrow \Xi'; \Delta'$ \hfill (2) inversion of (1) \\
$\mo \Delta_2; \Delta_1; \cdot ; B ; (\cdot ; \Delta) \rightarrow \Xi'; \Delta'$ \hfill (3) using theorem "Low level matching gives high level matching theorem" on (2) \\
$\mz \Delta_1 ; A \rightarrow \Delta_1$ \hfill (4) from same theorem on (2) \\
$\do \Delta_2; \Delta_1; \cdot ; B ; (\cdot ; \Delta) \rightarrow \Xi'; \Delta'$ \hfill (5) inversion of (3) \\
$\dz \Delta_2; \Delta_1; \cdot ; B \rightarrow \Xi'; \Delta'$ \hfill (6) derive soundness on (5) \\
$\az \Delta_1, \Delta_2; A \lolli B \rightarrow \Xi'; \Delta'$ \hfill (7) using rule on (6) and (4) \\
\end{itemize}

\section{Improved Simplified System}

\newcommand{\mc}{\m{mc} \;}

\subsection{High Level System}

\[
\infer[]
{\mz \Gamma; \cdot ; 1 \rightarrow 1}
{}
\]

\[
\infer[]
{\mz \Gamma; p ; p \rightarrow p}
{}
\tab
\infer[]
{\mz \Gamma, \bang p; \bang p \rightarrow \cdot}
{}
\]

\[
\infer[]
{\mz \Gamma; \Delta_1, \Delta_2 ; A \otimes B \rightarrow \Xi_1, \Xi_2}
{\mz \Gamma; \Delta_1 ; A \rightarrow \Xi_1 & \mz \Gamma; \Delta_2 ; B \rightarrow \Xi_2}
\]

\[
\infer[]
{\dz \Gamma ; \Delta ; \Xi ; \Delta_1 ; \Gamma_1; p, \Omega \rightarrow \Xi' ; \Delta'; \Gamma'}
{\dz \Gamma ; \Delta ; \Xi ; p, \Delta_1 ; \Gamma_1 ; \Omega \rightarrow \Xi' ; \Delta'; \Gamma'}
\tab
\infer[]
{\dz \Gamma; \Delta; \Xi; \Delta_1; \Gamma_1; 1, \Omega \rightarrow \Xi';\Delta';\Gamma'}
{\dz \Gamma; \Delta; \Xi; \Delta_1; \Gamma_1; \Omega \rightarrow \Xi';\Delta';\Gamma'}
\]

\[
\infer[]
{\dz \Gamma; \Delta; \Xi; \Delta_1; \Gamma_1; \bang p, \Omega \rightarrow \Xi'; \Delta'; \Gamma'}
{\dz \Gamma; \Delta; \Xi; \Delta_1; \bang p, \Gamma_1; \Omega \rightarrow \Xi'; \Delta'; \Gamma'}
\]

\[
\infer[]
{\dz \Gamma; \Delta;\Xi;\Delta_1;\Gamma_1; A \otimes B, \Omega \rightarrow \Xi'; \Delta'; \Gamma'}
{\dz \Gamma; \Delta;\Xi;\Delta_1;\Gamma_1; A, B, \Omega \rightarrow \Xi';\Delta'; \Gamma'}
\tab
\]

\[
\infer[]
{\dz \Gamma; \Delta; \Xi; \Delta_1; \Gamma_1; A \with B, \Omega \rightarrow \Xi';\Delta'; \Gamma'}
{\dz \Gamma; \Delta; \Xi; \Delta_1; \Gamma_1; A, \Omega \rightarrow \Xi';\Delta'; \Gamma'}
\tab
\infer[]
{\dz \Gamma; \Delta; \Xi; \Delta_1; \Gamma_1; A \with B, \Omega \rightarrow \Xi';\Delta';\Gamma'}
{\dz \Gamma; \Delta; \Xi; \Delta_1; \Gamma_1; B, \Omega \rightarrow \Xi';\Delta';\Gamma'}
\]

\[
\infer[]
{\dz \Gamma; \Delta ; \Xi; \Delta_1; \Gamma_1; \m{comp} A \lolli B, \Omega \rightarrow \Xi';\Delta';\Gamma'}
{\dz \Gamma; \Delta; \Xi; \Delta_1; \Gamma_1; 1 \with (A \lolli B \otimes \m{comp} A \lolli B), \Omega \rightarrow \Xi';\Delta';\Gamma'}
\]

\[
\infer[]
{\dz \Gamma; \Delta_a, \Delta_b; \Xi; \Delta_1; \Gamma_1; A \lolli B, \Omega \rightarrow \Xi';\Delta';\Gamma'}
{\mz \Gamma; \Delta_a; A \rightarrow \Delta_a & \dz \Gamma; \Delta_b ; \Xi; \Delta_a; \Delta_1; \Gamma_1; B, \Omega \rightarrow \Xi'; \Delta';\Gamma'}
\]

\[
\infer[]
{\dz \Delta; \Xi'; \Delta'; \Gamma'; \cdot \rightarrow \Xi';\Delta'; \Gamma'}
{}
\]

\[
\infer[]
{\az \Gamma; \Delta, \Delta''; A \lolli B \rightarrow \Xi'; \Delta';\Gamma'}
{\mz \Gamma; \Delta; A \rightarrow \Delta & \dz \Gamma; \Delta''; \Delta; \cdot ; \cdot ; B \rightarrow \Xi'; \Delta';\Gamma'}
\]

\[
\infer[]
{\doz \Gamma; \Delta; R, \Phi \rightarrow \Xi';\Delta';\Gamma'}
{\doz \Gamma; \Delta; R \rightarrow \Xi';\Delta';\Gamma'}
\]

\subsection{Low Level System}

\[
\infer[]
{\done \Gamma; \Delta; \Xi; \Delta_1; \Gamma_1; p, \Omega; C \rightarrow \Xi'; \Delta'; \Gamma'}
{\done \Gamma; \Delta; \Xi; p, \Delta_1; \Gamma_1; \Omega; C \rightarrow \Xi'; \Delta'; \Gamma'}
\tab
\infer[]
{\done \Gamma; \Delta; \Xi; \Delta_1; \Gamma_1; 1, \Omega; C \rightarrow \Xi';\Delta';\Gamma'}
{\done \Gamma; \Delta; \Xi; \Delta_1; \Gamma_1; \Omega; C \rightarrow \Xi'; \Delta';\Gamma'}
\]

\[
\infer[]
{\done \Gamma; \Delta; \Xi; \Delta_1; \Gamma_1; \bang p, \Omega; C \rightarrow \Xi'; \Delta'; \Gamma'}
{\done \Gamma; \Delta; \Xi; \Delta_1; \Gamma_1, \bang p; \Omega; C \rightarrow \Xi'; \Delta'; \Gamma'}
\]

\[
\infer[]
{\done \Gamma ; \Delta; \Xi; \Delta_1; \Gamma_1; A \otimes B, \Omega; C \rightarrow \Xi'; \Delta'; \Gamma'}
{\done \Gamma; \Delta; \Xi; \Delta_1; \Gamma_1; A, B, \Omega; C \rightarrow \Xi';\Delta'; \Gamma'}
\]

\[
\infer[]
{\done \Gamma; \Delta; \Xi; \Delta_1; \Gamma_1; \comp A \lolli B, \Omega; \cdot \rightarrow \Xi; \Delta'; \Gamma'}
{\mc \Gamma; \Delta ; \Xi; \Delta_1; \Gamma_1; \cdot; A ; H; (\done \Omega); \cdot \rightarrow \Xi'; \Delta'; \Gamma'}
\]

\[
\infer[ok]
{\mo \Gamma; \Delta, p ; \Xi; p, \Omega; H; C \rightarrow \Xi'; \Delta'; \Gamma'}
{\mo \Gamma; \Delta; \Xi, p; \Omega; H; C \rightarrow \Xi'; \Delta'; \Gamma'}
\tab
\infer[fail]
{\mo \Delta; \Xi; p, \Omega; H; C \rightarrow \Xi'; \Delta'}
{p \notin \Delta & \cont C ; H; \Xi'; \Delta'; \Gamma'}
\]

\[
\infer[\bang ok]
{\mo \Gamma, \bang p; \Delta; \Xi; \bang p, \Omega; H; C \rightarrow \Xi'; \Delta'; \Gamma'}
{\mo \Gamma, \bang p; \Delta; \Xi; \Omega; H; C \rightarrow \Xi'; \Delta'; \Gamma'}
\tab
\infer[\bang fail]
{\mo \Gamma; \Delta; \Xi; \bang p, \Omega; H; C \rightarrow \Xi'; \Delta'; \Gamma'}
{\bang p \notin \Gamma & \cont C ; H; \Xi'; \Delta'; \Gamma'}
\]

\[
\infer[]
{\mo \Gamma; \Delta; \Xi; A \otimes B, \Omega ; H ; C \rightarrow \Xi'; \Delta'; \Gamma'}
{\mo \Gamma; \Delta; \Xi; A, B, \Omega; H; C \rightarrow \Xi';\Delta'; \Gamma'}
\]

\[
\infer[]
{\mo \Gamma; \Delta; \Xi; \cdot ; H; (\Phi; \Delta) \rightarrow \Xi'; \Delta'; \Gamma'}
{\done \Gamma; \Delta; \Xi; \cdot ; \cdot; H; \cdot \rightarrow \Xi'; \Delta'; \Gamma'}
\]

\[
\infer[]
{\cont (\Phi; \Gamma; \Delta); \Xi'; \Delta'; \Gamma'}
{\doo \Gamma; \Delta; \Phi \rightarrow \Xi'; \Delta'; \Gamma'}
\]

\[
\infer[]
{\done \Gamma; \Delta; \Xi; \Delta_1; \Gamma_1; \cdot; \cdot \rightarrow \Xi; \Delta_1; \Gamma_1}
{}
\]

\[
\infer[]
{\ao \Gamma; \Delta; A \lolli B; C \rightarrow \Xi'; \Delta'; \Gamma'}
{\mo \Gamma; \Delta; \cdot; A; B; C \rightarrow \Xi'; \Delta'; \Gamma'}
\]

\[
\infer[]
{\doo \Gamma; \Delta; R, \Phi \rightarrow \Xi'; \Delta'; \Gamma'}
{\ao \Gamma; \Delta; R; (\Phi; \Gamma; \Delta) \rightarrow \Xi';\Delta'; \Gamma'}
\]

$\mc$ rules:

\[
\infer[\otimes]
{\mc \Gamma; \Delta; \Xi; \Delta_1; \Gamma_1; U; A \otimes B, \Omega; H; C; L \rightarrow \Xi'; \Delta'; \Gamma'}
{\mc \Gamma; \Delta; \Xi; \Delta_1; \Gamma_1; U; A, B, \Omega; H; C; L \rightarrow \Xi'; \Delta'; \Gamma'}
\]

\[
\infer[1]
{\mc \Gamma; \Delta; \Xi; \Delta_1; \Gamma_1; U; 1, \Omega; H; C; L \rightarrow \Xi'; \Delta'; \Gamma'}
{\mc \Gamma; \Delta; \Xi; \Delta_1; \Gamma_1; U; \Omega; H; C; L \rightarrow \Xi'; \Delta'; \Gamma'}
\]

\[
\infer[first1 \; p]
{\mc \Gamma; p, \Delta'', \Delta ; \Xi; \Delta_1; \Gamma_1; U; p, \Omega; H; C; \cdot \rightarrow \Xi'; \Delta'; \Gamma'}
{\mc \Gamma; \Delta'', \Delta; \Xi; \Delta_1; \Gamma_1; U, p; \Omega; H; (\comp p, \Omega; p; \Xi; \Delta''; C); (\comp p, \Omega; p; \Xi; \Delta''; C) \rightarrow \Xi'; \Delta'; \Gamma'}
\]

\[
\infer[first2 \; p]
{\mc \Gamma; p, \Delta'', \Delta ; \Xi; \Delta_1; \Gamma_1; U; p, \Omega; H; C; L \rightarrow \Xi'; \Delta'; \Gamma'}
{\mc \Gamma; \Delta'', \Delta; \Xi; \Delta_1; \Gamma_1; U, p; \Omega; H; (\comp p, \Omega; p; \Xi; \Delta''; C); L \rightarrow \Xi'; \Delta'; \Gamma' & L \ne \cdot}
\]

\[
\infer[first \; p \; not \; available]
{\mc \Gamma; \Delta; \Xi; \Delta_1; \Gamma_1; U; p, \Omega; H; C; L \rightarrow \Xi'; \Delta'; \Gamma'}
{\contc \Gamma; \Delta; \Xi; \Delta_1 ; \Gamma_1; U; H; C; L \rightarrow \Xi'; \Delta'; \Gamma' & p \notin \Delta}
\]

\[
\infer[first \; \bang p]
{\mc \Gamma, \bang p, \Gamma_n; \Delta; \Xi; \Delta_1; \Gamma_1; U; \bang p, \Omega; H; C; L \rightarrow \Xi'; \Delta'; \Gamma'}
{\mc \Gamma, \bang p, \Gamma_n; \Delta; \Xi; \Delta_1; \Gamma_1; U; \Omega; H; (\bang \comp \bang p, \Omega; \Xi; \Gamma_n; C); L \rightarrow \Xi'; \Delta'; \Gamma'}
\]

\[
\infer[first \; \bang p \; not \; available]
{\mc \Gamma; \Delta; \Xi; \Delta_1; \Gamma_1; U; \bang p, \Omega; H; C; L \rightarrow \Xi'; \Delta'; \Gamma'}
{\contc \Gamma; \Delta; \Xi; \Delta_1; \Gamma_1; U; H; C; L \rightarrow \Xi'; \Delta'; \Gamma' & \bang p \notin \Gamma}
\]

\[
\infer[finished]
{\mc \Gamma; \Delta; \Xi; \Delta_1; \Gamma_1; U; \cdot ; H; C; L \rightarrow \Xi'; \Delta'; \Gamma'}
{\dc \Gamma; \Delta; \Xi; \Delta_1; \Gamma_1; U; H; H; C; L \rightarrow \Xi'; \Delta'; \Gamma'}
\]

\[
\infer[consumed \; next1]
{\contc \Gamma; p, \Delta; \Xi; \Delta_1; \Gamma_1; U, p ; H ; (\comp p, \Omega; p; \Xi''; p, \Delta_n; C); (\comp p, \Omega; p; \Xi''; p, \Delta_n; C) \rightarrow \Xi'; \Delta'; \Gamma'}
{\mc \Gamma; \Delta; \Xi; \Delta_1; \Gamma_1; U, p; \Omega; H; (\comp p, \Omega; p; \Xi; \Delta_n; C); (\comp p, \Omega; p; \Xi; \Delta_n; C) \rightarrow \Xi'; \Delta'; \Gamma' & \Xi = \Xi'', \Xi_1}
\]

\[
\infer[consumed \; next2]
{\contc \Gamma; p, \Delta; \Xi; \Delta_1; \Gamma_1; U, p ; H ; (\comp p, \Omega; p; \Xi''; p, \Delta_n; C); L \rightarrow \Xi'; \Delta'; \Gamma'}
{\mc \Gamma; \Delta; \Xi; \Delta_1; \Gamma_1; U, p; \Omega; H; (\comp p, \Omega; p; \Xi; \Delta_n; C); L \rightarrow \Xi'; \Delta'; \Gamma' & \Xi = \Xi'', \Xi_1 & L \ne (\comp p, \Omega; p; \Xi''; p, \Delta_n; C)}
\]

\[
\infer[consumed \; next \; no \; choices1]
{\contc \Gamma; \Delta; \Xi; \Delta_1; \Gamma_1; U, p; H; (\comp p, \Omega; p; \Xi''; \cdot ; C); (\comp p, \Omega; p; \Xi''; \cdot; C) \rightarrow \Xi'; \Delta'; \Gamma}
{\contc \Gamma; \Delta; \Xi; \Delta_1; \Gamma_1; U; H; C; C \rightarrow \Xi'; \Delta'; \Gamma'}
\]

\[
\infer[consumed \; next \; no \; choices2]
{\contc \Gamma; \Delta; \Xi; \Delta_1; \Gamma_1; U, p; H; (\comp p, \Omega; p; \Xi''; \cdot ; C); L \rightarrow \Xi'; \Delta'; \Gamma}
{\contc \Gamma; \Delta; \Xi; \Delta_1; \Gamma_1; U; H; C; L \rightarrow \Xi'; \Delta'; \Gamma' & L \ne (\comp p, \Omega; p; \Xi''; \cdot; C)}
\]

\[
\infer[not \; consumed \; next1]
{\contc \Gamma; \Delta, p; \Xi; \Delta_1; \Gamma_1; U, p; H; (\comp p, \Omega; p; \Xi''; p, \Delta_n; C); (\comp p, \Omega; p; \Xi''; p, \Delta_n; C) \rightarrow \Xi'; \Delta'; \Gamma'}
{\mc \Gamma; \Delta, p; \Xi; \Delta_1; \Gamma_1; U, p; \Omega; H; (\comp p, \Omega; p; \Xi; \Delta_n; C); (\comp p, \Omega; p; \Xi; \Delta_n; C) \rightarrow \Xi'; \Delta'; \Gamma' & \Xi = \Xi''}
\]

\[
\infer[not \; consumed \; next2]
{\contc \Gamma; \Delta, p; \Xi; \Delta_1; \Gamma_1; U, p; H; (\comp p, \Omega; p; \Xi''; p, \Delta_n; C); L \rightarrow \Xi'; \Delta'; \Gamma'}
{\mc \Gamma; \Delta, p; \Xi; \Delta_1; \Gamma_1; U, p; \Omega; H; (\comp p, \Omega; p; \Xi; \Delta_n; C); L \rightarrow \Xi'; \Delta'; \Gamma' & \Xi = \Xi'' & L \ne (\comp p, \Omega; p; \Xi''; p, \Delta_n; C)}
\]

\[
\infer[not \; consumed \; no \; choices1]
{\contc \Gamma; \Delta; \Xi; \Delta_1; \Gamma_1; U, p; H; (\comp p, \Omega ; p ; \Xi''; \cdot ; C); (\comp p, \Omega; p; \Xi''; \cdot; C) \rightarrow \Xi'; \Delta'; \Gamma'}
{\contc \Gamma; \Delta, p; \Xi; \Delta_1; \Gamma_1; U; H; C; C \rightarrow \Xi'; \Delta'; \Gamma' & \Xi = \Xi''}
\]

\[
\infer[not \; consumed \; no \; choices2]
{\contc \Gamma; \Delta; \Xi; \Delta_1; \Gamma_1; U, p; H; (\comp p, \Omega ; p ; \Xi''; \cdot ; C); L \rightarrow \Xi'; \Delta'; \Gamma'}
{\contc \Gamma; \Delta, p; \Xi; \Delta_1; \Gamma_1; U; H; C; L \rightarrow \Xi'; \Delta'; \Gamma' & \Xi = \Xi'' & L \ne (\comp p, \Omega; p; \Xi''; \cdot; C)}
\]

\[
\infer[\bang next]
{\contc \Gamma; \Delta; \Xi; \Delta_1; \Gamma_1; U; H; (\bang \comp \bang p, \Omega; \Xi''; \bang p, \Gamma_n; C); L \rightarrow \Xi'; \Delta'; \Gamma'}
{\mc \Gamma; \Delta; \Xi; \Delta_1; \Gamma_1; U; \Omega; (\bang \comp \bang p, \Omega; \Xi; \Gamma_n; C); L \rightarrow \Xi'; \Delta'; \Gamma'}
\]


\[
\infer[\bang no \; options]
{\contc \Gamma; \Delta; \Xi; \Delta_1; \Gamma_1; U; H; (\bang \comp \bang p, \Omega; \Xi''; \cdot; C); L \rightarrow \Xi'; \Delta'; \Gamma'}
{\contc \Gamma; \Delta; \Xi; \Delta_1; \Gamma_1; U; H; C; L \rightarrow \Xi'; \Delta'; \Gamma'}
\]

\[
\infer[no \; more \; options]
{\contc \Gamma; \Delta; \Xi; \Delta_1;\Gamma_1; U; H; (\done \Omega); L \rightarrow \Xi' ; \Delta'; \Gamma'}
{\done \Gamma; \Delta; \Xi; \Delta_1; \Gamma_1; \Omega ; \cdot \rightarrow \Xi'; \Delta'; \Gamma'}
\]

\[
\infer[derive \; p]
{\dc \Gamma; \Delta; \Xi; \Delta_1; \Gamma_1; U; p, \Omega; H; C; L \rightarrow \Xi'; \Delta'; \Gamma'}
{\dc \Gamma; \Delta; \Xi; \Delta_1, p; \Gamma_1; U; \Omega; H; C; L \rightarrow \Xi'; \Delta'; \Gamma'}
\tab
\infer[derive \; 1]
{\dc \Gamma; \Delta; \Xi; \Delta_1; \Gamma_1; U; 1, \Omega; H; C; L \rightarrow \Xi'; \Delta'; \Gamma'}
{\dc \Gamma; \Delta; \Xi; \Delta_1; \Gamma_1; U; \Omega; H; C; L \rightarrow \Xi'; \Delta'; \Gamma'}
\]

\[
\infer[derive \; \bang p]
{\dc \Gamma; \Delta; \Xi; \Delta_1; \Gamma_1; U; \bang p, \Omega; H; C; L \rightarrow \Xi'; \Delta'; \Gamma'}
{\dc \Gamma; \Delta; \Xi; \Delta_1; \bang p, \Gamma_1; U; \Omega; H; C; L \rightarrow \Xi'; \Delta'; \Gamma'}
\]

\[
\infer[derive \; A \otimes B]
{\dc \Gamma'; \Delta; \Xi; \Delta_1; \Gamma_1; U; A \otimes B, \Omega; H; C; L \rightarrow \Xi'; \Delta'; \Gamma'}
{\dc \Gamma'; \Delta; \Xi; \Delta_1; \Gamma_1; U; A, B, \Omega; H; C; L \rightarrow \Xi'; \Delta'; \Gamma'}
\]

\[
\infer[derive \; end \; linear]
{\dc \Gamma; \Delta; \Xi; \Delta_1; \Gamma_1; U; \cdot; H; C ; L \rightarrow \Xi'; \Delta'; \Gamma'}
{\contc \Gamma; \Delta; \Xi; \Delta_1; \Gamma_1; U; H; L; L \rightarrow \Xi'; \Delta'; \Gamma' & L = (\comp p, \Omega;p;\Omega;\Xi''; \Delta_n; L') }
\]

\[
\infer[derive \; end \; not \; linear]
{\dc \Gamma; \Delta; \Xi; \Delta_1; \Gamma_1; U; \cdot; H; C ; \cdot \rightarrow \Xi'; \Delta'; \Gamma'}
{\contc \Gamma; \Delta; \Xi; \Delta_1; \Gamma_1; U; H; C; \cdot \rightarrow \Xi'; \Delta'; \Gamma' }
\]
































\begin{comment}
\section{Soundness}

If $\Psi; \Theta ; \Phi ; \Gamma ; \Delta \hookrightarrow \Gamma' ; \Delta'; \Xi'$ then $\Psi; \Theta, \Phi ; \Gamma; \Delta \Rightarrow \Gamma' ; \Delta' ; \Xi'$.


If $\m{apply1} \; \Psi; \Gamma ; \Delta; [R]; (\m{rule}; \cdot; \cdot; \Delta) \rightarrow \Gamma'; \Delta'; \Xi'$ then $\m{apply} \; \Psi; \Gamma; \Delta, [R] \rightarrow \Gamma'; \Delta'; \Xi'$.

Induction on the first judgment:

\begin{description}

\item[Forall Case:]

$\m{apply1} \; \Psi; \Gamma; \Delta; [\forall X : \tau. A]; (\m{rule}; ...) \rightarrow \Gamma'; \Delta'; \Xi'$ \hfill (1) assumption \\
$\m{apply1} \; \Psi; \Gamma; \Delta; [A\{M/X\}]; (\m{rule}; ...) \rightarrow \Gamma'; \Delta' ; \Xi' $ \hfill (2) inversion of (1) \\
$\val{M}{\tau}$   \hfill (3) inversion of (1) \\
$\m{apply} \; \Psi; \Gamma; \Delta, [A\{M/X\}] \rightarrow \Gamma'; \Delta'; \Xi'$ \hfill (4) i.h. on (2) \\
$\m{apply} \; \Psi ; \Gamma; \Delta, [\forall X: \tau. A] \rightarrow \Gamma'; \Delta'; \Xi'$ \hfill (5) rule on (4) and (3) \\

\item[Lolli Case:]

$\m{apply1} \; \Psi ; \Gamma; \Delta; [A \lolli B]; (\m{rule} ; ...) \rightarrow \Gamma'; \Delta'; \Xi'$ \hfill (1) assumption \\
$\m{matchbody} \; \Psi; \Gamma; \Delta; \cdot ; A ; \cdot ; B ; (\cdot, (\m{rule}; ...)) \rightarrow \Gamma'; \Delta'; \Xi'$ \hfill (2) inversion of (1) \\
$\m{derive1} \; \Psi; \Gamma; \Delta_2; \cdot; \cdot; B; (\cdot, (\m{rule}; ...)) \rightarrow \Gamma'; \Delta'; \Xi'$ \hfill (3) some theorem on (2) \\
$\m{derive} \; \Psi; \Gamma; \Delta_2; B; \cdot; \cdot \rightarrow \Gamma'; \Delta'; (\Xi' - \cdot)$ \hfill (4) derive theorem on (3) \\
$\m{match} \; \Psi; \Gamma; \Delta_1 \rightarrow [A]; \Delta_1$ \hfill (5) theorem on (2), $\Delta_1 \subset \Xi'$ \\
$\m{apply} \; \Psi; \Gamma; \Delta_1, \Delta_2 ; [A \lolli B] \rightarrow \Gamma'; \Delta'; \Xi'$ \hfill (6) using (4) and (5) \\
\end{description}


\subsection{Derive theorem}

If $\m{derive1} \; \Psi; \Gamma; \Delta; \Xi; \Gamma_1 ; \Delta_1; \Omega ; \cdot \rightarrow \Gamma' ; \Delta' ; \Xi'$ then $\m{derive} \; \Psi ; \Gamma ; \Delta; \Omega; \Delta_1; \Gamma_1 \rightarrow \Gamma'; \Delta'; (\Xi' - \Xi)$. \\

Induction on the first judgment.

\begin{description}
\item[Case:] $\m{derive1} \; \Psi ; \Gamma; \Delta; \Xi; \Gamma_1 ; \Delta_1 ; A \otimes B, \Omega ; \cdot \rightarrow \Gamma' ; \Delta' ; \Xi'$

$\m{derive1} \; \Psi ; \Gamma ; \Delta; \Xi; \Gamma_1; \Delta_1 ; A, B, \Omega ; \cdot \rightarrow \Gamma' ; \Delta' ; \Xi'$ \hfill (1) inversion \\
$\m{derive} \; \Psi ; \Gamma ; \Delta ; A, B, \Omega ; \Delta_1 ; \Gamma_1 \rightarrow \Gamma'; \Delta'; (\Xi' - \Xi)$ \hfill (2) i.h. on (1) \\
$\m{derive} \; \Psi ; \Gamma ; \Delta ; A \otimes B, \Omega ; \Delta_1 ; \Gamma_1 \rightarrow \Gamma'; \Delta' ; (\Xi' - \Xi)$ \hfill rule applied to (2) \\

\item[Case:] $\m{derive1} \; \Psi ; \Gamma ; \Delta; \Xi; \Gamma_1 ; \Delta_1; \exists X:\m{addr}, \Omega; \cdot \rightarrow \Gamma'; \Delta' ; \Xi'$

Same as the previous case.

\item[Case:] $\m{derive1} \; \Psi ; \Gamma ; \Delta; \Xi; \Gamma_1 ; \Delta_1; 1, \Omega; \cdot \rightarrow \Gamma'; \Delta' ; \Xi'$

Same as the previous case.

\item[Case:] $\m{derive1} \; \Psi ; \Gamma ; \Delta; \Xi; \Gamma_1 ; \Delta_1; \fact{name}{e_1}{e_2, ..., e_n}, \Omega; \cdot \rightarrow \Gamma'; \Delta' ; \Xi'$

Same as the previous case.

\item[Case:] $\m{derive1} \; \Psi ; \Gamma ; \Delta; \Xi; \Gamma_1 ; \Delta_1; \bang\fact{name}{e_1}{e_2, ..., e_n}, \Omega; \cdot \rightarrow \Gamma'; \Delta' ; \Xi'$

Same as the previous case.

\item[Case:] $\m{derive1} \; \Psi ; \Gamma; \Delta ; \Xi; \Gamma_1; \Delta_1 ; \cdot ; \cdot \rightarrow \Gamma_1; \Delta_1; \Xi$

$\m{derive} \; \Psi; \Gamma; \Delta; \cdot ; \Delta_1; \Gamma_1 \rightarrow \Gamma_1; \Delta_1; (\Xi - \Xi)$ \hfill using axiom \\


\end{description}
\end{comment}


\end{document}

